%%%%%%%%%%%%%%%%%%%%%%%%%%%%%%%%%%%%%%%%%%%%%%%%%%%%%%%%%%%%%%%%%%%%%%%%%%%%%%
\begin{problem}[问题3.2]
推导不可压缩流体流动的柱坐标方程.
\end{problem}
\begin{solution}
%\setlength{\columnseprule}{0.4pt}
%\begin{multicols}{2}
\textbf{解:}分别推导不可压缩流体流动在柱坐标下的连续性方程, 动量方程及能量方程:

\vspace{0.75em}
\noindent\textbf{连续性方程}
\vspace{0.75em}

\noindent 不可压缩流体流动连续性方程的一般形式为$\nabla\cdot\mathbf{v} = 0$, 代入柱坐标:
\[
\frac{1}{R}\frac{\partial }{\partial R}(Rv_R)
+ \frac{1}{R}\frac{\partial v_{\varphi}}{\partial \varphi} + \frac{\partial v_z}{\partial z} = 0
\]
因此柱坐标下的连续性方程为
\[
\frac{\partial (Rv_R)}{\partial R}+
\frac{\partial v_{\varphi}}{\partial \varphi} + R\frac{\partial v_z}{\partial z} = 0
\]

\vspace{0.75em}
\noindent\textbf{动量方程}
\vspace{0.75em}

\noindent 不可压缩流体流动动量方程的一般形式为
\[
\underbrace{\frac{d\mathbf{v}}{dt}\vphantom{\frac{1}{\rho}}}_{1} = \underbrace{- \frac{1}{\rho}\nabla p}_{2} + \underbrace{\mathbf{F}\vphantom{\frac{1}{\rho}}}_{3} +\underbrace{\mu\nabla^2\mathbf{v}\vphantom{\frac{1}{\rho}}}_{4}
\]
设柱坐标中的单位矢量为$\mathbf{n}_R$, $ \mathbf{n}_\varphi$, $\mathbf{n}_z$, 现分别求上式中的4项在柱坐标下的形式:

\begin{enumerate}
\item 在柱坐标$\mathbf{v} =\mathbf{v}_R+\mathbf{v}_\varphi + \mathbf{v}_z =v_R\mathbf{n}_R + v_\varphi \mathbf{n}_\varphi+v_z\mathbf{n}_z$, 因此第一项有
{\setlength\arraycolsep{1pt}
\begin{eqnarray}
\frac{d\mathbf{v}}{dt} & = & \frac{d}{dt}\Big(v_R\mathbf{n}_R+v_\varphi \mathbf{n}_\varphi+v_z\mathbf{n}_z\Big)
\nonumber\\
& = & \frac{dv_R}{dt}\mathbf{n}_R + v_R\frac{d\mathbf{n}_R}{dt}
+ \frac{dv_\varphi}{dt}\mathbf{n}_\varphi + v_\varphi\frac{d\mathbf{n}_\varphi}{dt}
+ \frac{dv_z}{dt}\mathbf{n}_z + v_z\frac{d\mathbf{n}_z}{dt}
\nonumber
\end{eqnarray}}
将上式中各项在柱坐标下展开:
\begin{multicols}{2}
\setlength{\abovedisplayskip}{-5pt}
\begin{eqnarray}
v_R\frac{d\mathbf{n}_R}{dt} & = & v_R\Big[\frac{\partial\mathbf{n}_R}{\partial t} + (\mathbf{v}\cdot \nabla)\mathbf{n}_R\Big]\nonumber\\
 & = &  v_R(\mathbf{v}\cdot \nabla)\mathbf{n}_R
\nonumber\\
& = & v_R\Big[
v_R\frac{\partial \mathbf{n}_R}{\partial R} + \frac{v_\varphi}{R}\frac{\partial\mathbf{n}_R}{\partial\varphi}
+ v_z\frac{\partial\mathbf{n}_R}{\partial z}
\Big]\nonumber\\
& = &  \frac{v_Rv_\varphi}{R}\mathbf{n}_\varphi
\nonumber\\
\nonumber\\
v_\varphi\frac{d\mathbf{n}_\varphi}{dt}
& = & v_\varphi\Big[\frac{\partial\mathbf{n}_\varphi}{\partial t} + (\mathbf{v}\cdot \nabla)\mathbf{n}_\varphi\Big]\nonumber\\
 & = &  v_\varphi(\mathbf{v}\cdot \nabla)\mathbf{n}_\varphi
\nonumber\\
& = &
v_\varphi\Big[
v_R\frac{\partial \mathbf{n}_\varphi}{\partial R} + \frac{v_\varphi}{R}\frac{\partial\mathbf{n}_\varphi}{\partial\varphi}
+ v_z\frac{\partial\mathbf{n}_\varphi}{\partial z}
\Big]
\nonumber\\
& = & -\frac{v_\varphi ^2}{R}\mathbf{n}_R
\nonumber\\
\nonumber\\
%
v_z\frac{d\mathbf{n}_z}{dt} & = &
v_z\Big[\frac{\partial\mathbf{n}_z}{\partial t} + (\mathbf{v}\cdot \nabla)\mathbf{n}_z\Big]\nonumber\\
 & = &  v_z(\mathbf{v}\cdot \nabla)\mathbf{n}_z
\nonumber\\
& = &
v_z\Big[
v_R\frac{\partial \mathbf{n}_z}{\partial R} + \frac{v_\varphi}{R}\frac{\partial\mathbf{n}_z}{\partial \varphi}
+ v_z\frac{\partial\mathbf{n}_z}{\partial z}
\Big]
\nonumber\\
& = & 0
\nonumber
\end{eqnarray}

\begin{eqnarray}
\frac{d v_R}{dt}\mathbf{n}_R & = &
   \Big[
       \frac{\partial v_R}{\partial t} + (\mathbf{v}\cdot\nabla)v_R
   \Big]\mathbf{n}_R\nonumber\\
   & = &
   \Big[
       \frac{\partial v_R}{\partial t} +
       v_R\frac{\partial v_R}{\partial R} +\nonumber\\
   & + & \frac{v_\varphi}{R}\frac{\partial v_R}{\partial\varphi}+v_z\frac{\partial v_R}{\partial z}
   \Big]\mathbf{n}_R
      \nonumber\\
   \nonumber\\
   \nonumber\\
      \frac{d v_\varphi}{dt}\mathbf{n}_\varphi & = &
   \Big[
       \frac{\partial v_\varphi}{\partial t} + (\mathbf{v}\cdot\nabla)v_\varphi
   \Big]\mathbf{n}_\varphi\nonumber\\
   & = &
   \Big[
       \frac{\partial v_\varphi}{\partial t} +
       v_R\frac{\partial v_\varphi}{\partial R} +\nonumber\\
   & + & \frac{v_\varphi}{R}\frac{\partial v_\varphi}{\partial\varphi}+v_z\frac{\partial v_\varphi}{\partial z}
   \Big]\mathbf{n}_\varphi
   \nonumber\\
   \nonumber\\
   \nonumber\\
   \frac{d v_z}{dt}\mathbf{n}_z& = &
   \Big[
       \frac{\partial v_z}{\partial t} + (\mathbf{v}\cdot\nabla)v_\theta
   \Big]\mathbf{n}_z\nonumber\\
   & = &
   \Big[
       \frac{\partial v_z}{\partial t} +
       v_R\frac{\partial v_z}{\partial R} +\nonumber\\
   & + & \frac{v_\varphi}{R}\frac{\partial v_z}{\partial\varphi}+v_z\frac{\partial v_z}{\partial z}
   \Big]\mathbf{n}_z
   \nonumber\\
   \nonumber
\end{eqnarray}
\end{multicols}

上三式的推导中, 用到了$\partial\mathbf{n}_R/\partial R = \partial\mathbf{n}_\varphi/\partial R=\partial\mathbf{n}_R/\partial z =  \partial\mathbf{n}_\varphi/\partial z= 0$, $\partial\mathbf{n}_R/\partial\varphi = \mathbf{n}_\varphi$, $\partial\mathbf{n}_\varphi/\partial\varphi = -\mathbf{n}_R$等结论. 最终可以得到
{\setlength\arraycolsep{2pt}
\begin{eqnarray}
\frac{d\mathbf{v}}{dt}
& = & \frac{dv_R}{dt}\mathbf{n}_R
+ \frac{dv_\varphi}{dt}\mathbf{n}_\varphi + \frac{dv_z}{dt}\mathbf{n}_z + \frac{v_Rv_\varphi}{R}\mathbf{n}_\varphi-\frac{v_\varphi ^2}{R}\mathbf{n}_R \nonumber\\
& = &
\Big(\frac{\partial v_R}{\partial t} +
       v_R\frac{\partial v_R}{\partial R} +
    +  \frac{v_\varphi}{R}\frac{\partial v_R}{\partial\varphi}+v_z\frac{\partial v_R}{\partial z} -\frac{v_\varphi ^2}{R}\Big)\mathbf{n}_R + \nonumber\\
& + &
\Big(\frac{\partial v_\varphi}{\partial t} +
       v_R\frac{\partial v_\varphi}{\partial R}
    +  \frac{v_\varphi}{R}\frac{\partial v_\varphi}{\partial\varphi}+v_z\frac{\partial v_\varphi}{\partial z}+\frac{v_Rv_\varphi}{R}\Big)\mathbf{n}_\varphi + \nonumber\\
& +& \Big(\frac{\partial v_z}{\partial t} +
       v_R\frac{\partial v_z}{\partial R} +
    +  \frac{v_\varphi}{R}\frac{\partial v_z}{\partial\varphi}+v_z\frac{\partial v_z}{\partial z}\Big)\mathbf{n}_z
\end{eqnarray}}

\item 对于第二项$-1/\rho\nabla p$有
\begin{equation}
-\frac{1}{\rho}\nabla p =
-\frac{1}{\rho}\frac{\partial p}{\partial R}\mathbf{n}_R
-\frac{1}{\rho}\frac{1}{R}\frac{\partial p}{\partial\varphi}\mathbf{n}_\varphi
-\frac{1}{\rho}\frac{\partial p}{\partial z}\mathbf{n}_z
\end{equation}
\item 对于第三项$\mathbf{F}$, 可表示成柱坐标三个方向上的分量和
\begin{equation}
\mathbf{F} = \mathbf{F}_R + \mathbf{F}_\varphi + \mathbf{F}_z = F_R\mathbf{n}_R + F_\varphi\mathbf{n}_\varphi + F_z\mathbf{n}_z
\end{equation}

%%%%%%%%%%%%%%%%%%%%%%%%%%%%%%%%%%%%%%%%
\item 由$\nabla^2\mathbf{v}=\Delta\mathbf{v}$,可知第四项有
\[
\nabla^2\mathbf{v}= \Big(
\Delta v_R - \frac{v_R}{R^2} - \frac{2}{R^2}\frac{\partial v_\varphi}{\partial\varphi}
\Big)\mathbf{n}_R
+\Big(
\Delta v_\varphi + \frac{2}{R^2}\frac{\partial v_R}{\partial\varphi}-\frac{v_\varphi}{R^2}
\Big)\mathbf{n}_\varphi
+\Delta v_z\mathbf{n}_z
\]

\noindent 其中$\Delta = \frac{\partial}{\partial R^2} +
\frac{1}{R^2}\frac{\partial^2}{\partial\varphi^2}+
\frac{\partial^2}{\partial z^2}+
\frac{1}{R}\frac{\partial}{\partial R}$
%%%%%%%%%%%%%%%%%%%%%%%%%%%%%%%%%%%%%%%%

%\item 由于$\nabla^2\mathbf{v} = \nabla(\nabla\cdot\mathbf{v}) - \nabla\times\nabla\times\mathbf{v}$, 因此求第四项可化为分别求$\mu\nabla(\nabla\cdot\mathbf{v})$和 $\mu\nabla\times\nabla\times\mathbf{v}$
%{\setlength\arraycolsep{2pt}
%\begin{eqnarray}\label{eq1q}
%   \nabla(\nabla\cdot\mathbf{v}) & = &\nabla\Big(
%\frac{1}{R}\frac{\partial}{\partial R}(Rv_R) + \frac{1}{R}\frac{\partial v_\varphi}{\partial\varphi} + \frac{\partial v_z}{\partial z} \Big)\notag \\
% & = &\Big(\frac{\partial}{\partial R}\mathbf{n}_R +\frac{1}{R}\frac{\partial}{\partial\varphi}\mathbf{n}_\varphi + \frac{\partial}{\partial z}\mathbf{n}_z\Big)\Big(
%\frac{1}{R}\frac{\partial}{\partial R}(Rv_R) + \frac{1}{R}\frac{\partial v_\varphi}{\partial\varphi} + \frac{\partial v_z}{\partial z} \Big)\notag \\
% & = &\Big(
%\frac{\partial^2V_R}{\partial R^2} - \frac{1}{R^2}\frac{\partial v_\varphi}{\partial\varphi} + \frac{1}{R}\frac{\partial^2v_\varphi}{\partial\varphi\partial R} +\frac{\partial^2v_z}{\partial\varphi\partial z} - \frac{v_R}{R^2} + \frac{1}{R}\frac{\partial v_R}{\partial\varphi}
%\Big)\mathbf{n}_R + \notag \\
%& + &\Big(
%\frac{1}{R}\frac{\partial^2v_R}{\partial\varphi\partial R}
%+\frac{1}{R^2}\frac{\partial^2v_\varphi}{\partial\varphi^2}
%+\frac{1}{R}\frac{\partial^2v_z}{\partial\varphi\partial z}
%+\frac{1}{R^2}\frac{\partial v_R}{\partial\varphi}
% \Big)\mathbf{n}_\varphi +\notag \\
%& + &\Big(
%\frac{\partial^2v_R}{\partial z\partial R}
%+\frac{1}{R}\frac{\partial^2v_\varphi}{\partial z\partial\varphi}
%\frac{\partial^2v_z}{\partial^2z}
%+\frac{1}{R}\frac{\partial v_R}{\partial z}
% \Big)\mathbf{n}_z\notag\\
%& & \notag\\
%%
%%
%\nabla\times\nabla\times\mathbf{v} & = & \nabla\times
%\Big[
%\big(\frac{1}{R}\frac{\partial v_z}{\partial\varphi} - \frac{\partial v_\varphi}{\partial z}\big)\mathbf{n}_R
%+\big(\frac{\partial v_R}{\partial z} - \frac{\partial v_z}{\partial R}\big)\mathbf{n}_\varphi
%+\frac{1}{R}\big(\frac{\partial}{\partial R}(Rv_\varphi) - \frac{\partial v_R}{\partial\varphi}\big)\mathbf{n}_z
%\Big]
%\notag\\
%& = & \Big(\frac{1}{R^2}\frac{\partial v_\varphi}{\partial\varphi} +
%      \frac{1}{R}\frac{\partial^2v_\varphi}{\partial\varphi\partial R}
%      - \frac{1}{R^2}\frac{\partial^2 v_R}{\partial\varphi^2}
%      -\frac{\partial^2 v_R}{\partial z^2}
%      +\frac{\partial^2 v_z}{\partial z\partial R}\Big)\mathbf{n}_R + \notag\\
%& + & \Big(
%      \frac{1}{R}\frac{\partial^2 v_z}{\partial z\partial\varphi} -
%      \frac{\partial^2v_\varphi}{\partial z^2}
%      + \frac{v_\varphi}{R^2}
%      -\frac{1}{R}\frac{\partial v_\varphi}{\partial R}
%      -\frac{\partial^2 v_\varphi}{\partial R^2}
%      -\frac{1}{R^2}\frac{\partial v_R}{\partial\varphi}
%      +\frac{1}{R}\frac{\partial^2 v_R}{\partial\varphi\partial R}
%      \Big)\mathbf{n}_\varphi + \notag\\
%& + & \Big(
%      -\frac{1}{R^2}\frac{\partial^2 v_z}{\partial\varphi^2} +
%      \frac{1}{R}\frac{\partial^2v_\varphi}{\partial\varphi\partial z}
%      + \frac{1}{R}\frac{\partial v_R}{\partial z}
%      +\frac{\partial^2 v_R}{\partial R\partial z}
%      -\frac{1}{R}\frac{\partial v_z}{\partial R}
%      -\frac{\partial^2 v_z}{\partial R^2}
%      \Big)\mathbf{n}_z
%\end{eqnarray}}
\end{enumerate}
根据式(1-4), 可写出$\mathbf{n}_R$, $\mathbf{n}_\varphi$, $\mathbf{n}_z$各方向上的动量方程
{\setlength\arraycolsep{2pt}
\begin{eqnarray}
\frac{\partial v_R}{\partial t}  +
v_R\frac{\partial v_R}{\partial R} +
\frac{v_\varphi}{R}\frac{\partial v_R}{\partial\varphi}+
v_z\frac{\partial v_R}{\partial z} -
\frac{v_\varphi^2}{R}
& = &
-\frac{1}{\rho}\frac{\partial p}{\partial R} + F_R + \mu\Big(
\Delta v_R-
\frac{v_R}{R^2}-
\frac{2}{R^2}\frac{\partial v_\varphi}{\partial\varphi}
\Big)\nonumber\\
\nonumber\\
\frac{\partial v_\varphi}{\partial t}+
v_R\frac{\partial v_\varphi}{\partial R}+
\frac{v_\varphi}{R}\frac{\partial v_\varphi}{\partial\varphi}+
v_z\frac{\partial v_\varphi}{\partial z} +
\frac{v_Rv_\varphi}{R}
& = &
-\frac{1}{\rho R}\frac{\partial p}{\partial\varphi} + F_\varphi + \mu\Big(
\Delta v_\varphi-
\frac{v_\varphi}{R^2}+
\frac{2}{R^2}\frac{\partial v_R}{\partial\varphi}
\Big)\nonumber\\
\nonumber\\
\frac{\partial v_z}{\partial t}+
v_R\frac{\partial v_z}{\partial R}+
\frac{v_\varphi}{R}\frac{\partial v_z}{\partial\varphi}+
v_z\frac{\partial v_z}{\partial z}
& = &
-\frac{1}{\rho}\frac{\partial p}{\partial z} + F_z + \mu\Delta v_z\nonumber
\end{eqnarray}}

\noindent 其中$\Delta = \frac{\partial}{\partial R^2} +
\frac{1}{R^2}\frac{\partial^2}{\partial\varphi^2}+
\frac{\partial^2}{\partial z^2}+
\frac{1}{R}\frac{\partial}{\partial R}$

\vspace{0.75em}
\noindent\textbf{能量方程}
\vspace{0.75em}

\noindent 能量方程的一般形式如下
\[
\frac{\partial e}{\partial t} +\mathbf{v}\cdot\nabla e = \frac{1}{\rho}\nabla\cdot(k\nabla T) + \Phi + \dot{q}
\]
其中$\Phi$为耗散函数. 各项在柱坐标下有
\[
\mathbf{v}\cdot\nabla e = (v_R\mathbf{n}_R + v_\varphi\mathbf{n}_\varphi + v_z\mathbf{n}_z)\cdot\Big(\frac{\partial e}{\partial R}\mathbf{n}_R
+ \frac{1}{R}\frac{\partial e}{\partial\varphi}\mathbf{n}_\varphi
+ \frac{\partial e}{\partial z}\Big)\mathbf{n}_z
=
v_R\frac{\partial e}{\partial R}
+ \frac{v_\varphi}{R}\frac{\partial e}{\partial\varphi}
+ v_z\frac{\partial e}{\partial z}
\]
\[
\frac{1}{\rho}\nabla\cdot(k\nabla T) = \frac{k}{\rho}\Delta T= \frac{k}{\rho}
\Big(\frac{\partial}{\partial R^2} +
\frac{1}{R^2}\frac{\partial^2}{\partial\varphi^2}+
\frac{\partial^2}{\partial z^2}+
\frac{1}{R}\frac{\partial}{\partial R}
\Big)T
\]
{\setlength\arraycolsep{2pt}
\begin{eqnarray}\label{phi}
\Phi & = & 2\mu\Big[
\Big(\frac{\partial v_R}{\partial R}\Big)^2 +
\Big(\frac{1}{R}\frac{\partial v_\theta}{\partial\theta} + \frac{v_R}{R}\Big)^2 +
\Big(\frac{\partial v_z}{\partial z}\Big)^2
\Big] + \nonumber\\
& &+\mu\Big[
\Big(\frac{1}{R}\frac{\partial v_z}{\partial\theta} + \frac{\partial v_\theta}{\partial z}\Big)^2 +
\Big(\frac{\partial v_R}{\partial z} + \frac{\partial v_z}{\partial R}\Big)^2 +
\Big(\frac{1}{R}\frac{\partial v_R}{\partial\theta} + \frac{\partial v_\theta}{\partial R}-\frac{v_\theta}{R}\Big)^2
\Big]
\end{eqnarray}}
因此柱坐标下的能量方程为
\[
\frac{\partial e}{\partial t} +v_R\frac{\partial e}{\partial R}
+ \frac{v_\varphi}{R}\frac{\partial e}{\partial\varphi}
+ v_z\frac{\partial e}{\partial z}
= \frac{k}{\rho}
\Big(
\frac{\partial T}{\partial R^2} +
\frac{1}{R^2}\frac{\partial^2 T}{\partial\varphi^2}+
\frac{\partial^2 T}{\partial z^2}+
\frac{1}{R}\frac{\partial T}{\partial R}
\Big) + \Phi + \dot{q}
\]
其中$\Phi$见式(\ref{phi}).
\end{solution} 
