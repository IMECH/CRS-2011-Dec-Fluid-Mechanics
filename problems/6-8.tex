\begin{problem}[问题6.8]
在复平面上推导由两个极其靠近的反向点涡构成的偶极子的复速度势.
\end{problem}

\begin{solution}
\textbf{解:} 如图\ref{p8f1}平面上两个反向点涡, 其的复速度势可由各自的速度势叠加, 即
\[
W(z) = \frac{\Gamma}{2\pi i}\ln(z-a) + \frac{-\Gamma}{2\pi i}\ln(z+a)
= -\frac{2a\Gamma}{2\pi i}\frac{\ln(z-a)-\ln(z+a)}{(z-a)-(z+a)}
\]
两个点涡构成的偶极子, 则$a\rightarrow 0$, 因此上式有
\begin{eqnarray}
\lim_{a\rightarrow 0} W(z) &=& -\frac{2a\Gamma}{2\pi i}\lim_{a\rightarrow 0}\frac{\ln(z-a)-\ln(z+a)}{(z-a)-(z+a)}
= -\frac{a\Gamma}{\pi i}\frac{1}{z}\nonumber\\
&=& -\frac{a\Gamma}{\pi i}(\cos\theta - i\sin\theta)= \frac{a\Gamma}{\pi r}\sin\theta + i\frac{a\Gamma}{\pi r}\cos\theta\nonumber
\end{eqnarray}
由此可以作出势函数和流函数等值线, 如图\ref{p8f2}所示, 程序见附录\ref{sec:cPotentStream}.
\begin{figure}[!htb]
\begin{minipage}[b]{.5\textwidth}
\centering
%\includegraphics[width=0.8\textwidth]{./figures/p8f1.pdf}
\usetikzlibrary{%
    decorations.pathreplacing,%
    decorations.pathmorphing,arrows
}

\begin{tikzpicture}[ media/.style={font={\footnotesize\sffamily}},
    interface/.style={
        postaction={draw,decorate,decoration={border,angle=-45,
                    amplitude=0.3cm,segment length=2mm}}},scale=1.5]

\clip(-2,-2.4)rectangle(2.4,2.1);
\draw[semithick,->,>=stealth'](-2,0)--(2,0) node[right]{$x$};
\draw[semithick,->,>=stealth'](0,-1.8)--(0,1.6) node[above]{$y$};



\fill[gray,draw=blue](-1,0) circle(0.05) node[below,blue]{$-a$};
\draw[semithick,red,->,>=stealth'](-1,-0.5) arc(-90:90:0.5);

\fill[gray,draw=blue](1,0) circle(0.05)node[below,blue]{$a$};
\draw[semithick,red,->,>=stealth'](1,-0.5) arc(-90:-270:0.5);

\node[blue] at (1,1){$a\rightarrow 0$};

%\fill[blue!20](0.5,1)--(0.5,0.05)--(2,0.05)--(2,0.75);

%\draw[semithick] (0.5,1.25)--(0.5,0.05)--(2,0.05)--(2,1.25);
%\draw[blue, semithick] (0.5,1)--(2,0.75);
%\draw[blue,dashed](2,0.75)--(0.75,0.75);
%\draw (1,0.75) arc(180:170:1) node[blue,above]{$\theta$};
%\draw [semithick,->,>=stealth',blue] (2.1,0.5)--(2.75,0.5) node[right]{$a$};



\end{tikzpicture}

\caption{\label{p8f1}两个点涡构成的偶极子示意图}
\end{minipage}%
\begin{minipage}[b]{.5\textwidth}
\centering
\pgfplotsset{compat=1.7}
\begin{tikzpicture}
\begin{axis}[width=0.9\textwidth,height=0.9\textwidth,
xlabel={$x$},
ytick={-5,0,5},xtick={-5,0,5},
%yticklabels={0.035,0.1,0.2,0.3,0.4,0.5},
ylabel={$y$},
xmin=-5,xmax=5,ymin=-5,ymax=5, 
yticklabel style={font=\scriptsize},
xticklabel style={font=\scriptsize},
xlabel style={font=\footnotesize},
ylabel style={font=\footnotesize},
legend style={font=\scriptsize,legend cell align=left},
]



\addplot[blue,domain=-5:5,samples=2,dashed]{0};
\addplot[red] coordinates {(0,-5) (0,5)};
\addlegendentry{$\varphi = K_1$};
\addlegendentry{$\psi = K_2$};

\foreach \r in {0.5,1, 1.5, 2.25, 3.5, 5, 7, 10, 20}{
  \edef\temp{\noexpand\draw[blue,dashed] (axis cs:0,\r) circle(\r);}
  \temp
  \edef\temp{\noexpand\draw[blue,dashed] (axis cs:0,-\r) circle(\r);}
  \temp
  \edef\temp{\noexpand\draw[red] (axis cs:{\r},0) circle(\r);}
  \temp
  \edef\temp{\noexpand\draw[red] (axis cs:{-\r},0) circle(\r);}
  \temp
}


\end{axis}





\end{tikzpicture}

%\includegraphics[width=0.8\textwidth]{./figures/p8.pdf}
\caption{\label{p8f2}势函数和流函数等值线}
\end{minipage}
\end{figure}

\end{solution}
