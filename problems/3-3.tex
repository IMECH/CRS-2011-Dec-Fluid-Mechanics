\begin{problem}[问题3.3]
推导不可压缩流体流动的球坐标方程.
\end{problem}
\begin{solution}
\textbf{解:}分别推导不可压缩流体流动在球坐标下的连续性方程, 动量方程及能量方程:

\vspace{5pt}
\noindent\textbf{连续性方程}
\vspace{5pt}

\noindent 不可压缩流体流动连续性方程的一般形式为$\nabla\cdot\mathbf{v} = 0$, 代入球坐标得球坐球下续性方程为:
\[
\frac{1}{R^2}\frac{\partial}{R}(R^2v_R) + \frac{1}{R\sin\theta}\frac{\partial}{\partial\theta}(v_\theta\sin\theta) + \frac{1}{R\sin\theta}\frac{\partial v_\varphi}{\partial\varphi} = 0
\]

\vspace{5pt}
\noindent\textbf{动量方程}
\vspace{5pt}

\noindent 不可压缩流体流动动量方程的一般形式为
\[
\underbrace{\frac{d\mathbf{v}}{dt}\vphantom{\frac{1}{\rho}}}_{1} = \underbrace{- \frac{1}{\rho}\nabla p}_{2} + \underbrace{\mathbf{F}\vphantom{\frac{1}{\rho}}}_{3} +\underbrace{\mu\nabla^2\mathbf{v}\vphantom{\frac{1}{\rho}}}_{4}
\]
设球坐标中的单位矢量为$\mathbf{n}_R$, $ \mathbf{n}_\theta$, $\mathbf{n}_\varphi$, 现分别求上式中的4项在球坐标下的形式:
\begin{enumerate}
\item 在球坐标下$\mathbf{v} = \mathbf{v}_R + \mathbf{v}_\theta + \mathbf{v}_\varphi = v_R\mathbf{n}_R + v_\theta\mathbf{n}_\theta + v_\varphi\mathbf{n}_\varphi$, 因此第一项
{\setlength\arraycolsep{1pt}
\begin{eqnarray}
\frac{d\mathbf{v}}{dt} & = & \frac{d}{dt}\Big(v_R\mathbf{n}_R+v_\theta \mathbf{n}_\theta+v_\varphi\mathbf{n}_\varphi\Big)
\nonumber\\
& = & \frac{dv_R}{dt}\mathbf{n}_R + v_R\frac{d\mathbf{n}_R}{dt}
+ \frac{dv_\theta}{dt}\mathbf{n}_\theta + v_\theta\frac{d\mathbf{n}_\theta}{dt}
+ \frac{dv_\varphi}{dt}\mathbf{n}_\varphi + v_\varphi\frac{d\mathbf{n}_\varphi}{dt}
\nonumber
\end{eqnarray}}
将上式中的各项在球坐标下展开:
\begin{multicols}{2}
\setlength{\abovedisplayskip}{-5pt}
\begin{eqnarray}
v_R\frac{d\mathbf{n}_R}{dt} & = & v_R\Big[\frac{\partial\mathbf{n}_R}{\partial t} + (\mathbf{v}\cdot \nabla)\mathbf{n}_R\Big]\nonumber\\
 & = &  v_R(\mathbf{v}\cdot \nabla)\mathbf{n}_R
\nonumber\\
& = & v_R\Big[
v_R
\frac{\partial \mathbf{n}_R}{\partial R} + \frac{v_\theta}{R}\frac{\partial\mathbf{n}_R}{\partial\theta}
+ \frac{v_\varphi}{R\sin\theta}\frac{\partial\mathbf{n}_R}{\partial\varphi}
\Big]\nonumber\\
& = &  \frac{v_Rv_\theta}{R}\mathbf{n}_\theta + \frac{v_Rv_\varphi}{R}\mathbf{n}_\varphi
%
\nonumber\\
\nonumber\\
%
v_\theta\frac{d\mathbf{n}_\theta}{dt} & = & v_\theta\Big[\frac{\partial\mathbf{n}_\theta}{\partial t} + (\mathbf{v}\cdot \nabla)\mathbf{n}_\theta\Big]\nonumber\\
 & = &  v_\theta(\mathbf{v}\cdot \nabla)\mathbf{n}_\theta
\nonumber\\
& = & v_\theta\Big[
v_R
\frac{\partial \mathbf{n}_\theta}{\partial R} + \frac{v_\theta}{R}\frac{\partial\mathbf{n}_\theta}{\partial\theta}
+ \frac{v_\varphi}{R\sin\theta}\frac{\partial\mathbf{n}_\theta}{\partial\varphi}
\Big]
\nonumber\\
& = & -\frac{v_\theta^2}{R}\mathbf{n}_R + \frac{v_\varphi v_\theta \cot\theta}{R}\mathbf{n}_\varphi
%
\nonumber\\
\nonumber\\
%
v_\varphi\frac{d\mathbf{n}_\varphi}{dt} & = & v_\varphi\Big[\frac{\partial\mathbf{n}_\varphi}{\partial t} + (\mathbf{v}\cdot \nabla)\mathbf{n}_\varphi\Big]\nonumber\\
 & = &  v_\varphi(\mathbf{v}\cdot \nabla)\mathbf{n}_\varphi
\nonumber\\
& = & v_\varphi\Big[
v_R\frac{\partial \mathbf{n}_\varphi}{\partial R} + \frac{v_\theta}{R}\frac{\partial\mathbf{n}_\varphi}{\partial\theta}
+ \frac{v_\varphi}{R\sin\theta}\frac{\partial\mathbf{n}_\varphi}{\partial\varphi}
\Big]
\nonumber\\
& = & -\frac{v_\varphi^2}{R}\cot\theta\mathbf{n}_\theta - \frac{v_\varphi^2}{R}\mathbf{n}_R
\nonumber
\end{eqnarray}

\begin{eqnarray}
\frac{d v_R}{dt}\mathbf{n}_R & = &
   \Big[
       \frac{\partial v_R}{\partial t} + (\mathbf{v}\cdot\nabla)v_R
   \Big]\mathbf{n}_R\nonumber\\
   & = &
   \Big[
       \frac{\partial v_R}{\partial t} +
       v_R\frac{\partial v_R}{\partial R} +\nonumber\\
   & + & \frac{v_\theta}{R}\frac{\partial v_R}{\partial\theta}+\frac{v_\varphi}{R\sin\theta}\frac{\partial v_R}{\partial\varphi}
   \Big]\mathbf{n}_R
   \nonumber\\
   \nonumber\\
   \nonumber\\
   \frac{d v_\theta}{dt}\mathbf{n}_\theta & = &
   \Big[
       \frac{\partial v_\theta}{\partial t} + (\mathbf{v}\cdot\nabla)v_\theta
   \Big]\mathbf{n}_\theta\nonumber\\
   & = &
   \Big[
       \frac{\partial v_\theta}{\partial t} +
       v_R\frac{\partial v_\theta}{\partial R} +\nonumber\\
   & + & \frac{v_\theta}{R}\frac{\partial v_\theta}{\partial\theta}+\frac{v_\varphi}{R\sin\theta}\frac{\partial v_\theta}{\partial\varphi}
   \Big]\mathbf{n}_\theta
      \nonumber\\
   \nonumber\\
   \nonumber\\
\frac{d v_\varphi}{dt}\mathbf{n}_\varphi & = &
   \Big[
       \frac{\partial v_\varphi}{\partial t} + (\mathbf{v}\cdot\nabla)v_\varphi
   \Big]\mathbf{n}_\varphi\nonumber\\
   & = &
   \Big[
       \frac{\partial v_\varphi}{\partial t} +
       v_R\frac{\partial v_\varphi}{\partial R} +\nonumber\\
   & + & \frac{v_\theta}{R}\frac{\partial v_\varphi}{\partial\theta}+\frac{v_\varphi}{R\sin\theta}\frac{\partial v_\varphi}{\partial\varphi}
   \Big]\mathbf{n}_\varphi
      \nonumber\\
   \nonumber\\
   \nonumber
\end{eqnarray}
\end{multicols}

以上三式的推导中,用到了$\partial\mathbf{n}_R/\partial\theta=\mathbf{n}_\theta$, $\partial\mathbf{n}_\theta/\partial\theta=-\mathbf{n}_R$, $\partial\mathbf{n}_R/\partial\varphi=\sin\theta\mathbf{n}_\varphi$,
$\partial\mathbf{n}_\theta/\partial\varphi=\cos\theta\mathbf{n}_\varphi$,
$\partial\mathbf{n}_\varphi/\partial\varphi=-(\cos\theta\mathbf{n}_\theta + \sin\theta\mathbf{n}_R)$及
$\partial\mathbf{n}_R/\partial R=\partial\mathbf{n}_\theta/\partial R=\partial\mathbf{n}_\varphi/\partial R =\partial\mathbf{n}_\varphi/\partial \theta=0$. 最终可以得到
\begin{eqnarray}
\frac{d\mathbf{v}}{dt}
& = &
\Big(\frac{\partial v_R}{\partial t} +
       v_R\frac{\partial v_R}{\partial R}
    +  \frac{v_\theta}{R}\frac{\partial v_R}{\partial\theta}+\frac{v_\varphi}{R\sin\theta}\frac{\partial v_R}{\partial\varphi}-\frac{v_\theta^2-v_\varphi^2}{R}\Big)\mathbf{n}_R\nonumber\\
& + &
\Big(\frac{\partial v_\theta}{\partial t} +
       v_R\frac{\partial v_\theta}{\partial R}
   + \frac{v_\theta}{R}\frac{\partial v_\theta}{\partial\theta}+\frac{v_\varphi}{R\sin\theta}\frac{\partial v_\theta}{\partial\varphi}+\frac{v_Rv_\theta-v_\varphi^2\cot\theta}{R} \Big)\mathbf{n}_\theta\nonumber\\
& +&
\Big(\frac{\partial v_\varphi}{\partial t} +
       v_R\frac{\partial v_\varphi}{\partial R}
    +  \frac{v_\theta}{R}\frac{\partial v_\varphi}{\partial\theta}+\frac{v_\varphi}{R\sin\theta}\frac{\partial v_\varphi}{\partial\varphi}
+\frac{v_Rv_\varphi+v_\varphi v_\theta \cot\theta}{R}\Big)\mathbf{n}_\varphi
\end{eqnarray}

\item 对于第二项中的$-1/\rho\nabla p$有
\begin{equation}
-\frac{1}{\rho}\nabla p =
-\frac{1}{\rho}\frac{\partial p}{\partial R}\mathbf{n}_R
-\frac{1}{\rho}\frac{1}{R}\frac{\partial p}{\partial\theta}\mathbf{n}_\theta
-\frac{1}{\rho}\frac{1}{R\sin\theta}\frac{\partial p}{\partial \varphi}\mathbf{n}_\varphi
\end{equation}

\item 对于第三项$\mathbf{F}$, 则可表示成三个方向上的分量
\begin{equation}
\mathbf{F} = \mathbf{F}_R + \mathbf{F}_\theta + \mathbf{F}_\varphi = F_R\mathbf{n}_R + F_\theta\mathbf{n}_\theta + F_\varphi\mathbf{n}_\varphi
\end{equation}

\item 由$\nabla^2\mathbf{v}=\Delta\mathbf{v}$,可知第四项可有
\begin{eqnarray}
\nabla^2\mathbf{v}
& = & \Big(
          \Delta v_R -
          \frac{2v_R}{R} -
          \frac{2}{R^2\sin\theta}\frac{\partial(v_\theta\sin\theta)}{\partial\theta} - \frac{2}{R^2\sin\theta}\frac{\partial v_\varphi}{\partial\varphi}
      \Big)\mathbf{n}_R +  \nonumber\\
& + & \Big(
          \Delta v_\theta +
          \frac{2}{R^2}\frac{\partial v_R}{\partial\theta} -
          \frac{v_\theta}{R^2\sin^2\theta} -
          \frac{2\cos\theta}{R^2\sin^2\theta}\frac{\partial v_\varphi}{\partial\varphi}
      \Big)\mathbf{n}_\theta + \nonumber\\
& + & \Big(
          \Delta v_\varphi +
          \frac{2}{R^2\sin\theta}\frac{\partial v_R}{\partial\varphi} +
          \frac{2\cos\theta}{R^2\sin^2\theta}\frac{\partial v_\theta}{\partial\varphi}-
          \frac{v_\varphi}{R^2\sin^2\theta}
      \Big)\mathbf{n}_\varphi
\end{eqnarray}

其中$\Delta = \frac{\partial^2}{\partial R^2} +
\frac{2}{R}\frac{\partial}{\partial R}+
\frac{\cot\theta}{R^2}\frac{\partial}{\partial\theta}+
\frac{1}{R^2}\frac{\partial^2}{\partial\theta^2}+
\frac{1}{R^2\sin^2\theta}\frac{\partial^2}{\partial\varphi^2}$
\end{enumerate}

\noindent 根据式(6-9), 可写出$\mathbf{n}_R$, $\mathbf{n}_\theta$, $\mathbf{n}_\varphi$方向的动量方程
\begin{eqnarray}
\frac{\partial v_R}{\partial t} &+&
v_R\frac{\partial v_R}{\partial R}+
\frac{v_\theta}{R}\frac{\partial v_R}{\partial\theta}
+\frac{v_\varphi}{R\sin\theta}\frac{\partial v_R}{\partial\varphi}-
\frac{v_\theta^2-v_\varphi^2}{R}\nonumber\\
 & = &  -\frac{1}{\rho}\frac{\partial p}{\partial R} + F_R +
      \mu\Big(
          \Delta v_R -
          \frac{2v_R}{R} -
          \frac{2}{R^2\sin\theta}\frac{\partial(v_\theta\sin\theta)}{\partial\theta} - \frac{2}{R^2\sin\theta}\frac{\partial v_\varphi}{\partial\varphi}
      \Big)
\nonumber\\
\nonumber\\
\frac{\partial v_\theta}{\partial t} &+&
v_R\frac{\partial v_\theta}{\partial R}+
\frac{v_\theta}{R}\frac{\partial v_\theta}{\partial\theta}
+\frac{v_\varphi}{R\sin\theta}\frac{\partial v_\theta}{\partial\varphi}+
\frac{v_Rv_\theta-v_\varphi^2\cot\theta}{R}\nonumber\\
 & = & -\frac{1}{\rho}\frac{1}{R}\frac{\partial p}{\partial\theta} + F_\theta+\mu\Big(
          \Delta v_\theta +
          \frac{2}{R^2}\frac{\partial v_R}{\partial\theta} -
          \frac{v_\theta}{R^2\sin^2\theta} -
          \frac{2\cos\theta}{R^2\sin^2\theta}\frac{\partial v_\varphi}{\partial\varphi}
      \Big)
\nonumber\\
\nonumber\\
\frac{\partial v_\varphi}{\partial t} &+&
v_R\frac{\partial v_\varphi}{\partial R}+
\frac{v_\theta}{R}\frac{\partial v_\varphi}{\partial\theta}
+\frac{v_\varphi}{R\sin\theta}\frac{\partial v_\varphi}{\partial\varphi}
+\frac{v_Rv_\varphi+v_\varphi v_\theta \cot\theta}{R}\nonumber\\
& = & -\frac{1}{\rho}\frac{1}{R\sin\theta}\frac{\partial p}{\partial \varphi} + F_\varphi+\mu\Big(
          \Delta v_\varphi +
          \frac{2}{R^2\sin\theta}\frac{\partial v_R}{\partial\varphi} +
          \frac{2\cos\theta}{R^2\sin^2\theta}\frac{\partial v_\theta}{\partial\varphi}-
          \frac{v_\varphi}{R^2\sin^2\theta}
      \Big)
 \nonumber
\end{eqnarray}

\noindent 其中$\Delta = \frac{\partial^2}{\partial R^2} +
\frac{2}{R}\frac{\partial}{\partial R}+
\frac{\cot\theta}{R^2}\frac{\partial}{\partial\theta}+
\frac{1}{R^2}\frac{\partial^2}{\partial\theta^2}+
\frac{1}{R^2\sin^2\theta}\frac{\partial^2}{\partial\varphi^2}$

\vspace{5pt}
\noindent\textbf{能量方程}
\vspace{5pt}

\noindent 能量方程的一般形式如下
\[
\frac{\partial e}{\partial t} +\mathbf{v}\cdot\nabla e = \frac{1}{\rho}\nabla\cdot(k\nabla T) + \Phi + \dot{q}
\]
其中$\Phi$为耗散函数. 各项在球坐标下有
{\setlength\arraycolsep{2pt}
\begin{eqnarray}
\mathbf{v}\cdot\nabla e & = & (v_R\mathbf{n}_R + v_\varphi\mathbf{n}_\theta + v_\varphi\mathbf{n}_\varphi)\cdot
\Big(\frac{\partial e}{\partial R}\mathbf{n}_R
+ \frac{1}{R}\frac{\partial e}{\partial\theta}\mathbf{n}_\theta
+ \frac{1}{R\sin\theta}\frac{\partial e}{\partial \varphi}\mathbf{n}_\varphi\Big)\nonumber\\
& = &
v_R\frac{\partial e}{\partial R} +
\frac{v_\theta }{R}\frac{\partial e}{\partial\theta}+
\frac{v_\varphi}{R\sin\theta}\frac{\partial e}{\partial \varphi}\nonumber
\end{eqnarray}}
\[
\frac{1}{\rho}\nabla\cdot(k\nabla T) = \frac{k}{\rho}\Delta T= \frac{k}{\rho}
\Big(
\frac{\partial^2}{\partial R^2} + \frac{2}{R}\frac{\partial}{\partial R}
+ \frac{\cot\theta}{R^2}\frac{\partial}{\partial\theta}+\frac{1}{R^2}\frac{\partial^2}{\partial\theta^2}
+\frac{1}{R^2\sin^2\theta}\frac{\partial^2}{\partial\varphi^2}
\Big)T
\]
{\setlength\arraycolsep{2pt}
\begin{eqnarray}\label{phiSphere}
\Phi  =  \mu
\Big\{
   2& \big[ &
        \big(\frac{\partial v_R}{\partial R}\big)^2
        +\big(\frac{1}{R}\frac{\partial v_\theta}{\partial\theta}+\frac{v_R}{R}\big)^2
        +\big(\frac{1}{R\sin\theta}\frac{\partial v_\varphi}{\partial\varphi}+\frac{v_R}{R}+\frac{v_\theta\cot\theta}{R}\big)^2
    \big]\nonumber\\
+ & \big[ &
        \frac{1}{R\sin\theta}\frac{\partial v_\theta}{\partial\varphi}
        +\frac{\sin\theta}{R}\frac{\partial}{\partial\theta}\big(\frac{v_\varphi}{\sin\theta}\big)
    \big]^2
+  \big[
         \frac{1}{R\sin\theta}\frac{\partial v_R}{\partial\varphi}
        +R\frac{\partial}{\partial R}\big(\frac{v_\varphi}{R}\big)
    \big]^2\nonumber\\
+ & \big[ &
        R\frac{\partial}{\partial R}\big(\frac{v_\theta}{R}\big)
        +\frac{1}{R}\frac{\partial v_R}{\partial\theta}
    \big]^2
\Big\}
\end{eqnarray}}
因此球坐标下的能量方程为
{\setlength\arraycolsep{2pt}
\begin{eqnarray}
\frac{\partial e}{\partial t}
+ v_R\frac{\partial e}{\partial R} +
\frac{v_\theta }{R}\frac{\partial e}{\partial\theta}+
\frac{v_\varphi}{R\sin\theta}\frac{\partial e}{\partial \varphi}
=
\frac{k}{\rho}
\Big(
\frac{\partial^2T}{\partial R^2} &+& \frac{2}{R}\frac{\partial T}{\partial R}
+ \frac{\cot\theta}{R^2}\frac{\partial T}{\partial\theta}+\frac{1}{R^2}\frac{\partial^2T}{\partial\theta^2}\nonumber\\
&+&\frac{1}{R^2\sin^2\theta}\frac{\partial^2T}{\partial\varphi^2}
\Big)
+\Phi
+\dot{q}
\end{eqnarray}}
其中$\Phi$见式(\ref{phiSphere}).
\end{solution} 
