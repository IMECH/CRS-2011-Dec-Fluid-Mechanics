\documentclass[12pt,a4paper,boxed,titlepage]{caspset}

% set 1-inch margins in the document
%\usepackage[left=1in,right=1in,top=1.2in,bottom=1in]{geometry}
\usepackage[left=1in,right=1in,top=1.2in,bottom=1in]{geometry}
\usepackage{lastpage}

% include this if you want to import graphics files with /includegraphics
\usepackage{multicol} 
\usepackage{capt-of}%%To get the caption
\usepackage{graphicx}
\usepackage{lscape}
\usepackage{amsmath,amsfonts,amsthm,amssymb}
\usepackage{mathtools}
\usepackage{hyperref}
\usepackage{setspace}
\usepackage{fancyhdr}
\usepackage{lastpage}
\usepackage{extramarks}
\usepackage{chngpage}
\usepackage{soul}
\usepackage[usenames,dvipsnames]{color}
\usepackage{graphicx,float,wrapfig}
\usepackage{ifthen}
\usepackage{listings}
\usepackage{courier}
\usepackage{multimedia}
\usepackage[toc,page,title,titletoc,header]{appendix}
\usepackage{color, soul}
\usepackage{tikz}
\usepackage{array}
\usepackage{multirow}
\usepackage{todonotes}
\usepackage{pdfpages}
\usepackage{attachfile2}
\usepackage{pgfplots}
\usetikzlibrary{%
    decorations.pathreplacing,%
    decorations.pathmorphing,arrows
}
\usetikzlibrary{calc}
\usepgfplotslibrary{polar}
%%%%%%%%%%%%%%%%%%%%%%%%%%%%%%%%%%%%%%%%%%%%%%%%%%%%%%
\usepackage{xeCJK}
%\usepackage{fontspec}
\setCJKmainfont[BoldFont=simhei.ttf]{simsun.ttf}
%\setCJKsansfont{simhei.ttf}
%\setCJKmonofont{simfang.ttf}

%\setCJKmainfont{Adobe Song Std}
%\setCJKmainfont[BoldFont=Adobe Heiti Std]{Adobe Song Std}
%%%%%%%%%%%%%%%%%%%%%%%%%%%%%%%%%%%%%%%%%%%%%%%%%%%%%%

\graphicspath{{figures/}}

\setulcolor{red}

\setlength{\marginparwidth}{1in}

\newcommand{\hmwkTitle}{流体力学导论}
%\newcommand{\hmwkSubTitle}{} % No subtitle, so this will be excluded
\newcommand{\hmwkDueDate}{\today}
\newcommand{\hmwkClass}{物理学院}
\newcommand{\hmwkClassTime}{Mon./Wed.{~}13:30}
\newcommand{\hmwkClassInstructor}{王智慧}
\newcommand{\hmwkAuthorName}{周吕文}

\hypersetup{pdfauthor={\hmwkAuthorName}, 
            pdftitle={流体力学导论作业整理}, 
            pdfsubject={\hmwkTitle, \hmwkClassInstructor},
            pdfkeywords={流体力学},
            pdfproducer={XeLateX with hyperref},
            pdfcreator={Xelatex}}


%% Setup the header and footer
\pagestyle{fancy}                                                       %
\lhead{\hmwkAuthorName}                                                 %
\chead{\hmwkClass\ (\hmwkClassInstructor): \hmwkTitle}  %
\rhead{第\ \thepage\ 页, 共\ \protect\pageref{LastPage} 页}          %



\makeatletter
\newcommand{\rmnum}[1]{\romannumeral #1}
\newcommand{\Rmnum}[1]{\expandafter\@slowromancap\romannumeral #1@}
\makeatother

\renewcommand\refname{\bf\large 参考文献}
\renewcommand\contentsname{\bf 目 \ \ \ 录}
\renewcommand\figurename{\bf 图}
\renewcommand\tablename{\bf 表}
\renewcommand{\appendixtocname}{附录}
\renewcommand{\appendixpagename}{附录}
\renewcommand\listfigurename{图目录}

% info for header block in upper right hand corner
%\name{周吕文{~}201128000718065}
%\class{物理学院{~}20110308班}
%\assignment{习题整理}
%\duedate{06/14/2014}

\newcommand\invisiblesection[1]{%
  \refstepcounter{section}%
  \addcontentsline{toc}{section}{\protect\numberline{\thesection}#1}%
  \sectionmark{#1}}

\newcommand\invisiblesubsection[1]{%
  \refstepcounter{subsection}%
  \addcontentsline{toc}{subsection}{\protect\numberline{\thesubsection}#1}%
  \subsectionmark{#1}}


\newcommand{\matlabscript}[2]
{\definecolor{MyDarkGreen}{rgb}{0.0,0.3,0.0}
\definecolor{hellgelb}{rgb}{0.96,0.96,0.96}
\definecolor{DarkPurple}{rgb}{0.6,0,0.4}
    \lstset{%
       language=Matlab,                        % Use MATLAB
        frame=single,                           % Single frame around code
        basicstyle=\footnotesize\ttfamily,    
        keywordstyle=[1]\color{blue}, 
        keywordstyle=[2]\color{DarkPurple}, 
        keywordstyle=[3]\color{blue}\underbar, 
        identifierstyle=,  
        commentstyle=\color{MyDarkGreen}\footnotesize,
        stringstyle=\color{DarkPurple}, 
        showstringspaces=false,            
        tabsize=5,     
        morekeywords={xlim,ylim,var,alpha,factorial,poissrnd,normpdf,normcdf},
        morekeywords=[2]{on, off, interp},
        morecomment=[l][\color{blue}]{...},    
     columns=fixed,
     tabsize=4,%
     frame=single,%
     framerule=1pt,
     extendedchars=true,%
     showspaces=false,%
     showstringspaces=false,%
     numbers=left,%
     numberstyle=\tiny\ttfamily,%
     numbersep=1em,%
     breaklines=true,%
     breakindent=10pt,%
     backgroundcolor=\color{hellgelb},%
     breakautoindent=true,%
     captionpos=t,%
     xleftmargin=1em,%
     xrightmargin=\fboxsep%
    }
\lstinputlisting[caption=#2,label=#1]{#1.m}}


\begin{document}
\title{流体力学导论(2011-2012秋)作业解答\\复习材料\\ \vspace{-20pt} }

\author{{\Large 授课老师:王智慧}\\ \vspace{50pt}\\ 周吕文\\ \href{mailto:zhou.lv.wen@gmail.com}{zhou.lv.wen@gmail.com}\\ \vspace{50pt}}
\date{中国科学院力学研究所\\2011年12月05日}
%\tableofcontents
\maketitle 

\noindent{\small\textbf{说明}: 本文档是由本人的流导作业, 参考课程网上所给答案整理而成. 目录中标题为\textcolor{blue}{\bf 蓝色}的个人以为比较重要. 时间有限, 难免有误, 发现问题, 请电邮我. 祝各位考试顺利.}
\vspace{-1em}
\enlargethispage{2\baselineskip}
\tableofcontents
\setcounter{page}{0}
\newpage

%\includepdf[pages=1-]{cover}
%\setcounter{page}{1}
\invisiblesection{第一章{~}流体与流体的物理性质}
\problemlist{\bf 第一章{~}流体与流体的物理性质}
\invisiblesubsection{平板Couette流的启动}
\begin{problem}[问题1.1]
1687年牛顿首先发表了他的剪切流动的实验结果,他的实验是在两相距$h$的平行板之间充满粘性流体后进行的,如图\ref{experiment}. 令下平行板固定不动,而使上平板在其自身平面以等速$U$向右运动. 实验指出, 平衡后作用于平板上的力与速度$U$
及平板面积$A$成正比,与平板间距$h$成反比,即
\[
F = \mu \frac{U}{h}A {~}\textrm{或}{~} \tau = \frac{F}{A}=\mu\frac{U}{h}
\]
讨论在牛顿剪切流动的实验中, 作用于上平板的外力在流体运动平衡前随时间的变化.
\end{problem}


\begin{solution}
\begin{figure}[!htb]
\begin{minipage}[c]{.5\textwidth}
\centering
%\includegraphics[width=0.9\textwidth]{experiment.pdf}
\usetikzlibrary{%
    decorations.pathreplacing,%
    decorations.pathmorphing,arrows
}

\begin{tikzpicture}
\draw[->,semithick, >=stealth'] (-0.5,0)--(6,0) node[below=3pt]{$x$};
\draw[->,semithick, >=stealth'] (0,-0.5)--(0,3.25)node[left=3pt]{$y$};
\draw[semithick] (-0.1,2.4)node[left]{$H$}--(0,2.4);
\node[below left=3pt] at (0,0){$o$};

\draw[fill=gray] (1,2.4) rectangle (5,2.5);
\draw[->,semithick, >=stealth',blue] (5.2,2.45)--(6,2.45)node[midway,above]{$U$};
\draw[->,semithick, >=stealth',blue] (2.5,2.8)--(4,2.8)node[midway,above]{$F$};

\draw[semithick,blue] (1.5,0)--(1.5,2.4) (1.5,0)--(3.1,2.4);

\draw[->,semithick, >=stealth',blue](1.5,0.6)--(1.9,0.6);
\draw[->,semithick, >=stealth',blue](1.5,1.2)--(2.3,1.2) node[right=5pt]{$u(y)$};
\draw[->,semithick, >=stealth',blue](1.5,1.8)--(2.7,1.8);

\end{tikzpicture}
\caption{\label{experiment}剪切流实验}
\end{minipage}%
\begin{minipage}[c]{.5\textwidth}
\centering
\usetikzlibrary{%
    decorations.pathreplacing,%
    decorations.pathmorphing,arrows
}

\begin{tikzpicture}
\draw[->,semithick, >=stealth'] (-0.5,0)--(6,0) node[below=3pt]{$x$};
\draw[->,semithick, >=stealth'] (0,-0.5)--(0,3.25)node[left=3pt]{$y$};
\draw[semithick] (-0.1,2.4)node[left]{$H$}--(0,2.4);
\node[below left=3pt] at (0,0){$o$};

\draw[fill=gray] (1,2.4) rectangle (5,2.5);
\draw[->,semithick, >=stealth',blue] (5.2,2.45)--(6,2.45)node[midway,above]{$U$};
\draw[->,semithick, >=stealth',blue] (2.5,2.8)--(4,2.8)node[midway,above]{$F$};

\draw[fill=red!40] (1,1.55) rectangle (5,1.95) node[midway,font=\footnotesize]{1};
\draw[fill=green!90!black] (1,1.05) rectangle (5,1.45)  node[midway,font=\footnotesize]{2};
\draw[fill=blue!40] (1,0.55) rectangle (5,0.95)  node[midway,font=\footnotesize]{3};

\draw[->,semithick, >=stealth',blue] (2.5,1.15)--(0.75,1.15) node[left]{$f_{32}$};
\draw[->,semithick, >=stealth',red] (3.5,1.35)--(5.25,1.35) node[right]{$f_{12}$};


\end{tikzpicture}
%\includegraphics[width=0.9\textwidth]{layers.pdf}
\caption{\label{layers}微元受力分析}
\end{minipage}
\end{figure}

\textbf{解:}如图\ref{layers}所示, 取高度为$\Delta h$面积为$A$的三个微元层, 标记为1,2,3. 设1,2,3层的速度分别为$v_1$,$v_2$,$v_3$; 微元1对微元2的作用力为$f_{12}$, 微元3对微元2的作用力为$f_{32}$. 则有
\begin{equation}
f_{12} = \mu \frac{v1-v2}{\Delta h}A=\mu \frac{du(y_{12})}{dy}A, {~}f_{32} = \mu \frac{v_3-v_2}{\Delta h}A= \mu \frac{du(y_{32})}{dy}A
\end{equation}
则对于微元2有
\begin{equation}
\frac{du(y_2)}{dt} = \frac{f_{12} - f_{32}}{\rho A \Delta h}
= \frac{f_{12} - f_{23}}{\rho A \Delta h} = \frac{\mu}{\rho}\frac{d^2u(y_2)}{dy^2}
\end{equation}
因此有以下微分方程组
\begin{equation}\label{diff}
\begin{dcases}
\frac{\partial u(y,t)}{\partial t} = \frac{\mu}{\rho} \frac{\partial ^2 u(y,t)}{\partial y^2} \\
\frac{\partial u(y,t)}{\partial x} = 0 \\
u(0,t) = 0, {~}u(H,t) = U, {~}u(y<H)|_{t=0} = 0
\end{dcases}
\end{equation}

\noindent\textbf{数值模拟法求解}

根据以上分析, 将微分方程组式(\ref{diff})离散成差分格式,并通过编程计算,可得到外力F随时间的近似变化. 例如, 以273.15K下的水为例, 并令$H=0.05$m, $U=1$m/s.将高$H=0.05$m的水等高的划分成$n = 100$层. 以$dt = 0.02$为时间步长, 每个时间步内,通过计算各层间的黏性力,来得到各层的加速度, 从而更新各层的速度. 图\ref{force}是外力$F$随时间的变化, 其平衡值为0.0354, 与理论值$F=\mu U/H = 1750\times 10^{-6}\times 1/0.05 = 0.0350$相差1.14\%. 图\ref{velocity}为不同时刻速度沿y方向的分布,$t=1000$s时速度成线性分布,这与牛顿所给出的结果是一致的.该计算所使用的程序见附录\ref{sec:ShearExperiment}.

\begin{figure}[!htb]
\begin{minipage}[c]{.5\textwidth}
\centering
%\includegraphics[width=0.9\textwidth]{force.pdf}
\begin{tikzpicture}
\begin{axis}[width=0.9\textwidth,height=0.75\textwidth,
xmin=-2,
xmax=1000,
xlabel={Time(s)},
ymin=-0.002,
ymax=0.5,
ytick={0.0354,0.1,0.2,0.3,0.4,0.5},
yticklabels={0.035,0.1,0.2,0.3,0.4,0.5},
ylabel={Force(N)},
yticklabel style={font=\scriptsize},
xticklabel style={font=\scriptsize},
xlabel style={font=\footnotesize},
ylabel style={font=\footnotesize}
]

\addplot[color=blue,dashed,domain=0:1000]{0.0354};
\addplot[no markers, smooth,red]
  table[row sep=crcr]{%
0	3.5\\
1	0.734305374682564\\
2	0.523454217532651\\
3	0.428561127577308\\
4	0.371650449039231\\
5	0.332686255938993\\
6	0.303865383638172\\
7	0.281434416211684\\
8	0.263334834391634\\
9	0.248330990656227\\
10	0.235630383921036\\
11	0.22469825068544\\
12	0.215158928170908\\
13	0.206739748990519\\
14	0.199237365019313\\
15	0.192496619886634\\
16	0.18639679376219\\
17	0.180842361161256\\
18	0.175756610150057\\
19	0.171077131664224\\
20	0.166752563890995\\
30	0.136190153414217\\
40	0.117960271744588\\
50	0.105515537739813\\
60	0.0963273400535577\\
70	0.0891852753705457\\
80	0.0834276368124073\\
90	0.0786581485082112\\
100	0.0746231321090408\\
110	0.071151732712454\\
120	0.0681242116812327\\
130	0.0654539500916878\\
140	0.063076652996078\\
150	0.0609435643755754\\
160	0.0590170301023314\\
170	0.0572674971715746\\
180	0.05567142669425\\
190	0.0542098088997883\\
200	0.0528670869882919\\
210	0.0516303658438363\\
220	0.050488823410101\\
230	0.0494332686621414\\
240	0.0484558070013036\\
250	0.0475495851612274\\
260	0.0467085954262233\\
270	0.0459275243693893\\
280	0.0452016351782601\\
290	0.0445266754331262\\
300	0.0438988042531376\\
310	0.0433145342402184\\
320	0.0427706847774666\\
330	0.0422643440804263\\
340	0.0417928380306875\\
350	0.041353704295644\\
360	0.0409446705955765\\
370	0.0405636362488099\\
380	0.040208656329467\\
390	0.0398779279264925\\
400	0.0395697781094865\\
410	0.0392826532955367\\
420	0.0390151097787045\\
430	0.0387658052351657\\
440	0.0385334910562569\\
450	0.0383170053917145\\
460	0.0381152668084958\\
470	0.0379272684883656\\
480	0.0377520729012517\\
490	0.0375888069021253\\
500	0.0374366572075955\\
510	0.0372948662150735\\
520	0.0371627281326662\\
530	0.0370395853922217\\
540	0.0369248253213969\\
550	0.0368178770534715\\
560	0.0367182086559546\\
570	0.0366253244610162\\
580	0.0365387625824411\\
590	0.0364580926052134\\
600	0.036382913435103\\
610	0.0363128512966771\\
620	0.0362475578691363\\
630	0.0361867085502058\\
640	0.036130000839075\\
650	0.0360771528300723\\
660	0.0360279018093647\\
670	0.035982002947553\\
680	0.0359392280815364\\
690	0.0358993645795062\\
700	0.03586221428335\\
710	0.0358275925231648\\
720	0.0357953271989354\\
730	0.0357652579247886\\
740	0.0357372352315385\\
750	0.0357111198235532\\
760	0.0356867818862336\\
770	0.0356641004406372\\
780	0.0356429627420598\\
790	0.0356232637195538\\
800	0.0356049054536047\\
810	0.035587796689372\\
820	0.0355718523830569\\
830	0.0355569932791549\\
840	0.0355431455164816\\
850	0.0355302402610206\\
860	0.035518213363754\\
870	0.0355070050417907\\
880	0.0354965595811954\\
890	0.0354868250600427\\
900	0.0354777530903255\\
910	0.0354692985774287\\
920	0.03546141949597\\
930	0.0354540766809042\\
940	0.0354472336328409\\
950	0.0354408563366115\\
960	0.0354349130921934\\
970	0.0354293743571326\\
980	0.0354242125996927\\
990	0.0354194021620075\\
1000	0.0354149191325349\\
};
\end{axis}
\end{tikzpicture}%

\caption{\label{force}外力$F$随时间的变化}
\end{minipage}%
\begin{minipage}[c]{.5\textwidth}
\centering
%\includegraphics[width=0.85\textwidth]{velocity.pdf}
\begin{tikzpicture}
\begin{axis}[width=0.9\textwidth,height=0.75\textwidth,
xmin=0,
xmax=0.05,
xlabel={H(m)},
ymin=0,
ymax=1,
ylabel={Velocity(m/s)},
yticklabel style={font=\scriptsize},
xticklabel style={font=\scriptsize},
xlabel style={font=\footnotesize},
ylabel style={font=\footnotesize},
      scaled x ticks = false,
      xticklabel style={/pgf/number format/fixed,
      /pgf/number format/1000 sep = \thinspace},
anchor=left of south west,
legend style={legend pos=north west,font=\scriptsize,legend cell align=left},
]

\addplot [color=red,solid]
  table[row sep=crcr]{%
0.0005	0\\
0.001	0.0100834743507922\\
0.0015	0.0201669663585983\\
0.002	0.0302504936626533\\
0.0025	0.0403340738666517\\
0.003	0.0504177245210224\\
0.0035	0.0605014631052565\\
0.004	0.0705853070103073\\
0.0045	0.0806692735210791\\
0.005	0.0907533797990224\\
0.0055	0.100837642864854\\
0.006	0.110922079581419\\
0.0065	0.121006706636711\\
0.007	0.131091540527067\\
0.0075	0.141176597540562\\
0.008	0.1512618937406\\
0.0085	0.16134744494975\\
0.009	0.171433266733804\\
0.0095	0.18151937438611\\
0.01	0.191605782912171\\
0.0105	0.201692507014535\\
0.011	0.211779561077991\\
0.0115	0.221866959155087\\
0.012	0.231954714951977\\
0.0125	0.242042841814621\\
0.013	0.252131352715349\\
0.0135	0.262220260239795\\
0.014	0.272309576574229\\
0.0145	0.282399313493281\\
0.015	0.292489482348091\\
0.0155	0.302580094054875\\
0.016	0.312671159083934\\
0.0165	0.322762687449112\\
0.017	0.332854688697714\\
0.0175	0.342947171900891\\
0.018	0.353040145644506\\
0.0185	0.36313361802049\\
0.019	0.373227596618692\\
0.0195	0.383322088519241\\
0.02	0.393417100285411\\
0.0205	0.403512637957018\\
0.021	0.413608707044333\\
0.0215	0.423705312522539\\
0.022	0.433802458826718\\
0.0225	0.443900149847385\\
0.023	0.453998388926574\\
0.0235	0.464097178854467\\
0.024	0.474196521866593\\
0.0245	0.484296419641567\\
0.025	0.494396873299407\\
0.0255	0.504497883400406\\
0.026	0.514599449944563\\
0.0265	0.524701572371586\\
0.027	0.534804249561458\\
0.0275	0.544907479835562\\
0.028	0.555011260958372\\
0.0285	0.565115590139704\\
0.029	0.575220464037524\\
0.0295	0.585325878761319\\
0.03	0.595431829876004\\
0.0305	0.605538312406399\\
0.031	0.615645320842232\\
0.0315	0.625752849143687\\
0.032	0.635860890747489\\
0.0325	0.645969438573511\\
0.033	0.656078485031903\\
0.0335	0.666188022030734\\
0.034	0.676298040984141\\
0.0345	0.686408532820973\\
0.035	0.696519487993927\\
0.0355	0.706630896489156\\
0.036	0.716742747836361\\
0.0365	0.726855031119325\\
0.037	0.736967734986909\\
0.0375	0.747080847664482\\
0.038	0.757194356965775\\
0.0385	0.767308250305153\\
0.039	0.777422514710286\\
0.0395	0.787537136835215\\
0.04	0.797652102973785\\
0.0405	0.807767399073449\\
0.041	0.817883010749417\\
0.0415	0.827998923299141\\
0.042	0.838115121717119\\
0.0425	0.848231590710003\\
0.043	0.858348314711998\\
0.0435	0.868465277900539\\
0.044	0.878582464212219\\
0.0445	0.888699857358965\\
0.045	0.898817440844438\\
0.0455	0.908935197980645\\
0.046	0.919053111904742\\
0.0465	0.929171165596011\\
0.047	0.939289341893001\\
0.0475	0.949407623510809\\
0.048	0.95952599305848\\
0.0485	0.969644433056524\\
0.049	0.979762925954512\\
0.0495	0.989881454148749\\
0.05	1\\
};
\addlegendentry{t=1000s};

\addplot [color=red!70!blue,densely dashed]
  table[row sep=crcr]{%
0.0005	0\\
0.001	0.00950594677208322\\
0.0015	0.0190124926765428\\
0.002	0.0285202362426647\\
0.0025	0.0380297747941604\\
0.003	0.0475417038478953\\
0.0035	0.0570566165144347\\
0.004	0.0665751029010088\\
0.0045	0.0760977495174991\\
0.005	0.0856251386860422\\
0.0055	0.0951578479548454\\
0.006	0.104696449516803\\
0.0065	0.1142415096335\\
0.007	0.123793588065179\\
0.0075	0.133353237507248\\
0.008	0.142921003033892\\
0.0085	0.152497421549352\\
0.009	0.162083021247419\\
0.0095	0.171678321079691\\
0.01	0.181283830233124\\
0.0105	0.190900047617406\\
0.011	0.200527461362672\\
0.0115	0.210166548328056\\
0.012	0.219817773621581\\
0.0125	0.229481590131881\\
0.013	0.239158438072197\\
0.0135	0.248848744537142\\
0.014	0.25855292307265\\
0.0145	0.268271373259566\\
0.015	0.27800448031128\\
0.0155	0.287752614685815\\
0.016	0.297516131712769\\
0.0165	0.307295371235467\\
0.017	0.317090657268707\\
0.0175	0.326902297672422\\
0.018	0.336730583841611\\
0.0185	0.346575790412824\\
0.019	0.356438174987524\\
0.0195	0.366317977872585\\
0.02	0.376215421838197\\
0.0205	0.386130711893424\\
0.021	0.396064035079634\\
0.0215	0.406015560282015\\
0.022	0.415985438059367\\
0.0225	0.425973800492337\\
0.023	0.435980761050262\\
0.0235	0.446006414476744\\
0.024	0.456050836694068\\
0.0245	0.466114084726585\\
0.025	0.476196196643103\\
0.0255	0.486297191518378\\
0.026	0.496417069413718\\
0.0265	0.506555811376739\\
0.027	0.516713379460263\\
0.0275	0.52688971676034\\
0.028	0.537084747473372\\
0.0285	0.547298376972254\\
0.029	0.557530491901482\\
0.0295	0.567780960291122\\
0.03	0.578049631689525\\
0.0305	0.588336337314655\\
0.031	0.598640890223883\\
0.0315	0.608963085502077\\
0.032	0.619302700467788\\
0.0325	0.62965949489734\\
0.033	0.640033211266585\\
0.0335	0.650423575010099\\
0.034	0.660830294797531\\
0.0345	0.671253062826855\\
0.035	0.681691555134212\\
0.0355	0.692145431920034\\
0.036	0.702614337891134\\
0.0365	0.713097902618396\\
0.037	0.723595740909722\\
0.0375	0.734107453197858\\
0.038	0.744632625942704\\
0.0385	0.755170832047702\\
0.039	0.765721631289893\\
0.0395	0.776284570763194\\
0.04	0.786859185334467\\
0.0405	0.797444998111905\\
0.041	0.80804152092527\\
0.0415	0.8186482548175\\
0.042	0.829264690547188\\
0.0425	0.839890309101429\\
0.043	0.85052458221851\\
0.0435	0.861166972919934\\
0.044	0.871816936051217\\
0.0445	0.882473918830939\\
0.045	0.893137361407474\\
0.0455	0.903806697422854\\
0.046	0.914481354583187\\
0.0465	0.925160755235063\\
0.047	0.935844316947359\\
0.0475	0.946531453097862\\
0.048	0.957221573464119\\
0.0485	0.967914084817915\\
0.049	0.978608391522779\\
0.0495	0.989303896133924\\
0.05	1\\
};
\addlegendentry{t=500s};

\addplot [color=red!30!blue,dashdotted]
  table[row sep=crcr]{%
0.0005	0\\
0.001	0.00524135730404738\\
0.0015	0.0104873917269853\\
0.002	0.0157427765460405\\
0.0025	0.0210121773555081\\
0.003	0.0263002482262919\\
0.0035	0.0316116278666871\\
0.004	0.0369509357848836\\
0.0045	0.0423227684537193\\
0.005	0.0477316954782777\\
0.0055	0.053182255767003\\
0.006	0.0586789537070958\\
0.0065	0.0642262553450529\\
0.007	0.069828584573328\\
0.0075	0.0754903193242138\\
0.008	0.081215787772176\\
0.0085	0.0870092645460141\\
0.009	0.0928749669523733\\
0.0095	0.0988170512122895\\
0.01	0.104839608712614\\
0.0105	0.110946662274333\\
0.011	0.117142162439984\\
0.0115	0.123429983782523\\
0.012	0.129813921238225\\
0.0125	0.136297686466328\\
0.013	0.142884904238376\\
0.0135	0.149579108860355\\
0.014	0.156383740630925\\
0.0145	0.163302142339239\\
0.015	0.170337555805998\\
0.0155	0.177493118471588\\
0.016	0.184771860035317\\
0.0165	0.192176699149903\\
0.017	0.199710440175581\\
0.0175	0.207375769998274\\
0.018	0.215175254916483\\
0.0185	0.223111337601625\\
0.019	0.231186334136712\\
0.0195	0.239402431138336\\
0.02	0.247761682967034\\
0.0205	0.2562660090312\\
0.021	0.264917191189754\\
0.0215	0.27371687125884\\
0.022	0.282666548627867\\
0.0225	0.291767577990214\\
0.023	0.301021167193924\\
0.0235	0.310428375217702\\
0.024	0.319990110277503\\
0.0245	0.32970712806892\\
0.025	0.339580030150544\\
0.0255	0.349609262473348\\
0.026	0.359795114061072\\
0.0265	0.370137715846427\\
0.027	0.380637039667807\\
0.0275	0.391292897431034\\
0.028	0.40210494044047\\
0.0285	0.413072658903641\\
0.029	0.42419538161327\\
0.0295	0.435472275810418\\
0.03	0.446902347232143\\
0.0305	0.458484440346829\\
0.031	0.470217238780055\\
0.0315	0.482099265933547\\
0.032	0.494128885799475\\
0.0325	0.506304303971984\\
0.033	0.518623568857529\\
0.0335	0.531084573085222\\
0.034	0.543685055118016\\
0.0345	0.556422601065199\\
0.035	0.569294646696252\\
0.0355	0.582298479655749\\
0.036	0.595431241878577\\
0.0365	0.608689932204324\\
0.037	0.622071409189295\\
0.0375	0.635572394114195\\
0.038	0.649189474185076\\
0.0385	0.662919105924777\\
0.039	0.676757618751609\\
0.0395	0.690701218741684\\
0.04	0.704745992570825\\
0.0405	0.718887911631618\\
0.041	0.733122836320764\\
0.0415	0.747446520491481\\
0.042	0.761854616065348\\
0.0425	0.776342677797589\\
0.043	0.790906168189445\\
0.0435	0.805540462540926\\
0.044	0.820240854136916\\
0.0445	0.835002559559259\\
0.045	0.849820724117176\\
0.0455	0.864690427388051\\
0.046	0.879606688860389\\
0.0465	0.894564473670457\\
0.047	0.909558698423927\\
0.0475	0.924584237093608\\
0.048	0.939635926984183\\
0.0485	0.954708574754672\\
0.049	0.969796962489245\\
0.0495	0.984895853806844\\
0.05	1\\
};
\addlegendentry{t=200s};





\addplot [color=blue,dashed]
  table[row sep=crcr]{%
0.0005	0\\
0.001	0.00128963377633779\\
0.0015	0.00258478842751139\\
0.002	0.00389099137734999\\
0.0025	0.0052137831045593\\
0.003	0.00655872356246881\\
0.0035	0.00793139846978801\\
0.004	0.00933742542977268\\
0.0045	0.010782459835548\\
0.005	0.0122722005197712\\
0.0055	0.0138123951073456\\
0.006	0.0154088450305264\\
0.0065	0.0170674101664842\\
0.007	0.0187940130582277\\
0.0075	0.0205946426807276\\
0.008	0.0224753577151415\\
0.0085	0.024442289295215\\
0.009	0.0265016431912337\\
0.0095	0.0286597013983281\\
0.01	0.0309228230974949\\
0.0105	0.033297444959394\\
0.011	0.0357900807628225\\
0.0115	0.0384073203017459\\
0.012	0.0411558275568993\\
0.0125	0.0440423381102498\\
0.013	0.0470736557830392\\
0.0135	0.0502566484807083\\
0.014	0.0535982432307308\\
0.0145	0.0571054204022647\\
0.015	0.060785207099552\\
0.0155	0.064644669724161\\
0.016	0.0686909057044647\\
0.0165	0.0729310343941806\\
0.017	0.0773721871453426\\
0.0175	0.0820214965647398\\
0.018	0.0868860849666154\\
0.0185	0.0919730520382693\\
0.019	0.0972894617391299\\
0.0195	0.10284232845784\\
0.02	0.108638602455927\\
0.0205	0.114685154630674\\
0.021	0.120988760633862\\
0.0215	0.127556084387082\\
0.022	0.134393661038329\\
0.0225	0.141507879408525\\
0.023	0.14890496398047\\
0.0235	0.156590956486488\\
0.024	0.164571697154637\\
0.0245	0.172852805676863\\
0.025	0.181439661965699\\
0.0255	0.190337386769271\\
0.026	0.199550822217177\\
0.0265	0.209084512372423\\
0.027	0.218942683866946\\
0.0275	0.229129226700248\\
0.028	0.239647675282401\\
0.0285	0.250501189804035\\
0.029	0.26169253801693\\
0.0295	0.273224077509442\\
0.03	0.285097738561264\\
0.0305	0.297315007661791\\
0.031	0.30987691177582\\
0.0315	0.322784003439238\\
0.032	0.336036346765919\\
0.0325	0.349633504445155\\
0.033	0.363574525806569\\
0.0335	0.377857936026697\\
0.034	0.39248172654821\\
0.0345	0.40744334677906\\
0.035	0.422739697134805\\
0.0355	0.43836712348285\\
0.036	0.454321413042486\\
0.0365	0.470597791789376\\
0.037	0.487190923407502\\
0.0375	0.504094909825716\\
0.038	0.521303293369761\\
0.0385	0.538809060554171\\
0.039	0.556604647531717\\
0.0395	0.574681947211101\\
0.04	0.593032318046532\\
0.0405	0.61164659449555\\
0.041	0.630515099134134\\
0.0415	0.649627656410751\\
0.042	0.668973608013611\\
0.0425	0.688541829818008\\
0.043	0.708320750373356\\
0.0435	0.728298370882352\\
0.044	0.748462286617653\\
0.0445	0.768799709714678\\
0.045	0.789297493272546\\
0.0455	0.809942156688857\\
0.046	0.830719912148073\\
0.0465	0.851616692177631\\
0.047	0.872618178180662\\
0.0475	0.893709829849443\\
0.048	0.914876915359301\\
0.0485	0.936104542238868\\
0.049	0.957377688809181\\
0.0495	0.978681236081324\\
0.05	1\\
};
\addlegendentry{t=100s};


\addplot [color=blue!50!black, densely dotted]
  table[row sep=crcr]{%
0.0005	0\\
0.001	5.54196008843999e-05\\
0.0015	0.000111866432611803\\
0.002	0.000170380662462923\\
0.0025	0.000232028397314256\\
0.003	0.000297914819540485\\
0.0035	0.000369197520287892\\
0.004	0.000447100092638193\\
0.0045	0.000532926044352024\\
0.005	0.000628073086311607\\
0.0055	0.000734047848453457\\
0.006	0.000852481069872879\\
0.0065	0.000985143303870018\\
0.007	0.00113396117196978\\
0.0075	0.00130103419336404\\
0.008	0.00148865220777587\\
0.0085	0.00169931340041725\\
0.009	0.00193574292749634\\
0.0095	0.00220091212962541\\
0.01	0.00249805830849418\\
0.0105	0.00283070502932411\\
0.011	0.00320268289793645\\
0.0115	0.00361815074679553\\
0.012	0.00408161714918748\\
0.0125	0.0045979621648393\\
0.013	0.00517245920386702\\
0.0135	0.00581079687907704\\
0.014	0.00651910069946246\\
0.0145	0.00730395444038856\\
0.015	0.00817242100861865\\
0.0155	0.00913206260318585\\
0.016	0.0101909599563774\\
0.0165	0.0113577304229951\\
0.017	0.012641544670835\\
0.0175	0.0140521417112491\\
0.018	0.0155998419959901\\
0.0185	0.0172955582955771\\
0.019	0.0191508040654437\\
0.0195	0.0211776989994416\\
0.02	0.0233889714661541\\
0.0205	0.0257979575222191\\
0.021	0.0284185961987449\\
0.0215	0.0312654207621807\\
0.022	0.0343535456599218\\
0.0225	0.0376986488736993\\
0.023	0.0413169494206036\\
0.0235	0.0452251797625646\\
0.024	0.0494405529103674\\
0.0245	0.0539807240378557\\
0.025	0.0588637464558872\\
0.0255	0.0641080218337746\\
0.026	0.0697322445982783\\
0.0265	0.0757553404865012\\
0.027	0.0821963992790509\\
0.0275	0.0890746017932352\\
0.028	0.0964091412724817\\
0.0285	0.104219139367146\\
0.029	0.112523556962894\\
0.0295	0.121341100175324\\
0.03	0.130690121892778\\
0.0305	0.140588519312737\\
0.031	0.151053627979976\\
0.0315	0.162102112896106\\
0.032	0.173749857329339\\
0.0325	0.186011850009535\\
0.033	0.19890207144594\\
0.0335	0.212433380152733\\
0.034	0.226617399609664\\
0.0345	0.241464406821006\\
0.035	0.256983223364951\\
0.0355	0.273181109846778\\
0.036	0.290063664682014\\
0.0365	0.307634728139829\\
0.037	0.325896292571642\\
0.0375	0.344848419734957\\
0.038	0.364489166097655\\
0.0385	0.384814516973134\\
0.039	0.405818330291876\\
0.0395	0.427492290760377\\
0.04	0.449825875094137\\
0.0405	0.472806328938031\\
0.041	0.496418656005378\\
0.0415	0.520645619877095\\
0.042	0.545467758805238\\
0.0425	0.570863413761943\\
0.043	0.596808769866274\\
0.0435	0.623277911208904\\
0.044	0.65024288897909\\
0.0445	0.67767380268131\\
0.045	0.705538894111566\\
0.0455	0.733804653646965\\
0.046	0.762435938288282\\
0.0465	0.791396100784997\\
0.047	0.82064712906716\\
0.0475	0.850149795109701\\
0.048	0.879863812263601\\
0.0485	0.909748000005832\\
0.049	0.939760454987275\\
0.0495	0.96985872719566\\
0.05	1\\
};
\addlegendentry{t=50s};

\end{axis}
\end{tikzpicture}%

\caption{\label{velocity}不同时刻速度沿$y$方向的分布}
\end{minipage}
\end{figure}
\textbf{评论:}该模型简捷有效的模拟了剪切流实验,并得到了与实验结果较吻合的结果(平衡后的$F$值及速度分布与实验结果基本一致).
但由于模拟本身的限制,使得该模型及模拟存在缺陷:

\begin{enumerate}
\item 计算所得到的最终平衡值受到步长$dt$及层高$dh$的取值影响,
但当$dh$取值小于一定程度时,其变化对$F$的最终平衡值影响变得很小.当$dt\rightarrow 0$, $dh\rightarrow 0$ 应有$F$收敛于理论解.
\item 当$dt$和$dh$取值不适当时,
某些层会因为在$dt$内速度改变量$a\times dt$过大而改变速度方向,导致系统崩溃. 但可以通过分析来确定$dt$和$dh$取值以避免这种情况.
\item 当$dt$取值较小,层数$n$较大,初速度$U$较大时,计算机需要计算较长的时间. 但就本题所取算例来讲,其计算在PC机上只需几秒.
\end{enumerate}

\noindent\textbf{分离变量法求解}

根据微分方程组式(\ref{diff}), 可用分离变量法求解. 令$u=v+w$, 其中$v=A(t)y + B(t)$满足非齐次边界条件,则$v|_{y=0} = B = 0$, $v|_{y=h} = A(t)h = U$, 故有$v=U/h y$. $w$满足
\[
\begin{dcases}
\frac{\partial w}{\partial t} = v \frac{\partial ^2 w}{\partial y^2} \\
w|_{t=0} = -\frac{U}{h}y\\
w|_{y=0} = 0, {~}u|_{y=h} = 0
\end{dcases}
\]
令$w=Y(y)T(t)$, 则
\[
T' Y = v Y''T \Longrightarrow  \frac{T'}{vT} = \frac{Y''}{Y} = -\lambda(\lambda\geq 0)
\]
故有
\[
\begin{dcases}
Y'' + \lambda Y = 0 \\
Y(0) = 0, Y(h) = 0
\end{dcases}
\]
因此有$\lambda_n = (\frac{n\pi}{h})^2$, $Y_n = \sin\frac{n\pi}{h}y$, $n=1, 2,\cdots$. 代入$w$得
\[
w = \sum_{n=1}^{\infty}T_n(t)\sin\frac{n\pi}{h}y
\]
又由$w|_{t=0} = -\frac{U}{h}y = \sum_{n=1}^\infty\varphi_n\sin\frac{n\pi}{h}y$, 并令$\xi=\frac{\pi}{h}y$, 有
{\setlength\arraycolsep{2pt}
\begin{eqnarray}
\varphi_n & = & \frac{2}{h}\int_0^h\Big(-\frac{U}{h}y\Big)\sin\frac{n\pi}{h}y dy
\nonumber\\
& = & \frac{2}{h}\int_0^\pi\Big(-\frac{U}{h}\frac{h}{\pi}\xi\Big)\sin n\xi\Big(\frac{h}{\pi} d\xi\Big)
\nonumber\\
& = & -\frac{2U}{\pi^2}\int_0^h\xi\sin n\xi d\xi\nonumber\\
& = & - \frac{2U}{n\pi}(-1)^{n+1}\nonumber
\end{eqnarray}}
对于$T_n$有
\[
\begin{dcases}
T_n' = - v\frac{n^2\pi^2}{h^2} \\
T_n|_{t=0} = \varphi_n = -\frac{2U}{n\pi}(-1)^{n+1}
\end{dcases}
\]
故有
\[
T_n = \varphi_n\exp\Big(\frac{-vn^2\pi^2}{h^2}t\Big) = -\frac{2U}{n\pi}(-1)^{n+1}\exp\Big(\frac{-vn^2\pi^2}{h^2}t\Big)
\]
因此有
\[
u = v + w = \frac{U}{h}y + \sum_{n=1}^\infty\Big(-\frac{2U}{n\pi}\Big)(-1)^{n+1}\exp\Big(\frac{-vn^2\pi^2}{h^2}t\Big)\sin\frac{n\pi}{h}y
\]
所以可求得$\tau$:
{\setlength\arraycolsep{2pt}
\begin{eqnarray}
\tau & = & \mu\frac{\partial u}{\partial y}\Big|_{y=h}
\nonumber\\
& = & \mu\frac{U}{h} + \mu\sum_{n=1}^\infty\Big(-\frac{2U}{n\pi}\Big)(-1)^{n+1}\exp\Big(\frac{-vn^2\pi^2}{h^2}t\Big)(-1)^n\frac{n\pi}{h}
\nonumber\\
& = & \mu\frac{U}{h}\Big(1+2\sum_{n=1}^\infty\exp\big(\frac{-vn^2\pi^2}{h^2}t\big)\Big)\nonumber
\end{eqnarray}}
并有$t=0+$,$\tau = \infty$; $t=\infty$,$\tau = \mu \frac{U}{h}$
\end{solution}

\invisiblesubsection{\textcolor{blue}{两肥皂泡合并}}
\newpage
\begin{problem}[问题1.2]
使两个半径分别为$a_1$及$a_2$的球形肥皂泡合并,合并后肥皂泡内的气体温度恢复到初始值,
试证明合并后的肥皂泡半径r满足:
\[
p_0r^3+4\sigma r^2 = p_0(a_1^3+a_2^3)+4\sigma(a_1^2+a_2^2)
\]

其中$p_0$为环境压强, $\sigma$为空气-液体表面处的张力系数.
\end{problem}

\begin{solution}
\textbf{解:}由理想气体方程$pV=nR_0T$,球体体积公式$V=4/3\pi r^3$及液面内外压强差公式$\Delta p = 2\sigma/r+2\sigma/r=4\sigma/r$(内外两层表面:气液表面+液气表面),两球形肥皂泡合并前后有
\[
(p_0+\Delta p_1)V_1 = n_1R_0T, {~} V_1 = \frac{4}{3}\pi a_1^3, {~} \Delta p_1 = \frac{4\sigma}{a_1}
\]
\[
(p_0+\Delta p_2)V_2 = n_2R_0T, {~} V_2 = \frac{4}{3}\pi a_2^3, {~} \Delta p_2 = \frac{4\sigma}{a_2}
\]
\[
(p_0+\Delta p)V = nR_0T, {~} V = \frac{4}{3}\pi r^3, {~} \Delta p = \frac{4\sigma}{r}
\]
又由合并前后的物质的量守恒$n=(n_1+n_2)\Rightarrow nR_0T = (n_1+n_2)R_0T$得:
\[
(p_0+\frac{4\sigma}{r})\frac{4}{3}\pi r^3 = (p_0+\frac{4\sigma}{a_1})\frac{4}{3}\pi a_1^3 + (p_0+\frac{4\sigma}{a_2})\frac{4}{3}\pi a_2^3
\]
将上式整理可得
\[
p_0r^3+4\sigma r^2 = p_0(a_1^3+a_2^3)+4\sigma(a_1^2+a_2^2)
\]
\end{solution}

\invisiblesubsection{漂浮固体相互接近或远离}
\begin{problem}[问题1.3]
若两小固体漂浮在液面上. 试说明, 当两固体均被液体浸润或当它们均不被浸润时, 表面张力
的作用是使相邻两固体相互接近; 当一个是被浸润而另一个不被浸润时, 表面张力是使它们彼此远离.
\end{problem}
\begin{solution}
\begin{figure}[!htb]
\begin{minipage}[b]{.33\textwidth}
\centering
\usetikzlibrary{%
    decorations.pathreplacing,%
    decorations.pathmorphing,arrows
}
\begin{tikzpicture}
\clip (0,-0.1) rectangle(5,2.1);

\fill[blue!25,draw=black] (-0.1,1.1)--(0.75,1.1) arc(-90:0:0.25)-- (2,1.95) arc(-180:0:0.5)--(4,1.35) arc(-180:-90:0.25)  --(5.1,1.1)--(5.1,0)--(-0.1,0)--cycle;
\draw[semithick,fill=gray] (1,1) rectangle (2,2) (3,1) rectangle(4,2);
\draw[white,densely dashed,thick] (1,1.3)--(2,1.3) (3,1.3)--(4,1.3) (1,1.85)--(2,1.85) (3,1.85)--(4,1.85);
\draw[->,semithick, >=stealth',blue] (0.5,1.55)node[left=-3pt]{$p_0$}--(1,1.55);
\draw[<-,semithick, >=stealth',blue](2,1.55)-- (2.5,1.55) node[below]{$p$};
\end{tikzpicture}
\caption{\label{float01}浸润-浸润}
\end{minipage}%
\begin{minipage}[b]{.33\textwidth}
\centering
\usetikzlibrary{%
    decorations.pathreplacing,%
    decorations.pathmorphing,arrows
}
\begin{tikzpicture}
\clip (0,-0.1) rectangle(5,2.1);
\fill[blue!25,draw=black] (-0.1,1.95)--(0.75,1.95) arc(90:0:0.25) -- (2,1.1) arc(180:0:0.5)--(4,1.7) arc(180:90:0.25)  --(5.1,1.95)--(5.1,0)--(-0.1,0)--cycle;
\draw[semithick,fill=gray] (1,1) rectangle (2,2) (3,1) rectangle(4,2);

\draw[white,densely dashed,thick] (1,1.2)--(2,1.2) (3,1.2)--(4,1.2) (1,1.75)--(2,1.75) (3,1.75)--(4,1.75);

\draw[->,semithick, >=stealth',blue] (0.5,1.55)node[left=-3pt]{$p$}--(1,1.55);
\draw[<-,semithick, >=stealth',blue](2,1.55)-- (2.5,1.55) node[below]{$p_0$};
\end{tikzpicture}
\caption{\label{float02}不润湿-不润湿}
\end{minipage}
\begin{minipage}[b]{.33\textwidth}
\centering
\usetikzlibrary{%
    decorations.pathreplacing,%
    decorations.pathmorphing,arrows
}
\begin{tikzpicture}
\clip (0,-0.1) rectangle(5,2.1);

\fill[blue!25,draw=black] (-0.1,1.5)--(0.525,1.5) arc(90:0:0.475)-- (2,1.25) arc(180:110:0.15) --(2.9,1.55) arc(-70:0:0.15)--(4,1.975) arc(-180:-90:0.475)  --(5.1,1.5)--(5.1,0)--(-0.1,0)--cycle;

\draw[semithick,fill=gray] (1,1) rectangle (2,2) (3,1) rectangle(4,2);
\draw[white,densely dashed,thick] (1,1.1)--(2,1.1) (3,1.7)--(4,1.7) (1,1.3)--(2,1.3) (3,1.9)--(4,1.9);

\draw[->,semithick, >=stealth',blue] (0.5,1.2)node[left=-3pt]{$p_0$}--(1,1.2);
\draw[<-,semithick, >=stealth',blue](2,1.2)-- (2.5,1.2) node[below]{$p$};
\end{tikzpicture}
\caption{\label{float03}不润湿-浸润}
\end{minipage}
\end{figure}
\textbf{解:}如图\ref{float01}-\ref{float03}所示,是本题中所要考虑的三种情况.当两物块没有接近时,各自在各个方向上受的到表面张力的水平分量相互平衡,竖值分量与重力及浮力平衡, 表现为静止.
当两物块靠近时,
表面张力有缩小两物块间液体的表面积, 至使小物块内外两侧出现压力差
,结果表现为相互接近或远离.下面就这三种情况一一分析:
\begin{enumerate}
\item \textbf{浸润-浸润}:如图\ref{float01}中两物块,当它们靠近时,它
们之间的凹形液面在表面张力的作用下将缩小表
面积使其上升, $p<p_0$, 因此两物块相互接近.
\item \textbf{不润湿-不润湿}:如图\ref{float02}中两物块,当它们靠近时,它
们之间的凸形液面在表面张力的作用下将缩小表
面积使其下降, $p<p_0$, 因此两物块相互接近.
\item \textbf{浸润-不润湿}:如图\ref{float03}中两物块,当它们靠近时,左物块的右侧凸液面及右物块的左侧凹液面在表面张力的作用下缩小表面积,对于左物块有$p>p_0$受得向左的合力, 同样对于右块可知受到向右的合力, 因此它们彼此远离.
\end{enumerate}
\end{solution}




\newpage
\invisiblesection{第二章{~}流体运动学}
\problemlist{\bf 第二章{~}流体运动学}
\invisiblesubsection{\textcolor{blue}{已知速度求迹线}}
\begin{problem}[问题2.1]
已知速度分布: $u=x-2y$, $v = x-y$. 求初始时刻过$(a,b)$点的质点所经历的轨迹.
\end{problem}
\begin{solution}
\textbf{解:}由迹线微分方程:
\[
\begin{dcases}
\frac{dx}{dt} = u = x-2y \\
\frac{dy}{dt} = v = x -y
\end{dcases}
\]
积分得
\[
\begin{dcases}
y = C_1\cos(t) + C_2\sin(t)\\
x = (C_1+C_2)\cos(t)  - (C_1-C_2)\sin(t)
\end{dcases}
\]
当$t=0$时, $x=a$, $y=b$, 可解得$C_1=b$, $C_2=a-b$. 故初始时刻过$(a,b)$点的质点所经历的轨迹为
\[
\begin{dcases}
y = b\cos(t) + (a-b)\sin(t)\\
x = a\cos(t) + (a-2b)\sin(t)
\end{dcases}
\]
\end{solution} 

\invisiblesubsection{\textcolor{blue}{已知速度求流线}}
\begin{problem}[问题2.2]
已知速度分布: $u=e^{-kt}x$, $v=-e^{-kt}y$. 求流线方程.
\end{problem}
\begin{solution}
\textbf{解:}由流线微分方程
\[
\frac{dx}{u} = \frac{dy}{v} \Longrightarrow \frac{dx}{e^{-kt}x} = \frac{dy}{-e^{-kt}y}
\]
积分得流线方程
\[
\int \frac{dx}{x} = \int \frac{dy}{-y} \Longrightarrow  xy = C
\]
其中$C$为常数.
\end{solution}


\invisiblesubsection{\textcolor{blue}{已知速度求流线, 迹线, 脉线}}
\begin{problem}[问题2.3]
考虑一个二维流场: $u=x(1+2t)$, $v=y$, $w=0$. 试确定并比较:
\begin{itemize}
\item [(a)] $t=0$时刻, 通过点$(1,1)$的流线;
\item [(b)] $t=0$时刻,通过点$(1,1)$的质点的迹线;
\item [(c)] 通过点$(1,1)$的质点在$t$时刻所形成的脉线.
\end{itemize}
\end{problem}
\begin{solution}
\begin{itemize}
\item [(a)]由流线微分方程:
\[
\frac{dx}{u} = \frac{dy}{v} \Longrightarrow \frac{dx}{x(1+2t)} = \frac{dy}{y}
\]
积分得流线方程
\[
\int \frac{dx}{x(1+2t)} = \int \frac{dy}{y} \Longrightarrow y = C x^{1/(1+2t)}
\]
其中$C$为常数.初始条件可得$C=1$, 故$t=0$时刻, 通过点$(1,1)$的流线为
\[
y = x
\]

\item [(b)]由迹线微分方程
\[
\frac{dx}{dt} = u = x(1+2t), {~}  \frac{dy}{dt} = v = y
\]
积分得
\[
x =  C_1 e^{t + t^2}, {~} y = C_2 e^t
\]
由初始条件得$C_1=C_2=1$, 故$t=0$时刻,通过点$(1,1)$的质点的迹线为
\[
x = e^{\ln y + (\ln y)^2}
\]

\item [(c)]由(b)中迹线参数方程
\begin{equation}\label{c}
x =  C_1e^{t + t^2}, {~} y = C_2e^t
\end{equation}
在$t_0$时刻时位置为$(1,1)$, 则有
\[
C_1 = \frac{1}{e^{t_0+t_0^2}}, {~} C_2 =\frac{1}{e^{t_0}}
\]
代入(\ref{c})式,得
\[
x =  \frac{1}{e^{t_0+t_0^2}} e^{t + t^2}, {~} y = \frac{1}{e^{t_0}}e^t
\]
清去$t_0$得通过点$(1,1)$的质点在$t$时刻所形成的脉线
\[
x = \frac{e^{t + t^2}}{e^{(t-\ln y)+(t - \ln y)^2}} = e^{(1+2t)\ln y - (\ln y)^2}
\]
\end{itemize}
\vspace{5pt}

\begin{multicols}{2} 
下面给出流线, 迹线, 脉线的定义及流线, 迹线, 脉线的曲线对比图(如图\ref{fig:StreamPathStreak}, 该图程序见附录\ref{sec:StreamPathStreak}). 对比发现, 对于本题, $t=0$时刻的迹线同时也是$t=0$时刻的脉线.
\begin{itemize}
\item \textbf{流线}是某一相同时刻在流场中画出的一条空间曲线, 在该时刻, 曲线上的所有质点的速度矢量均与这条曲线相切.
\item \textbf{迹线}是单个质点在连续时间过程内的流动轨迹线.
\item \textbf{脉线}是在某一时间间隔内相继经过空间一固定点的流体质点依次串连起来而成的曲线.
\end{itemize}
\begin{center}
\begin{tikzpicture}
\begin{axis}[%
width=0.48\textwidth,
height=0.47\textwidth,
xmin=0,xmax=5,
xlabel={$x$},
ymin=0,ymax=3.7,
ylabel={$y$},
anchor=left of south west,
legend style={legend pos=north west,font=\scriptsize,legend cell align=left},
yticklabel style={font=\scriptsize},
xticklabel style={font=\scriptsize},
xlabel style={font=\footnotesize},
ylabel style={font=\footnotesize,yshift=-15pt},
]
\addplot [no markers,domain=0:4]{x};
\addlegendentry{Stream Line};

\addplot [no markers,densely dashed,domain=0:4,samples=50,green!40!black]({exp(ln(\x)+ln(\x)*ln(\x))},\x);
\addlegendentry{Path Line / Streak Line(t = 0)};

\addplot [no markers,dashed,domain=0:4,samples=80,red]({exp(2*ln(\x)+ln(\x)*ln(\x))},\x);
\addlegendentry{Streak Line(t = 0.5)};

\addplot [no markers,dashdotted,domain=0:4,samples=200,blue]({exp(3*ln(\x)+ln(\x)*ln(\x))},\x);
\addlegendentry{Streak Line(t = 1.0)};
\end{axis}
\end{tikzpicture}%

\captionof{figure}{流线, 迹线, 脉线的比较}\label{fig:StreamPathStreak}
\end{center}
\end{multicols}
\end{solution}


\newpage
\invisiblesection{第三章{~}流体动力学基本方程}
\problemlist{\bf 第三章{~}流体动力学基本方程}
\invisiblesubsection{\textcolor{blue}{N-S方程组的假设条件}}
\begin{problem}[问题3.1]
下列流体力学方程组(N-S方程组)必须满足什么样的假设条件?
\[
\underbrace{\frac{\partial\rho}{\partial t} + \frac{\partial}{\partial x_i}(\rho u_i) = 0}_{1}
\]
\[
\frac{\partial u_i}{\partial t} + u_j\frac{\partial u_i}{\partial x_j}
= \underbrace{- \frac{1}{\rho}\frac{\partial p}{\partial x_i}+\frac{1}{\rho}\frac{\partial p}{\partial x_i} \Big(\lambda\frac{\partial u_k}{\partial x_k}\Big) + \frac{1}{\rho}\frac{\partial}{\partial x_j}\Big[\mu\Big(\frac{\partial u_i}{\partial x_j}+ \frac{\partial u_j}{\partial x_i}\Big)\Big]}_{2} + f_i
\]
\[
\underbrace{C_v\Big(\frac{\partial T}{\partial t} + u_i\frac{\partial T}{\partial x_i}\Big)
}_{3}= \underbrace{-\frac{p}{\rho}\frac{\partial u_i}{\partial x_i}
+ \frac{1}{\rho}\Phi}_{2}
+\frac{1}{\rho}\frac{\partial}{\partial x_i}\Big(\underbrace{k\frac{\partial T}{\partial x_i}}_{4}\Big)
+ q_0
\]
\[
\underbrace{\rho = \rho(p,T)}_{5}
\]
\end{problem}

\begin{solution}
\textbf{解:} 以上四式必须满足以下假设条件:
\begin{enumerate}
\item 连续介质假设
\item 流体本构-牛顿液体 $\sigma_{ij} = -p\delta_{ij}+\lambda S_{kk}\delta_{ij} + 2\mu S_{ij}$.
且满足斯托克斯做的三个假设:
\begin{itemize}
\item 当流体静止时, 应变速率为零, 流体的应力就是静止压强.
\item 应力张量是应变速率张量的线性函数.
\item 流体是各向同性的.
\end{itemize}
\item 完全气体(内能仅是温度的函数), 温度变化不大, $C_v$为常数.
\item 传热本构-傅里叶定律$q_i = -k\frac{\partial T}{\partial x_i}$. 传热各向同性.
\item 均匀系统.
\end{enumerate}

\end{solution} 

\invisiblesubsection{推导不可压流体流动的柱坐标方程}
%%%%%%%%%%%%%%%%%%%%%%%%%%%%%%%%%%%%%%%%%%%%%%%%%%%%%%%%%%%%%%%%%%%%%%%%%%%%%%
\begin{problem}[问题3.2]
推导不可压缩流体流动的柱坐标方程.
\end{problem}
\begin{solution}
%\setlength{\columnseprule}{0.4pt}
%\begin{multicols}{2}
\textbf{解:}分别推导不可压缩流体流动在柱坐标下的连续性方程, 动量方程及能量方程:

\vspace{5pt}
\noindent\textbf{连续性方程}
\vspace{5pt}

\noindent 不可压缩流体流动连续性方程的一般形式为$\nabla\cdot\mathbf{v} = 0$, 代入柱坐标:
\[
\frac{1}{R}\frac{\partial }{\partial R}(Rv_R)
+ \frac{1}{R}\frac{\partial v_{\varphi}}{\partial \varphi} + \frac{\partial v_z}{\partial z} = 0
\]
因此柱坐标下的连续性方程为
\[
\frac{\partial (Rv_R)}{\partial R}+
\frac{\partial v_{\varphi}}{\partial \varphi} + R\frac{\partial v_z}{\partial z} = 0
\]

\vspace{5pt}
\noindent\textbf{动量方程}
\vspace{5pt}

\noindent 不可压缩流体流动动量方程的一般形式为
\[
\underbrace{\frac{d\mathbf{v}}{dt}\vphantom{\frac{1}{\rho}}}_{1} = \underbrace{- \frac{1}{\rho}\nabla p}_{2} + \underbrace{\mathbf{F}\vphantom{\frac{1}{\rho}}}_{3} +\underbrace{\mu\nabla^2\mathbf{v}\vphantom{\frac{1}{\rho}}}_{4}
\]
设柱坐标中的单位矢量为$\mathbf{n}_R$, $ \mathbf{n}_\varphi$, $\mathbf{n}_z$, 现分别求上式中的4项在柱坐标下的形式:

\begin{enumerate}
\item 在柱坐标$\mathbf{v} =\mathbf{v}_R+\mathbf{v}_\varphi + \mathbf{v}_z =v_R\mathbf{n}_R + v_\varphi \mathbf{n}_\varphi+v_z\mathbf{n}_z$, 因此第一项有
{\setlength\arraycolsep{1pt}
\begin{eqnarray}
\frac{d\mathbf{v}}{dt} & = & \frac{d}{dt}\Big(v_R\mathbf{n}_R+v_\varphi \mathbf{n}_\varphi+v_z\mathbf{n}_z\Big)
\nonumber\\
& = & \frac{dv_R}{dt}\mathbf{n}_R + v_R\frac{d\mathbf{n}_R}{dt}
+ \frac{dv_\varphi}{dt}\mathbf{n}_\varphi + v_\varphi\frac{d\mathbf{n}_\varphi}{dt}
+ \frac{dv_z}{dt}\mathbf{n}_z + v_z\frac{d\mathbf{n}_z}{dt}
\nonumber
\end{eqnarray}}
将上式中各项在柱坐标下展开:
\begin{multicols}{2}
\setlength{\abovedisplayskip}{-5pt}
\begin{eqnarray}
v_R\frac{d\mathbf{n}_R}{dt} & = & v_R\Big[\frac{\partial\mathbf{n}_R}{\partial t} + (\mathbf{v}\cdot \nabla)\mathbf{n}_R\Big]\nonumber\\
 & = &  v_R(\mathbf{v}\cdot \nabla)\mathbf{n}_R
\nonumber\\
& = & v_R\Big[
v_R\frac{\partial \mathbf{n}_R}{\partial R} + \frac{v_\varphi}{R}\frac{\partial\mathbf{n}_R}{\partial\varphi}
+ v_z\frac{\partial\mathbf{n}_R}{\partial z}
\Big]\nonumber\\
& = &  \frac{v_Rv_\varphi}{R}\mathbf{n}_\varphi
\nonumber\\
\nonumber\\
v_\varphi\frac{d\mathbf{n}_\varphi}{dt}
& = & v_\varphi\Big[\frac{\partial\mathbf{n}_\varphi}{\partial t} + (\mathbf{v}\cdot \nabla)\mathbf{n}_\varphi\Big]\nonumber\\
 & = &  v_\varphi(\mathbf{v}\cdot \nabla)\mathbf{n}_\varphi
\nonumber\\
& = &
v_\varphi\Big[
v_R\frac{\partial \mathbf{n}_\varphi}{\partial R} + \frac{v_\varphi}{R}\frac{\partial\mathbf{n}_\varphi}{\partial\varphi}
+ v_z\frac{\partial\mathbf{n}_\varphi}{\partial z}
\Big]
\nonumber\\
& = & -\frac{v_\varphi ^2}{R}\mathbf{n}_R
\nonumber\\
\nonumber\\
%
v_z\frac{d\mathbf{n}_z}{dt} & = &
v_z\Big[\frac{\partial\mathbf{n}_z}{\partial t} + (\mathbf{v}\cdot \nabla)\mathbf{n}_z\Big]\nonumber\\
 & = &  v_z(\mathbf{v}\cdot \nabla)\mathbf{n}_z
\nonumber\\
& = &
v_z\Big[
v_R\frac{\partial \mathbf{n}_z}{\partial R} + \frac{v_\varphi}{R}\frac{\partial\mathbf{n}_z}{\partial \varphi}
+ v_z\frac{\partial\mathbf{n}_z}{\partial z}
\Big]
\nonumber\\
& = & 0
\nonumber
\end{eqnarray}

\begin{eqnarray}
\frac{d v_R}{dt}\mathbf{n}_R & = &
   \Big[
       \frac{\partial v_R}{\partial t} + (\mathbf{v}\cdot\nabla)v_R
   \Big]\mathbf{n}_R\nonumber\\
   & = &
   \Big[
       \frac{\partial v_R}{\partial t} +
       v_R\frac{\partial v_R}{\partial R} +\nonumber\\
   & + & \frac{v_\varphi}{R}\frac{\partial v_R}{\partial\varphi}+v_z\frac{\partial v_R}{\partial z}
   \Big]\mathbf{n}_R
      \nonumber\\
   \nonumber\\
   \nonumber\\
      \frac{d v_\varphi}{dt}\mathbf{n}_\varphi & = &
   \Big[
       \frac{\partial v_\varphi}{\partial t} + (\mathbf{v}\cdot\nabla)v_\varphi
   \Big]\mathbf{n}_\varphi\nonumber\\
   & = &
   \Big[
       \frac{\partial v_\varphi}{\partial t} +
       v_R\frac{\partial v_\varphi}{\partial R} +\nonumber\\
   & + & \frac{v_\varphi}{R}\frac{\partial v_\varphi}{\partial\varphi}+v_z\frac{\partial v_\varphi}{\partial z}
   \Big]\mathbf{n}_\varphi
   \nonumber\\
   \nonumber\\
   \nonumber\\
   \frac{d v_z}{dt}\mathbf{n}_z& = &
   \Big[
       \frac{\partial v_z}{\partial t} + (\mathbf{v}\cdot\nabla)v_\theta
   \Big]\mathbf{n}_z\nonumber\\
   & = &
   \Big[
       \frac{\partial v_z}{\partial t} +
       v_R\frac{\partial v_z}{\partial R} +\nonumber\\
   & + & \frac{v_\varphi}{R}\frac{\partial v_z}{\partial\varphi}+v_z\frac{\partial v_z}{\partial z}
   \Big]\mathbf{n}_z
   \nonumber\\
   \nonumber
\end{eqnarray}
\end{multicols}

上三式的推导中, 用到了$\partial\mathbf{n}_R/\partial R = \partial\mathbf{n}_\varphi/\partial R=\partial\mathbf{n}_R/\partial z =  \partial\mathbf{n}_\varphi/\partial z= 0$, $\partial\mathbf{n}_R/\partial\varphi = \mathbf{n}_\varphi$, $\partial\mathbf{n}_\varphi/\partial\varphi = -\mathbf{n}_R$等结论. 最终可以得到
{\setlength\arraycolsep{2pt}
\begin{eqnarray}
\frac{d\mathbf{v}}{dt}
& = & \frac{dv_R}{dt}\mathbf{n}_R
+ \frac{dv_\varphi}{dt}\mathbf{n}_\varphi + \frac{dv_z}{dt}\mathbf{n}_z + \frac{v_Rv_\varphi}{R}\mathbf{n}_\varphi-\frac{v_\varphi ^2}{R}\mathbf{n}_R \nonumber\\
& = &
\Big(\frac{\partial v_R}{\partial t} +
       v_R\frac{\partial v_R}{\partial R} +
    +  \frac{v_\varphi}{R}\frac{\partial v_R}{\partial\varphi}+v_z\frac{\partial v_R}{\partial z} -\frac{v_\varphi ^2}{R}\Big)\mathbf{n}_R + \nonumber\\
& + &
\Big(\frac{\partial v_\varphi}{\partial t} +
       v_R\frac{\partial v_\varphi}{\partial R}
    +  \frac{v_\varphi}{R}\frac{\partial v_\varphi}{\partial\varphi}+v_z\frac{\partial v_\varphi}{\partial z}+\frac{v_Rv_\varphi}{R}\Big)\mathbf{n}_\varphi + \nonumber\\
& +& \Big(\frac{\partial v_z}{\partial t} +
       v_R\frac{\partial v_z}{\partial R} +
    +  \frac{v_\varphi}{R}\frac{\partial v_z}{\partial\varphi}+v_z\frac{\partial v_z}{\partial z}\Big)\mathbf{n}_z
\end{eqnarray}}

\item 对于第二项$-1/\rho\nabla p$有
\begin{equation}
-\frac{1}{\rho}\nabla p =
-\frac{1}{\rho}\frac{\partial p}{\partial R}\mathbf{n}_R
-\frac{1}{\rho}\frac{1}{R}\frac{\partial p}{\partial\varphi}\mathbf{n}_\varphi
-\frac{1}{\rho}\frac{\partial p}{\partial z}\mathbf{n}_z
\end{equation}
\item 对于第三项$\mathbf{F}$, 可表示成柱坐标三个方向上的分量和
\begin{equation}
\mathbf{F} = \mathbf{F}_R + \mathbf{F}_\varphi + \mathbf{F}_z = F_R\mathbf{n}_R + F_\varphi\mathbf{n}_\varphi + F_z\mathbf{n}_z
\end{equation}

%%%%%%%%%%%%%%%%%%%%%%%%%%%%%%%%%%%%%%%%
\item 由$\nabla^2\mathbf{v}=\Delta\mathbf{v}$,可知第四项有
\[
\nabla^2\mathbf{v}= \Big(
\Delta v_R - \frac{v_R}{R^2} - \frac{2}{R^2}\frac{\partial v_\varphi}{\partial\varphi}
\Big)\mathbf{n}_R
+\Big(
\Delta v_\varphi + \frac{2}{R^2}\frac{\partial v_R}{\partial\varphi}-\frac{v_\varphi}{R^2}
\Big)\mathbf{n}_\varphi
+\Delta v_z\mathbf{n}_z
\]

\noindent 其中$\Delta = \frac{\partial}{\partial R^2} +
\frac{1}{R^2}\frac{\partial^2}{\partial\varphi^2}+
\frac{\partial^2}{\partial z^2}+
\frac{1}{R}\frac{\partial}{\partial R}$
%%%%%%%%%%%%%%%%%%%%%%%%%%%%%%%%%%%%%%%%

%\item 由于$\nabla^2\mathbf{v} = \nabla(\nabla\cdot\mathbf{v}) - \nabla\times\nabla\times\mathbf{v}$, 因此求第四项可化为分别求$\mu\nabla(\nabla\cdot\mathbf{v})$和 $\mu\nabla\times\nabla\times\mathbf{v}$
%{\setlength\arraycolsep{2pt}
%\begin{eqnarray}\label{eq1q}
%   \nabla(\nabla\cdot\mathbf{v}) & = &\nabla\Big(
%\frac{1}{R}\frac{\partial}{\partial R}(Rv_R) + \frac{1}{R}\frac{\partial v_\varphi}{\partial\varphi} + \frac{\partial v_z}{\partial z} \Big)\notag \\
% & = &\Big(\frac{\partial}{\partial R}\mathbf{n}_R +\frac{1}{R}\frac{\partial}{\partial\varphi}\mathbf{n}_\varphi + \frac{\partial}{\partial z}\mathbf{n}_z\Big)\Big(
%\frac{1}{R}\frac{\partial}{\partial R}(Rv_R) + \frac{1}{R}\frac{\partial v_\varphi}{\partial\varphi} + \frac{\partial v_z}{\partial z} \Big)\notag \\
% & = &\Big(
%\frac{\partial^2V_R}{\partial R^2} - \frac{1}{R^2}\frac{\partial v_\varphi}{\partial\varphi} + \frac{1}{R}\frac{\partial^2v_\varphi}{\partial\varphi\partial R} +\frac{\partial^2v_z}{\partial\varphi\partial z} - \frac{v_R}{R^2} + \frac{1}{R}\frac{\partial v_R}{\partial\varphi}
%\Big)\mathbf{n}_R + \notag \\
%& + &\Big(
%\frac{1}{R}\frac{\partial^2v_R}{\partial\varphi\partial R}
%+\frac{1}{R^2}\frac{\partial^2v_\varphi}{\partial\varphi^2}
%+\frac{1}{R}\frac{\partial^2v_z}{\partial\varphi\partial z}
%+\frac{1}{R^2}\frac{\partial v_R}{\partial\varphi}
% \Big)\mathbf{n}_\varphi +\notag \\
%& + &\Big(
%\frac{\partial^2v_R}{\partial z\partial R}
%+\frac{1}{R}\frac{\partial^2v_\varphi}{\partial z\partial\varphi}
%\frac{\partial^2v_z}{\partial^2z}
%+\frac{1}{R}\frac{\partial v_R}{\partial z}
% \Big)\mathbf{n}_z\notag\\
%& & \notag\\
%%
%%
%\nabla\times\nabla\times\mathbf{v} & = & \nabla\times
%\Big[
%\big(\frac{1}{R}\frac{\partial v_z}{\partial\varphi} - \frac{\partial v_\varphi}{\partial z}\big)\mathbf{n}_R
%+\big(\frac{\partial v_R}{\partial z} - \frac{\partial v_z}{\partial R}\big)\mathbf{n}_\varphi
%+\frac{1}{R}\big(\frac{\partial}{\partial R}(Rv_\varphi) - \frac{\partial v_R}{\partial\varphi}\big)\mathbf{n}_z
%\Big]
%\notag\\
%& = & \Big(\frac{1}{R^2}\frac{\partial v_\varphi}{\partial\varphi} +
%      \frac{1}{R}\frac{\partial^2v_\varphi}{\partial\varphi\partial R}
%      - \frac{1}{R^2}\frac{\partial^2 v_R}{\partial\varphi^2}
%      -\frac{\partial^2 v_R}{\partial z^2}
%      +\frac{\partial^2 v_z}{\partial z\partial R}\Big)\mathbf{n}_R + \notag\\
%& + & \Big(
%      \frac{1}{R}\frac{\partial^2 v_z}{\partial z\partial\varphi} -
%      \frac{\partial^2v_\varphi}{\partial z^2}
%      + \frac{v_\varphi}{R^2}
%      -\frac{1}{R}\frac{\partial v_\varphi}{\partial R}
%      -\frac{\partial^2 v_\varphi}{\partial R^2}
%      -\frac{1}{R^2}\frac{\partial v_R}{\partial\varphi}
%      +\frac{1}{R}\frac{\partial^2 v_R}{\partial\varphi\partial R}
%      \Big)\mathbf{n}_\varphi + \notag\\
%& + & \Big(
%      -\frac{1}{R^2}\frac{\partial^2 v_z}{\partial\varphi^2} +
%      \frac{1}{R}\frac{\partial^2v_\varphi}{\partial\varphi\partial z}
%      + \frac{1}{R}\frac{\partial v_R}{\partial z}
%      +\frac{\partial^2 v_R}{\partial R\partial z}
%      -\frac{1}{R}\frac{\partial v_z}{\partial R}
%      -\frac{\partial^2 v_z}{\partial R^2}
%      \Big)\mathbf{n}_z
%\end{eqnarray}}
\end{enumerate}
根据式(1-4), 可写出$\mathbf{n}_R$, $\mathbf{n}_\varphi$, $\mathbf{n}_z$各方向上的动量方程
{\setlength\arraycolsep{2pt}
\begin{eqnarray}
\frac{\partial v_R}{\partial t}  +
v_R\frac{\partial v_R}{\partial R} +
\frac{v_\varphi}{R}\frac{\partial v_R}{\partial\varphi}+
v_z\frac{\partial v_R}{\partial z} -
\frac{v_\varphi^2}{R}
& = &
-\frac{1}{\rho}\frac{\partial p}{\partial R} + F_R + \mu\Big(
\Delta v_R-
\frac{v_R}{R^2}-
\frac{2}{R^2}\frac{\partial v_\varphi}{\partial\varphi}
\Big)\nonumber\\
\nonumber\\
\frac{\partial v_\varphi}{\partial t}+
v_R\frac{\partial v_\varphi}{\partial R}+
\frac{v_\varphi}{R}\frac{\partial v_\varphi}{\partial\varphi}+
v_z\frac{\partial v_\varphi}{\partial z} +
\frac{v_Rv_\varphi}{R}
& = &
-\frac{1}{\rho R}\frac{\partial p}{\partial\varphi} + F_\varphi + \mu\Big(
\Delta v_\varphi-
\frac{v_\varphi}{R^2}+
\frac{2}{R^2}\frac{\partial v_R}{\partial\varphi}
\Big)\nonumber\\
\nonumber\\
\frac{\partial v_z}{\partial t}+
v_R\frac{\partial v_z}{\partial R}+
\frac{v_\varphi}{R}\frac{\partial v_z}{\partial\varphi}+
v_z\frac{\partial v_z}{\partial z}
& = &
-\frac{1}{\rho}\frac{\partial p}{\partial z} + F_z + \mu\Delta v_z\nonumber
\end{eqnarray}}

\noindent 其中$\Delta = \frac{\partial}{\partial R^2} +
\frac{1}{R^2}\frac{\partial^2}{\partial\varphi^2}+
\frac{\partial^2}{\partial z^2}+
\frac{1}{R}\frac{\partial}{\partial R}$

\vspace{5pt}
\noindent\textbf{能量方程}
\vspace{5pt}

\noindent 能量方程的一般形式如下
\[
\frac{\partial e}{\partial t} +\mathbf{v}\cdot\nabla e = \frac{1}{\rho}\nabla\cdot(k\nabla T) + \Phi + \dot{q}
\]
其中$\Phi$为耗散函数. 各项在柱坐标下有
\[
\mathbf{v}\cdot\nabla e = (v_R\mathbf{n}_R + v_\varphi\mathbf{n}_\varphi + v_z\mathbf{n}_z)\cdot\Big(\frac{\partial e}{\partial R}\mathbf{n}_R
+ \frac{1}{R}\frac{\partial e}{\partial\varphi}\mathbf{n}_\varphi
+ \frac{\partial e}{\partial z}\Big)\mathbf{n}_z
=
v_R\frac{\partial e}{\partial R}
+ \frac{v_\varphi}{R}\frac{\partial e}{\partial\varphi}
+ v_z\frac{\partial e}{\partial z}
\]
\[
\frac{1}{\rho}\nabla\cdot(k\nabla T) = \frac{k}{\rho}\Delta T= \frac{k}{\rho}
\Big(\frac{\partial}{\partial R^2} +
\frac{1}{R^2}\frac{\partial^2}{\partial\varphi^2}+
\frac{\partial^2}{\partial z^2}+
\frac{1}{R}\frac{\partial}{\partial R}
\Big)T
\]
{\setlength\arraycolsep{2pt}
\begin{eqnarray}\label{phi}
\Phi & = & 2\mu\Big[
\Big(\frac{\partial v_R}{\partial R}\Big)^2 +
\Big(\frac{1}{R}\frac{\partial v_\theta}{\partial\theta} + \frac{v_R}{R}\Big)^2 +
\Big(\frac{\partial v_z}{\partial z}\Big)^2
\Big] + \nonumber\\
& &+\mu\Big[
\Big(\frac{1}{R}\frac{\partial v_z}{\partial\theta} + \frac{\partial v_\theta}{\partial z}\Big)^2 +
\Big(\frac{\partial v_R}{\partial z} + \frac{\partial v_z}{\partial R}\Big)^2 +
\Big(\frac{1}{R}\frac{\partial v_R}{\partial\theta} + \frac{\partial v_\theta}{\partial R}-\frac{v_\theta}{R}\Big)^2
\Big]
\end{eqnarray}}
因此柱坐标下的能量方程为
\[
\frac{\partial e}{\partial t} +v_R\frac{\partial e}{\partial R}
+ \frac{v_\varphi}{R}\frac{\partial e}{\partial\varphi}
+ v_z\frac{\partial e}{\partial z}
= \frac{k}{\rho}
\Big(
\frac{\partial T}{\partial R^2} +
\frac{1}{R^2}\frac{\partial^2 T}{\partial\varphi^2}+
\frac{\partial^2 T}{\partial z^2}+
\frac{1}{R}\frac{\partial T}{\partial R}
\Big) + \Phi + \dot{q}
\]
其中$\Phi$见式(\ref{phi}).
\end{solution} 

\invisiblesubsection{推导不可压流体流动的球坐标方程}
\begin{problem}[问题3.3]
推导不可压缩流体流动的球坐标方程.
\end{problem}
\begin{solution}
\textbf{解:}分别推导不可压缩流体流动在球坐标下的连续性方程, 动量方程及能量方程:

\vspace{0.75em}
\noindent\textbf{连续性方程}
\vspace{0.75em}

\noindent 不可压缩流体流动连续性方程的一般形式为$\nabla\cdot\mathbf{v} = 0$, 代入球坐标得球坐球下续性方程为:
\[
\frac{1}{R^2}\frac{\partial}{R}(R^2v_R) + \frac{1}{R\sin\theta}\frac{\partial}{\partial\theta}(v_\theta\sin\theta) + \frac{1}{R\sin\theta}\frac{\partial v_\varphi}{\partial\varphi} = 0
\]

\vspace{0.75em}
\noindent\textbf{动量方程}
\vspace{0.75em}

\noindent 不可压缩流体流动动量方程的一般形式为
\[
\underbrace{\frac{d\mathbf{v}}{dt}\vphantom{\frac{1}{\rho}}}_{1} = \underbrace{- \frac{1}{\rho}\nabla p}_{2} + \underbrace{\mathbf{F}\vphantom{\frac{1}{\rho}}}_{3} +\underbrace{\mu\nabla^2\mathbf{v}\vphantom{\frac{1}{\rho}}}_{4}
\]
设球坐标中的单位矢量为$\mathbf{n}_R$, $ \mathbf{n}_\theta$, $\mathbf{n}_\varphi$, 现分别求上式中的4项在球坐标下的形式:
\begin{enumerate}
\item 在球坐标下$\mathbf{v} = \mathbf{v}_R + \mathbf{v}_\theta + \mathbf{v}_\varphi = v_R\mathbf{n}_R + v_\theta\mathbf{n}_\theta + v_\varphi\mathbf{n}_\varphi$, 因此第一项
{\setlength\arraycolsep{1pt}
\begin{eqnarray}
\frac{d\mathbf{v}}{dt} & = & \frac{d}{dt}\Big(v_R\mathbf{n}_R+v_\theta \mathbf{n}_\theta+v_\varphi\mathbf{n}_\varphi\Big)
\nonumber\\
& = & \frac{dv_R}{dt}\mathbf{n}_R + v_R\frac{d\mathbf{n}_R}{dt}
+ \frac{dv_\theta}{dt}\mathbf{n}_\theta + v_\theta\frac{d\mathbf{n}_\theta}{dt}
+ \frac{dv_\varphi}{dt}\mathbf{n}_\varphi + v_\varphi\frac{d\mathbf{n}_\varphi}{dt}
\nonumber
\end{eqnarray}}
将上式中的各项在球坐标下展开:
\begin{multicols}{2}
\setlength{\abovedisplayskip}{-5pt}
\begin{eqnarray}
v_R\frac{d\mathbf{n}_R}{dt} & = & v_R\Big[\frac{\partial\mathbf{n}_R}{\partial t} + (\mathbf{v}\cdot \nabla)\mathbf{n}_R\Big]\nonumber\\
 & = &  v_R(\mathbf{v}\cdot \nabla)\mathbf{n}_R
\nonumber\\
& = & v_R\Big[
v_R
\frac{\partial \mathbf{n}_R}{\partial R} + \frac{v_\theta}{R}\frac{\partial\mathbf{n}_R}{\partial\theta}
+ \frac{v_\varphi}{R\sin\theta}\frac{\partial\mathbf{n}_R}{\partial\varphi}
\Big]\nonumber\\
& = &  \frac{v_Rv_\theta}{R}\mathbf{n}_\theta + \frac{v_Rv_\varphi}{R}\mathbf{n}_\varphi
%
\nonumber\\
\nonumber\\
%
v_\theta\frac{d\mathbf{n}_\theta}{dt} & = & v_\theta\Big[\frac{\partial\mathbf{n}_\theta}{\partial t} + (\mathbf{v}\cdot \nabla)\mathbf{n}_\theta\Big]\nonumber\\
 & = &  v_\theta(\mathbf{v}\cdot \nabla)\mathbf{n}_\theta
\nonumber\\
& = & v_\theta\Big[
v_R
\frac{\partial \mathbf{n}_\theta}{\partial R} + \frac{v_\theta}{R}\frac{\partial\mathbf{n}_\theta}{\partial\theta}
+ \frac{v_\varphi}{R\sin\theta}\frac{\partial\mathbf{n}_\theta}{\partial\varphi}
\Big]
\nonumber\\
& = & -\frac{v_\theta^2}{R}\mathbf{n}_R + \frac{v_\varphi v_\theta \cot\theta}{R}\mathbf{n}_\varphi
%
\nonumber\\
\nonumber\\
%
v_\varphi\frac{d\mathbf{n}_\varphi}{dt} & = & v_\varphi\Big[\frac{\partial\mathbf{n}_\varphi}{\partial t} + (\mathbf{v}\cdot \nabla)\mathbf{n}_\varphi\Big]\nonumber\\
 & = &  v_\varphi(\mathbf{v}\cdot \nabla)\mathbf{n}_\varphi
\nonumber\\
& = & v_\varphi\Big[
v_R\frac{\partial \mathbf{n}_\varphi}{\partial R} + \frac{v_\theta}{R}\frac{\partial\mathbf{n}_\varphi}{\partial\theta}
+ \frac{v_\varphi}{R\sin\theta}\frac{\partial\mathbf{n}_\varphi}{\partial\varphi}
\Big]
\nonumber\\
& = & -\frac{v_\varphi^2}{R}\cot\theta\mathbf{n}_\theta - \frac{v_\varphi^2}{R}\mathbf{n}_R
\nonumber
\end{eqnarray}

\begin{eqnarray}
\frac{d v_R}{dt}\mathbf{n}_R & = &
   \Big[
       \frac{\partial v_R}{\partial t} + (\mathbf{v}\cdot\nabla)v_R
   \Big]\mathbf{n}_R\nonumber\\
   & = &
   \Big[
       \frac{\partial v_R}{\partial t} +
       v_R\frac{\partial v_R}{\partial R} +\nonumber\\
   & + & \frac{v_\theta}{R}\frac{\partial v_R}{\partial\theta}+\frac{v_\varphi}{R\sin\theta}\frac{\partial v_R}{\partial\varphi}
   \Big]\mathbf{n}_R
   \nonumber\\
   \nonumber\\
   \nonumber\\
   \frac{d v_\theta}{dt}\mathbf{n}_\theta & = &
   \Big[
       \frac{\partial v_\theta}{\partial t} + (\mathbf{v}\cdot\nabla)v_\theta
   \Big]\mathbf{n}_\theta\nonumber\\
   & = &
   \Big[
       \frac{\partial v_\theta}{\partial t} +
       v_R\frac{\partial v_\theta}{\partial R} +\nonumber\\
   & + & \frac{v_\theta}{R}\frac{\partial v_\theta}{\partial\theta}+\frac{v_\varphi}{R\sin\theta}\frac{\partial v_\theta}{\partial\varphi}
   \Big]\mathbf{n}_\theta
      \nonumber\\
   \nonumber\\
   \nonumber\\
\frac{d v_\varphi}{dt}\mathbf{n}_\varphi & = &
   \Big[
       \frac{\partial v_\varphi}{\partial t} + (\mathbf{v}\cdot\nabla)v_\varphi
   \Big]\mathbf{n}_\varphi\nonumber\\
   & = &
   \Big[
       \frac{\partial v_\varphi}{\partial t} +
       v_R\frac{\partial v_\varphi}{\partial R} +\nonumber\\
   & + & \frac{v_\theta}{R}\frac{\partial v_\varphi}{\partial\theta}+\frac{v_\varphi}{R\sin\theta}\frac{\partial v_\varphi}{\partial\varphi}
   \Big]\mathbf{n}_\varphi
      \nonumber\\
   \nonumber\\
   \nonumber
\end{eqnarray}
\end{multicols}

以上三式的推导中,用到了$\partial\mathbf{n}_R/\partial\theta=\mathbf{n}_\theta$, $\partial\mathbf{n}_\theta/\partial\theta=-\mathbf{n}_R$, $\partial\mathbf{n}_R/\partial\varphi=\sin\theta\mathbf{n}_\varphi$,
$\partial\mathbf{n}_\theta/\partial\varphi=\cos\theta\mathbf{n}_\varphi$,
$\partial\mathbf{n}_\varphi/\partial\varphi=-(\cos\theta\mathbf{n}_\theta + \sin\theta\mathbf{n}_R)$及
$\partial\mathbf{n}_R/\partial R=\partial\mathbf{n}_\theta/\partial R=\partial\mathbf{n}_\varphi/\partial R =\partial\mathbf{n}_\varphi/\partial \theta=0$. 最终可以得到
\begin{eqnarray}
\frac{d\mathbf{v}}{dt}
& = &
\Big(\frac{\partial v_R}{\partial t} +
       v_R\frac{\partial v_R}{\partial R}
    +  \frac{v_\theta}{R}\frac{\partial v_R}{\partial\theta}+\frac{v_\varphi}{R\sin\theta}\frac{\partial v_R}{\partial\varphi}-\frac{v_\theta^2-v_\varphi^2}{R}\Big)\mathbf{n}_R\nonumber\\
& + &
\Big(\frac{\partial v_\theta}{\partial t} +
       v_R\frac{\partial v_\theta}{\partial R}
   + \frac{v_\theta}{R}\frac{\partial v_\theta}{\partial\theta}+\frac{v_\varphi}{R\sin\theta}\frac{\partial v_\theta}{\partial\varphi}+\frac{v_Rv_\theta-v_\varphi^2\cot\theta}{R} \Big)\mathbf{n}_\theta\nonumber\\
& +&
\Big(\frac{\partial v_\varphi}{\partial t} +
       v_R\frac{\partial v_\varphi}{\partial R}
    +  \frac{v_\theta}{R}\frac{\partial v_\varphi}{\partial\theta}+\frac{v_\varphi}{R\sin\theta}\frac{\partial v_\varphi}{\partial\varphi}
+\frac{v_Rv_\varphi+v_\varphi v_\theta \cot\theta}{R}\Big)\mathbf{n}_\varphi
\end{eqnarray}

\item 对于第二项中的$-1/\rho\nabla p$有
\begin{equation}
-\frac{1}{\rho}\nabla p =
-\frac{1}{\rho}\frac{\partial p}{\partial R}\mathbf{n}_R
-\frac{1}{\rho}\frac{1}{R}\frac{\partial p}{\partial\theta}\mathbf{n}_\theta
-\frac{1}{\rho}\frac{1}{R\sin\theta}\frac{\partial p}{\partial \varphi}\mathbf{n}_\varphi
\end{equation}

\item 对于第三项$\mathbf{F}$, 则可表示成三个方向上的分量
\begin{equation}
\mathbf{F} = \mathbf{F}_R + \mathbf{F}_\theta + \mathbf{F}_\varphi = F_R\mathbf{n}_R + F_\theta\mathbf{n}_\theta + F_\varphi\mathbf{n}_\varphi
\end{equation}

\item 由$\nabla^2\mathbf{v}=\Delta\mathbf{v}$,可知第四项可有
\begin{eqnarray}
\nabla^2\mathbf{v}
& = & \Big(
          \Delta v_R -
          \frac{2v_R}{R} -
          \frac{2}{R^2\sin\theta}\frac{\partial(v_\theta\sin\theta)}{\partial\theta} - \frac{2}{R^2\sin\theta}\frac{\partial v_\varphi}{\partial\varphi}
      \Big)\mathbf{n}_R +  \nonumber\\
& + & \Big(
          \Delta v_\theta +
          \frac{2}{R^2}\frac{\partial v_R}{\partial\theta} -
          \frac{v_\theta}{R^2\sin^2\theta} -
          \frac{2\cos\theta}{R^2\sin^2\theta}\frac{\partial v_\varphi}{\partial\varphi}
      \Big)\mathbf{n}_\theta + \nonumber\\
& + & \Big(
          \Delta v_\varphi +
          \frac{2}{R^2\sin\theta}\frac{\partial v_R}{\partial\varphi} +
          \frac{2\cos\theta}{R^2\sin^2\theta}\frac{\partial v_\theta}{\partial\varphi}-
          \frac{v_\varphi}{R^2\sin^2\theta}
      \Big)\mathbf{n}_\varphi
\end{eqnarray}

其中$\Delta = \frac{\partial^2}{\partial R^2} +
\frac{2}{R}\frac{\partial}{\partial R}+
\frac{\cot\theta}{R^2}\frac{\partial}{\partial\theta}+
\frac{1}{R^2}\frac{\partial^2}{\partial\theta^2}+
\frac{1}{R^2\sin^2\theta}\frac{\partial^2}{\partial\varphi^2}$
\end{enumerate}

\noindent 根据式(6-9), 可写出$\mathbf{n}_R$, $\mathbf{n}_\theta$, $\mathbf{n}_\varphi$方向的动量方程
\begin{eqnarray}
\frac{\partial v_R}{\partial t} &+&
v_R\frac{\partial v_R}{\partial R}+
\frac{v_\theta}{R}\frac{\partial v_R}{\partial\theta}
+\frac{v_\varphi}{R\sin\theta}\frac{\partial v_R}{\partial\varphi}-
\frac{v_\theta^2-v_\varphi^2}{R}\nonumber\\
 & = &  -\frac{1}{\rho}\frac{\partial p}{\partial R} + F_R +
      \mu\Big(
          \Delta v_R -
          \frac{2v_R}{R} -
          \frac{2}{R^2\sin\theta}\frac{\partial(v_\theta\sin\theta)}{\partial\theta} - \frac{2}{R^2\sin\theta}\frac{\partial v_\varphi}{\partial\varphi}
      \Big)
\nonumber\\
\nonumber\\
\frac{\partial v_\theta}{\partial t} &+&
v_R\frac{\partial v_\theta}{\partial R}+
\frac{v_\theta}{R}\frac{\partial v_\theta}{\partial\theta}
+\frac{v_\varphi}{R\sin\theta}\frac{\partial v_\theta}{\partial\varphi}+
\frac{v_Rv_\theta-v_\varphi^2\cot\theta}{R}\nonumber\\
 & = & -\frac{1}{\rho}\frac{1}{R}\frac{\partial p}{\partial\theta} + F_\theta+\mu\Big(
          \Delta v_\theta +
          \frac{2}{R^2}\frac{\partial v_R}{\partial\theta} -
          \frac{v_\theta}{R^2\sin^2\theta} -
          \frac{2\cos\theta}{R^2\sin^2\theta}\frac{\partial v_\varphi}{\partial\varphi}
      \Big)
\nonumber\\
\nonumber\\
\frac{\partial v_\varphi}{\partial t} &+&
v_R\frac{\partial v_\varphi}{\partial R}+
\frac{v_\theta}{R}\frac{\partial v_\varphi}{\partial\theta}
+\frac{v_\varphi}{R\sin\theta}\frac{\partial v_\varphi}{\partial\varphi}
+\frac{v_Rv_\varphi+v_\varphi v_\theta \cot\theta}{R}\nonumber\\
& = & -\frac{1}{\rho}\frac{1}{R\sin\theta}\frac{\partial p}{\partial \varphi} + F_\varphi+\mu\Big(
          \Delta v_\varphi +
          \frac{2}{R^2\sin\theta}\frac{\partial v_R}{\partial\varphi} +
          \frac{2\cos\theta}{R^2\sin^2\theta}\frac{\partial v_\theta}{\partial\varphi}-
          \frac{v_\varphi}{R^2\sin^2\theta}
      \Big)
 \nonumber
\end{eqnarray}

\noindent 其中$\Delta = \frac{\partial^2}{\partial R^2} +
\frac{2}{R}\frac{\partial}{\partial R}+
\frac{\cot\theta}{R^2}\frac{\partial}{\partial\theta}+
\frac{1}{R^2}\frac{\partial^2}{\partial\theta^2}+
\frac{1}{R^2\sin^2\theta}\frac{\partial^2}{\partial\varphi^2}$

\vspace{0.75em}
\noindent\textbf{能量方程}
\vspace{0.75em}

\noindent 能量方程的一般形式如下
\[
\frac{\partial e}{\partial t} +\mathbf{v}\cdot\nabla e = \frac{1}{\rho}\nabla\cdot(k\nabla T) + \Phi + \dot{q}
\]
其中$\Phi$为耗散函数. 各项在球坐标下有
{\setlength\arraycolsep{2pt}
\begin{eqnarray}
\mathbf{v}\cdot\nabla e & = & (v_R\mathbf{n}_R + v_\varphi\mathbf{n}_\theta + v_\varphi\mathbf{n}_\varphi)\cdot
\Big(\frac{\partial e}{\partial R}\mathbf{n}_R
+ \frac{1}{R}\frac{\partial e}{\partial\theta}\mathbf{n}_\theta
+ \frac{1}{R\sin\theta}\frac{\partial e}{\partial \varphi}\mathbf{n}_\varphi\Big)\nonumber\\
& = &
v_R\frac{\partial e}{\partial R} +
\frac{v_\theta }{R}\frac{\partial e}{\partial\theta}+
\frac{v_\varphi}{R\sin\theta}\frac{\partial e}{\partial \varphi}\nonumber
\end{eqnarray}}
\[
\frac{1}{\rho}\nabla\cdot(k\nabla T) = \frac{k}{\rho}\Delta T= \frac{k}{\rho}
\Big(
\frac{\partial^2}{\partial R^2} + \frac{2}{R}\frac{\partial}{\partial R}
+ \frac{\cot\theta}{R^2}\frac{\partial}{\partial\theta}+\frac{1}{R^2}\frac{\partial^2}{\partial\theta^2}
+\frac{1}{R^2\sin^2\theta}\frac{\partial^2}{\partial\varphi^2}
\Big)T
\]
{\setlength\arraycolsep{2pt}
\begin{eqnarray}\label{phiSphere}
\Phi  =  \mu
\Big\{
   2& \big[ &
        \big(\frac{\partial v_R}{\partial R}\big)^2
        +\big(\frac{1}{R}\frac{\partial v_\theta}{\partial\theta}+\frac{v_R}{R}\big)^2
        +\big(\frac{1}{R\sin\theta}\frac{\partial v_\varphi}{\partial\varphi}+\frac{v_R}{R}+\frac{v_\theta\cot\theta}{R}\big)^2
    \big]\nonumber\\
+ & \big[ &
        \frac{1}{R\sin\theta}\frac{\partial v_\theta}{\partial\varphi}
        +\frac{\sin\theta}{R}\frac{\partial}{\partial\theta}\big(\frac{v_\varphi}{\sin\theta}\big)
    \big]^2
+  \big[
         \frac{1}{R\sin\theta}\frac{\partial v_R}{\partial\varphi}
        +R\frac{\partial}{\partial R}\big(\frac{v_\varphi}{R}\big)
    \big]^2\nonumber\\
+ & \big[ &
        R\frac{\partial}{\partial R}\big(\frac{v_\theta}{R}\big)
        +\frac{1}{R}\frac{\partial v_R}{\partial\theta}
    \big]^2
\Big\}
\end{eqnarray}}
因此球坐标下的能量方程为
{\setlength\arraycolsep{2pt}
\begin{eqnarray}
\frac{\partial e}{\partial t}
+ v_R\frac{\partial e}{\partial R} +
\frac{v_\theta }{R}\frac{\partial e}{\partial\theta}+
\frac{v_\varphi}{R\sin\theta}\frac{\partial e}{\partial \varphi}
=
\frac{k}{\rho}
\Big(
\frac{\partial^2T}{\partial R^2} &+& \frac{2}{R}\frac{\partial T}{\partial R}
+ \frac{\cot\theta}{R^2}\frac{\partial T}{\partial\theta}+\frac{1}{R^2}\frac{\partial^2T}{\partial\theta^2}\nonumber\\
&+&\frac{1}{R^2\sin^2\theta}\frac{\partial^2T}{\partial\varphi^2}
\Big)
+\Phi
+\dot{q}
\end{eqnarray}}
其中$\Phi$见式(\ref{phiSphere}).
\end{solution} 


\newpage
\invisiblesection{第四章{~}流体静力学}
\problemlist{\bf 第四章{~}流体静力学}
\invisiblesubsection{\textcolor{blue}{加速容器中的液体倾角}}
\begin{problem}[问题4.1]
求一加速运动的容器中盛有的液体自由面与水平面的夹角.
\begin{center}
%\includegraphics[width=0.3\textwidth]{./homework04/problem01.pdf}
\usetikzlibrary{%
    decorations.pathreplacing,%
    decorations.pathmorphing,arrows
}
\begin{tikzpicture}[ media/.style={font={\footnotesize\sffamily}},
    interface/.style={
        postaction={draw,decorate,decoration={border,angle=-45,
                    amplitude=0.3cm,segment length=2mm}}},scale=1.5]
\draw[thick,interface](0,0)--(3,0);


\fill[blue!20](0.5,1)--(0.5,0.05)--(2,0.05)--(2,0.75);

\draw[semithick] (0.5,1.25)--(0.5,0.05)--(2,0.05)--(2,1.25);
\draw[blue, semithick] (0.5,1)--(2,0.75);
\draw[blue,dashed](2,0.75)--(0.75,0.75);
\draw (1,0.75) arc(180:170:1) node[blue,above]{$\theta$};

\draw [semithick,->,>=stealth',blue] (2.1,0.5)--(2.75,0.5) node[right]{$a$};



\end{tikzpicture}

\end{center}
\end{problem}

\begin{solution}
\textbf{解:} 均加速运动的液体相对静止, 压强除重力场外, 水平方向的惯性力也有贡献
\begin{eqnarray}
\nabla p & = & -\rho\nabla gz-\rho\nabla ax\nonumber\\
         & = & -\rho\nabla(gz+ax)\nonumber\\
         & = & -\nabla(\rho gz+\rho ax)\nonumber
\end{eqnarray}
积分得
\[
p + \rho gz+\rho ax = \mathrm{const}
\]
对于液面,压强为$p_0$, 因此对于液面上的点$(x,z)$有
\[
p_0 + \rho gz+\rho ax = \mathrm{const}\Rightarrow \frac{d(p_0 + \rho gz+\rho ax)}{dx} = 0
\]
可解得液面的斜率$dz/dx$:
\[
\frac{dz}{dx}= -\frac{a}{g}
\]
因此, 液体自由面与水平面的夹角为$\theta = \arctan (a/g)$
\end{solution} 

\invisiblesubsection{自引力形星的压力}
%%%%%%%%%%%%%%%%%%%%%%%%%%%%%%%%%%%%%%%%%%%%%%%%%%%%%%%%%%%%%%%%%%%%%%%%%%%%%%
\begin{problem}[问题4.2]
有一自引力球形星,其密度随距中心距离r的变化如下:
\[
\rho = \rho_0(1-\beta r^2)
\]
试推导出在中心处压力的表达式, 并证明如果平均密度为表面密度的2倍, 则中心处压力是具有均匀密度且有同样总质量的星球中心处压力的13/8倍.
\end{problem}

\begin{solution}
\textbf{解:}设引力球形星半径为$R$, 与该形星球心距离为$r$的一个球形区域内的总质量为$m(r)$. 则有
\begin{equation}\label{dmdr}
\frac{dm(r)}{dr} = 4\pi r^2\rho(r)
\end{equation}
代入密度函数并积分
\[
m(r) = \int_0^r4\pi r'^2\rho_0(1-\beta r'^2)dr'
= 4\pi\rho_0\int_0^r r'^2-\beta r'^4dr'
= 4\pi\rho_0(\frac{1}{3}r^3-\frac{1}{5}\beta r^5 )
\]
该形星质量为$m(R) = 4\pi\rho_0(\frac{1}{3}R^3-\frac{1}{5}\beta R^5 )$. 下面分别来完成题中要求的推导和证明:

\begin{itemize}
\item \textbf{推导中心处压力的表达式}

由流体静力学平衡方程
\[
\nabla p = Gm(r)\rho(r)\nabla \frac{1}{r} \Longrightarrow \frac{dp(r)}{dr}= -\frac{Gm(r)}{r^2}\rho_0(1-\beta r^2)
\]
其中$G$为引力常数, $Gm(r)/r^2$即为距离球心$r$处的重力加速度处的重力加速度. 对上式积分可得
\begin{eqnarray}
p(r) & = & \int -\frac{Gm(r)}{r^2}\rho_0(1-\beta r^2) dr \nonumber\\
     & = & -4G\pi\rho_0^2\int\Big(\frac{1}{3}r-\frac{1}{5}\beta r^3\Big)(1-\beta r^2) dr\nonumber\\
     & = & -4G\pi\rho_0^2\int \frac{1}{5}\beta^2r^5 + \frac{1}{3}r -\frac{8}{15}\beta r^3 dr\nonumber\\
     & = & -4G\pi\rho_0^2\Big(\frac{1}{30}\beta^2r^6 +\frac{1}{6}r^2 -\frac{2}{15}\beta r^4\Big) + C\nonumber\\
     & = & -\frac{2G\pi r^2\rho_0(\beta^2r^4 - 4\beta r^2 + 5)}{15} + C\nonumber
\end{eqnarray}
代入定解条件$p(R)=0$得
\[
-\frac{2G\pi R^2\rho_0(\beta^2R^4 - 4\beta R^2 + 5)}{15} + C \Longrightarrow
C = \frac{2G\pi R^2\rho_0(\beta^2R^4 - 4\beta R^2 + 5)}{15}
\]
因此可得压强
\[
p(r) = 4G\pi\rho_0^2\Big(\frac{2}{15}\beta r^4 - \frac{1}{30}\beta^2r^6 -\frac{1}{6}r^2\Big) + \frac{2G\pi R^2\rho_0(\beta^2R^4 - 4\beta R^2 + 5)}{15}
\]
代入$r=0$可得中心处的压强
\begin{equation}\label{p_0}
p(0) = \frac{2G\pi R^2\rho_0(\beta^2R^4 - 4\beta R^2 + 5)}{15}
\end{equation}

\item \textbf{证明若平均密度为表面的2倍, 则中心压力是同质量密度均匀的星球的13/8倍}

由于该形星平均密度为表面密度的2倍, 故有
\[
\frac{m(R)}{4/3\pi R^3} = 2 \rho(R) \Longrightarrow
\frac{4\pi\rho_0(R^3/3-\beta R^5/5)}{4/3\pi R^3} = 2 \rho_0(1-\beta R^2)
\]
解得$R=\sqrt{5/(7\beta)}$, 代入式(\ref{dmdr})和(\ref{p_0})得
\[
m(R) = \frac{80\pi\sqrt{35}\rho_0}{1029\beta^{3/2}}, {~~}
p(0) = \frac{260G\pi\rho_0^2}{1029\beta}
\]

具有均匀密度且有同样总质量的星球(下称星球2)的密度为
\[
\rho' = \frac{m(R)}{4/3\pi R^3} = \frac{4}{7}\rho_0
\]
与星球2球心距离为$r$的一个球形区域内的总质量为$m'(r)$
\[
m'(r) = \rho'\frac{4}{3}\pi r^3 = \frac{4}{7}\rho_0\frac{4}{3}\pi r^3 = \frac{16}{21}\rho_0\pi r^3
\]
设具有均匀密度且有同样总质量的星球$r$处的压强为$p'(r)$, 则有
\[
\nabla p'(r) = Gm'(r)\rho'\nabla \frac{1}{r} \Longrightarrow \frac{dp'(r)}{dr}= -\frac{Gm'(r)\rho'}{r^2}
\]
积分得
\begin{eqnarray}
p'(r) & = & \int -\frac{Gm'(r)\rho'}{r^2}dr\nonumber\\
      & = & -G\int \frac{16}{21}\rho_0\pi r^3 \frac{4}{7}\rho_0\frac{1}{r^2} dr\nonumber\\
      & = & -G\frac{4^3}{7^3}\pi\rho_0^2\int r dr\nonumber\\
      & = & -G\frac{32}{147}\pi\rho_0^2r^2 + C'\nonumber
\end{eqnarray}
由$p'(R)=0$代入$R=\sqrt{5/(7\beta)}$得
\[
-G\frac{32}{147}\pi\rho_0^2\frac{5}{7\beta} + C' = 0
\]
得$c'=\frac{160G\pi\rho_0^2}{1029\beta}$, 因此
\[
p'(0) = C' = \frac{160G\pi\rho_0^2}{1029\beta}
\]
比较$p(0)$及$p'(0)$有
\[
\frac{p'(0)}{p(0)} = \frac{160G\pi\rho_0^2}{1029\beta}\Big/\frac{260G\pi\rho_0^2}{1029\beta} = \frac{8}{13}
\]
因此, 如果平均密度为表面密度的2倍, 则中心处压力是具有均匀密度且有同样总质量的星球中心处压力的13/8倍.
\end{itemize}
\end{solution} 

\invisiblesubsection{\textcolor{blue}{L型管内液体加速度和压强}}
\begin{problem}[问题4.3]如右图所示, L型等截面管的A
\vspace{-2em}

\begin{multicols}{2}
~

B和BC两段各长为L, 内充满不可压缩理想流体,
初始时刻C端封闭. 某时刻C端突然打开, 求:
\begin{enumerate}
\item 此刻管内流体的加速度大小.
\item 沿管的压强分布.
\end{enumerate}
\textbf{思考:}如果竖直管道截面积是水平管道的两倍, 情况又如何变化.
\begin{center}
\usetikzlibrary{%
    decorations.pathreplacing,%
    decorations.pathmorphing,arrows
}
\begin{tikzpicture}[scale=1.75]
\draw[semithick](0,2)--(0,0)--(2,0);
\draw[semithick](0.15,2)node[right]{$A$}--(0.15,0.15)node[above right]{$B$}--(2,0.15) node[above]{$C$};
\fill[red] (2,0.075) node[right]{$o$}circle(0.03);
\draw[densely dashed,red,->,>=stealth'] (2,0.075)--(0.075,0.075)--(0.075,1.5) node[right]{$s$};
\draw[blue](-0.05,2)--(-0.3,2) (-0.05,0)--(-0.3,0) (0,-0.05)--(0,-0.3) (2,-0.05)--(2,-0.3);

\draw[blue,<-,>=stealth'] (-0.175,0)--(-0.175,0.75) node[above]{$L$}; 
\draw[blue,->,>=stealth'] (-0.175,1.25)--(-0.175,2); 

\draw[blue,<-,>=stealth'] (0,-0.175)--(0.75,-0.175) node[right]{$L$}; 
\draw[blue,->,>=stealth'] (1.25,-0.175)--(2,-0.175); 
\end{tikzpicture}

\end{center}
\end{multicols}
\end{problem}

\begin{solution}
\textbf{解:}
如图中所示, 以$C$为原点,建立沿L型等截面管的自然坐标系$s$. 因为L型管AB段(或BC段)等截面, 因此液体在AB段(或BC段)任意点的速度和加速度都相等. 即$\frac{\partial v}{\partial s} = 0$. 故速度势
\[
\varphi(s,t) =\begin{dcases}
 v_{AB}(t)s & s \leq L \\
v_{AB}(t)L + v_{BC}(t)(s-L) & s > L
\end{dcases}
\]
对$t=0$时刻的$A,C$及任意一点$s$应用Bernoulli方程
\begin{equation}
\frac{\partial\varphi_A}{\partial t} + \frac{1}{2}v_A^2 + \frac{p_A}{\rho} + gz_A = \frac{\partial\varphi_C}{\partial t} + \frac{1}{2}v_C^2 + \frac{p_C}{\rho} + gz_C =
\frac{\partial\varphi_s}{\partial t} + \frac{1}{2}v_s^2 + \frac{p_s}{\rho} + gz_s
\end{equation}
$t=0$时刻的$A,C$及任意一点$s$的速度$v_A = v_C = v_s=0$. 故上式可化为
\[
L\frac{\partial v_A}{\partial t} + L\frac{\partial v_C}{\partial t} + gL = 0
\]

\[
\begin{dcases}
 s\frac{\partial v_s}{\partial t} + \frac{p_s}{\rho} = \frac{p_C}{\rho} = \frac{p_C}{\rho}=\frac{p_0}{\rho} & s \leq L \\
 (s-L)\frac{\partial v_s}{\partial t} + L\frac{\partial v_C}{\partial t} + \frac{p_s}{\rho} + g(s-L) =  \frac{p_C}{\rho} = \frac{p_0}{\rho} & s > L
\end{dcases}
\]
由于$dv_A/dt=\partial v_A/\partial t + v_A \partial v_A/\partial s = \partial v_A/\partial t = a_A$.同理$\partial v_C/\partial t = a_C$ 因此有
\begin{equation}\label{a}
L\frac{d v_A}{d t} + L\frac{d v_C}{d t} + gL = 0 \Longrightarrow a_A + a_C = -g
\end{equation}
\begin{equation}\label{p}
p_s = \begin{dcases}
p_0 - s\rho a_C & s \leq L \\
p_0 - \rho[(s-L)a_A + La_C + g(s-L)]& s > L
\end{dcases}
\end{equation}
下面分别考虑以下两种情况
\begin{itemize}
\item \textbf{竖直管道截面积 = \textcolor{white}{2}水平管道的面积:} $a_C = a_A$

将$a_C = a_A$代入式(\ref{a})得液体加速度
\[
a = a_C = a_A = - \frac{1}{2}g
\]
其中负号表示液体加速度与自然坐标系$s$的正方向相反. 将加速度代入式(\ref{p})得
\[
p_s = 
\begin{dcases}
p_0 + \frac{1}{2}\rho gs & s \leq L \\
p_0 + \frac{1}{2}\rho g(2L-s) & s > L
\end{dcases}
\]

\item \textbf{竖直管道截面积 = 2水平管道的面积:} $a_C = 2a_A$


将$a_C = 2a_A$代入式(\ref{a})得液体加速度
\[
a_C =  - \frac{2}{3}g,{~~} a_A = - \frac{1}{3}g
\]
其中负号表示液体加速度与自然坐标系$s$的正方向相反. 将加速度代入式(\ref{p})得
\[
p_s = \begin{dcases}
p_0 + \frac{2}{3}\rho gs & s \leq L \\
p_0 + \frac{2}{3}\rho g(2L-s) & s > L
\end{dcases}
\]

\end{itemize}



\end{solution} 


\newpage
\invisiblesection{第五章{~}无粘流动的一般理论}
\problemlist{\bf 第五章{~}无粘流动的一般理论}
\invisiblesubsection{虹吸管的流量}
\begin{problem}[问题5.1]一虹吸管放于水桶中, 其位置如
\vspace{-2em}
\begin{multicols}{2}
~

右图所示. 如果水桶及虹吸管的截面积分别为$A$和$B$, 且$A\gg B$, 试计算虹吸管的流量. 水看作是理想不可压缩的, 且受重力作用, 运动是定常的.
\begin{center}
\usetikzlibrary{%
    decorations.pathreplacing,%
    decorations.pathmorphing,arrows
}
\begin{tikzpicture}[scale=1.35]
\fill[blue!20](-0.25,1)--(-0.25,0)--(1.65,0)--(1.65,1);
\draw[blue](-0.25,1)--(1.65,1) node[midway,above]{$A$};
\draw[semithick](-0.25,1.5)--(-0.25,0)--(1.65,0)--(1.65,1.5);

\draw[red](0.95,0.75) arc(180:90:0.75 and 1) arc(90:0:0.5) --(2.2,0.25) ;

\draw[red](1.05,0.75) arc(180:90:0.65 and 0.9) arc(90:0:0.4) --(2.1,0.25)node[below]{$B$};

\draw[blue](2.25,1)--(2.6,1) (2.25,0.25)--(2.6,0.25);

\draw[blue,<->,>=stealth'] (2.425,0.25)--(2.425,1) node[above]{$h$}; 

\end{tikzpicture}

%\includegraphics[width=0.28\textwidth]{./figures/problem01.pdf}
\end{center}
\end{multicols}
\end{problem}

\begin{solution}
\textbf{解:}对$A$, $B$面应用Bernoulli方程
\[
\frac{1}{2}v_A^2 + \frac{p_A}{\rho} + gz_A =
\frac{1}{2}v_B^2 + \frac{p_B}{\rho} + gz_B
\]
由质量守恒$Av_A = Bv_B$及$A\gg B$, 可知$v_A\rightarrow 0$. 又知$p_A = p_B = p_0$及$z_A-z_B = h$. 因此上式可化为
\[
\frac{1}{2}v_B^2 = g(z_A-z_B) = gh \Longrightarrow  v_B = \sqrt{2gh}
\]
因此,$A$,$B$两液面高度差为$h$时流量为
\[
q = Bv_B = B\sqrt{2gh}
\]
若考虑到随着水流从$B$口不断流出, $A$, $B$液面差$H(t)$(为也$h$区分,这里使用$H(t)$, $H(0) = h$)将非常缓慢的变小, $v_B(t)$也缓慢变化. 有如下关系
\[
Av_A = Bv_B\Longrightarrow -A\frac{d(H)}{dt} = Bv_B(t) \Longrightarrow A\frac{dH}{dt} = -B\sqrt{2gH}
\Longrightarrow \frac{dH}{\sqrt{H}} = -\frac{B}{A}\sqrt{2g}dt
\]
考虑到$h(0)=h$积分得
\[
H(t) = \Big[\sqrt{h}-\frac{B}{2A}\sqrt{2g}t\Big]^2
\]
因此, 流量是时间的函数
\[
q(t) = B\sqrt{2gH(t)} = B\Big(\sqrt{2gh}-\frac{B}{A}gt\Big)
\]

\end{solution} 

\newpage
\invisiblesubsection{水箱小孔射流截面面积}
\begin{problem}[问题5.2]一水箱底部有一小孔, 射流的截

\vspace{-2em}
\begin{multicols}{2}
~

面积为$A(x)$, 在小孔处$x=0$, 截面积为$A_0$. 通过不断注水使水箱中水高$h$保持常数, 水箱的横截面远比小孔的大. 设流体是理想, 不可压缩的, 求射流截面积随x的变化规律$A(x)$.
\begin{center}
%\includegraphics[width=0.25\textwidth]{./figures/problem02.pdf}
\usetikzlibrary{%
    decorations.pathreplacing,%
    decorations.pathmorphing,arrows
}
\begin{tikzpicture}[scale=1.35]
\fill[blue!20](-0.25,1)--(-0.25,0)--(1.65,0)--(1.65,1);
\draw[blue](-0.25,1)--(1.65,1) node[midway,above]{$c$};
\draw[semithick](-0.25,1.5)--(-0.25,0)--(1.65,0)--(1.65,1.5);

\draw[semithick,red](0.95,0.75) arc(180:90:0.75 and 1) --(2.25,1.75) (1.05,0.75) arc(180:90:0.65 and 0.9)--(2.25,1.65);
\draw[semithick,red,->,>=stealth'] (2.25,1.5)--(1.8,1.5);

\draw[blue](2,1)--(2.35,1) (2,0)--(2.35,0);

\draw[blue,<->,>=stealth'] (2.175,0)--(2.175,1) node[above]{$h$}; 

\draw[blue,->,>=stealth'] (0.25,0)node[above]{$o$}--(0.25,-0.65) node[above left]{$x$} ; 

\draw[red,semithick](0.45,0) arc(90:0:0.2 and 0.65) (0.95,0) arc(90:180:0.2 and 0.65) node[above right]{$A(x)$};

\end{tikzpicture}

\end{center}
\end{multicols}
\end{problem}

\begin{solution}
\textbf{解:}设水箱的液面为c, 则对$c$, $o$, 及任意射流面A(x)处应用Bernoulli方程
\[
\frac{1}{2}v_c^2 + \frac{p_c}{\rho} - gx_c =
\frac{1}{2}v_o^2 + \frac{p_o}{\rho} - gx_o =
\frac{1}{2}v_x^2 + \frac{p_x}{\rho} - gx
\]
由于$v_c=0$, $p_c = p_o = p_x$, $x_c=-h$,$x_o = 0$,因此上式可化为
\[
gh = \frac{1}{2}v_o^2 = \frac{1}{2}v_x^2 -gx
\]
因此可求得任意射流$x$处的射流速度
\[
v_x = \sqrt{2g(h+x)} , {~~} v_o = \sqrt{2gh}
\]
又因为$A(x)v_x = A_ov_o=A_o\sqrt{2gh}$, 因此可求得射流面积$A(x)$
\[
A(x) = A_o\sqrt{\frac{h}{h+x}}
\]
\end{solution} 


\newpage
\invisiblesection{第六章{~}无粘不可压缩流体的无旋流动}
\problemlist{\bf 第六章{~}无粘不可压缩流体的无旋流动}
\invisiblesubsection{\textcolor{blue}{已知速度分布求速度势及环流量}}
\begin{problem}[问题6.1]
已知下列两个速度分布
\begin{equation}\label{eq60101}
u = \frac{cx}{x^2 + y^2},{~} v = \frac{cy}{x^2+y^2}
\end{equation}
\begin{equation}\label{eq60102}
u = \frac{-cy}{x^2 + y^2},{~} v = \frac{cx}{x^2+y^2}
\end{equation}
其中$c$为常数. 求:
\begin{enumerate}
\item 速度势$\varphi$, 流函数$\psi$和复速度势$W$, 并画出等势线和流线;
\item 围绕坐标原点作一封闭曲线, 求沿此封闭曲线的环量$\Gamma$及流量$Q$.
\end{enumerate}
\end{problem}
%%%%%%%%%%%%%%%%%%%%%%%%%%%%%%%%%%%%%%%%%%%%%%%%%%%%%%%%%%%%%%%%%%%%%%
\begin{solution}
\textbf{解:} 分别对两个速度分布求解(左右栏分别为两个速度分布求解过程)
\begin{enumerate}
\item
由速度势函数及流函公式$\varphi = \int udx + vdy$, $\psi = \int -vdx + udy$. 代入速度分布可求得速度势函数, 流函及复速度势
\setlength{\columnseprule}{0.4pt}
\begin{multicols}{2}
\setlength{\abovedisplayskip}{-10pt}
\begin{eqnarray}
\varphi &=& c\int \frac{xdx + ydy}{x^2+y^2}\nonumber\\
 &=& \frac{c}{2}\ln(x^2+y^2) + C_1\nonumber\\
 & &\nonumber\\
\psi &=& c\int \frac{-ydx + xdy}{x^2+y^2}\nonumber\\
     &=& c\arctan\frac{y}{x} + C_2\nonumber\\
     & &\nonumber\\
W &=& c\Big[\ln\sqrt{x^2+y^2} + i\arctan\frac{y}{x}\Big]\nonumber\\
  &=& c(\ln r + i\theta)\nonumber
\end{eqnarray}

%%%%%%%%%%%%%%%%%%%%%%%%%%%%%%%%%%%%%%%%%%%%%%%%%%%%%%%%%%%%%%%%%%%%%

\begin{eqnarray}
\varphi &=& c\int \frac{-ydx + xdy}{x^2+y^2}\nonumber\\
 &=& c\arctan\frac{y}{x} + C_1\nonumber\\
 & &\nonumber\\
\psi &=& c\int \frac{-xdx - ydy}{x^2+y^2}\nonumber\\
     &=& -\frac{c}{2}\ln(x^2+y^2) + C_2\nonumber\\
     & &\nonumber\\
W &=& c\Big[\arctan\frac{y}{x}- i\ln\sqrt{x^2+y^2}\Big]\nonumber\\
  &=& c(\theta - i\ln r)\nonumber
\end{eqnarray}
\end{multicols}
式(\ref{eq60101})和式(\ref{eq60102})对应的的等势线和流线分别如图\ref{p01n01}和\ref{p01n02}所示, 程序见附录\ref{sec:cPotentStream}.
\begin{figure}[!htb]
\begin{minipage}[b]{.5\textwidth}
\centering
%\includegraphics[width=0.55\textwidth]{./figures/p1f1.pdf}
\pgfplotsset{compat=1.7}
\begin{tikzpicture}
\begin{axis}[width=0.9\textwidth,height=0.9\textwidth,
xlabel={$x$},
ytick={-5,0,5},xtick={-5,0,5},
%yticklabels={0.035,0.1,0.2,0.3,0.4,0.5},
ylabel={$y$},
xmin=-5,xmax=5,ymin=-5,ymax=5, 
yticklabel style={font=\scriptsize},
xticklabel style={font=\scriptsize},
xlabel style={font=\footnotesize},
ylabel style={font=\footnotesize},
legend style={font=\scriptsize,legend cell align=left},
]

\foreach \x in {1,...,5} {
  \edef\temp{\noexpand\draw[blue,dashed] (axis cs:0,0) circle [radius=\x];}
  \temp
}

\foreach \t in {0,...,20}{
  \edef\temp{\noexpand\draw[red] (axis cs:0,0)--++({18*\t}:1000);}
  \temp
}


\addplot[dashed, color=blue,domain=0:2*pi,samples=360,smooth]({0.01*sin(deg(\x))}, {0.01*cos(deg(\x))});
\addlegendentry{$\varphi = K_1$};


\addplot[red,domain=-0.01:0.01,samples=2] {0};
\addlegendentry{$\psi = K_2$};

\end{axis}
\end{tikzpicture}%

\caption{\label{p01n01}式(1)对应的等势线和流线}
\end{minipage}%
\begin{minipage}[b]{.5\textwidth}
\centering
\pgfplotsset{compat=1.7}
\begin{tikzpicture}
\begin{axis}[width=0.9\textwidth,height=0.9\textwidth,
xlabel={$x$},
ytick={-5,0,5},xtick={-5,0,5},
%yticklabels={0.035,0.1,0.2,0.3,0.4,0.5},
ylabel={$y$},
xmin=-5,xmax=5,ymin=-5,ymax=5, 
yticklabel style={font=\scriptsize},
xticklabel style={font=\scriptsize},
xlabel style={font=\footnotesize},
ylabel style={font=\footnotesize},
legend style={font=\scriptsize,legend cell align=left},
]

\foreach \x in {1,...,5} {
  \edef\temp{\noexpand\draw[red] (axis cs:0,0) circle [radius=\x];}
  \temp
}

\foreach \t in {1,...,20}{
  \edef\temp{\noexpand\draw[blue,dashed] (axis cs:0,0)--++({18*\t}:1000);}
  \temp
}

\addplot[dashed,color=blue,domain=-0.01:0.01,samples=2] {0};
\addlegendentry{$\varphi = K_1$};

\addplot[red,domain=0:2*pi,samples=360,smooth]({0.01*sin(deg(\x))}, {0.01*cos(deg(\x))});
\addlegendentry{$\psi = K_2$};

\end{axis}
\end{tikzpicture}%

%\includegraphics[width=0.55\textwidth]{./figures/p1f2.pdf}
\caption{\label{p01n02}式(2)对应的等势线和流线}
\end{minipage}
\end{figure}

\item 沿封闭曲线的环量和流量
\[
\Gamma = \oint_c udx + vdy = \oint \frac{\partial \varphi}{\partial x}dx
+ \frac{\partial \varphi}{\partial y}dy = \oint d\varphi
\]
\[
Q=\oint_c udy-vdx=\oint \frac{\partial \psi}{\partial y}dy
+ \frac{\partial \psi}{\partial x}dx = \oint d\varphi
\]
代入速度分布得

\setlength{\columnseprule}{0.4pt}
\begin{multicols}{2}
\setlength{\abovedisplayskip}{-10pt}
\begin{eqnarray}
\Gamma &=& \oint d\varphi = c\ln r\Big|_{-}^{+}= 0\nonumber\\
& &\nonumber\\
Q &=& \oint d\psi=c\theta\Big|_{-}^{+} = 2c\pi\nonumber
\end{eqnarray}
%%%%%%%%%%%%%%%%%%%%%%%%%%%%%%%%%%%%%%%%%%%%%

\begin{eqnarray}
\Gamma &=& \oint d\varphi=c\theta\Big|_{-}^{+}=2c\pi\nonumber\\
& &\nonumber\\
Q &=& \oint d\psi = -c\ln r\Big|_{-}^{+}=  0\nonumber
\end{eqnarray}
\end{multicols}
\end{enumerate}
\end{solution} 

\invisiblesubsection{\textcolor{blue}{已知势函数求速度和流函数}}
\begin{problem}[问题6.2]
理想, 不可压缩, 定常流动, 在极坐标中势函数为
\[
\varphi = r^{1/2}\cos\frac{\theta}{2}
\]
求速度分量与流函数, 并分析流函数图案.
\end{problem}
\begin{solution}
\textbf{解:} 设速度在径向和法向的分量分别为$v_r$, $v_\theta$, 则有
\[
v_r = \frac{\partial \varphi}{\partial r} =\frac{1}{2\sqrt{r}}\cos\frac{\theta}{2}
,\qquad
v_\theta = \frac{1}{r}\frac{\partial \varphi}{\partial\theta} = -\frac{1}{2\sqrt{r}}\sin\frac{\theta}{2}
\]
\vspace{-1.5em}

\begin{multicols}{2}
\noindent 又由$v_r=\frac{1}{r}\frac{\partial\psi}{\partial \theta}$,
 $v_\theta=-\frac{\partial\psi}{\partial r}$得
\[
\frac{1}{r}\frac{\partial\psi}{\partial \theta}= \frac{1}{2\sqrt{r}}\cos\frac{\theta}{2}
,\quad
-\frac{\partial\psi}{\partial r}=-\frac{1}{2\sqrt{r}}\sin\frac{\theta}{2}
\]
\noindent 因此有
\[
\psi = r^{1/2}\sin\frac{\theta}{2}
\]
对于流线和流函数图案,可令
\[
\varphi = r^{1/2}\cos\frac{\theta}{2}=K_1,\quad
\psi = r^{1/2}\sin\frac{\theta}{2}=K_2
\]
其中$K_1$, $K_2$为常数, 则可得到一组流线和流函数图案, 如图\ref{fig:polarStream}所示, 程序见附录\ref{sec:cPotentStream}. 从图上可知, 流线与流函数线是始终垂直的, 因此已知流线, 流函数即可确定.

\vspace{2em}

\begin{center}
\begin{tikzpicture}
\begin{polaraxis}[width=0.45\textwidth,height=0.45\textwidth,ymin=0,ymax=200,
yticklabel style={font=\scriptsize},
xticklabel style={font=\scriptsize},
xlabel style={font=\footnotesize},
ylabel style={font=\footnotesize},
legend style={font=\scriptsize,legend cell align=left},]
	\addplot+[mark=none,domain=-170:170,samples=102,smooth,blue,dashed]{2/(cos(0.5*\x)^2)}; 
        \addlegendentry{$\varphi = K_1$};
	\addplot+[mark=none,domain=10:350,samples=102,smooth,red]{2/(sin(0.5*\x)^2)}; 
        \addlegendentry{$\psi = K_2$};

	\addplot+[mark=none,domain=-170:170,samples=34,smooth,blue,dashed]{20/(cos(0.5*\x)^2)}; 
	\addplot+[mark=none,domain=-170:170,samples=34,smooth,blue,dashed]{40/(cos(0.5*\x)^2)}; 
	\addplot+[mark=none,domain=-170:170,samples=34,smooth,blue,dashed]{60/(cos(0.5*\x)^2)}; 
	\addplot+[mark=none,domain=-170:170,samples=34,smooth,blue,dashed]{80/(cos(0.5*\x)^2)}; 
	\addplot+[mark=none,domain=-170:170,samples=34,smooth,blue,dashed]{100/(cos(0.5*\x)^2)}; 
	\addplot+[mark=none,domain=-170:170,samples=34,smooth,blue,dashed]{120/(cos(0.5*\x)^2)}; 
	\addplot+[mark=none,domain=-170:170,samples=34,smooth,blue,dashed]{140/(cos(0.5*\x)^2)}; 
	\addplot+[mark=none,domain=-170:170,samples=34,smooth,blue,dashed]{160/(cos(0.5*\x)^2)}; 
	\addplot+[mark=none,domain=-170:170,samples=34,smooth,blue,dashed]{180/(cos(0.5*\x)^2)}; 


	\addplot+[mark=none,domain=10:350,samples=34,smooth,red]{20/(sin(0.5*\x)^2)}; 
	\addplot+[mark=none,domain=10:350,samples=34,smooth,red]{40/(sin(0.5*\x)^2)}; 
	\addplot+[mark=none,domain=10:350,samples=34,smooth,red]{60/(sin(0.5*\x)^2)}; 
	\addplot+[mark=none,domain=10:350,samples=34,smooth,red]{80/(sin(0.5*\x)^2)}; 
	\addplot+[mark=none,domain=10:350,samples=34,smooth,red]{100/(sin(0.5*\x)^2)}; 
	\addplot+[mark=none,domain=10:350,samples=34,smooth,red]{120/(sin(0.5*\x)^2)}; 
	\addplot+[mark=none,domain=10:350,samples=34,smooth,red]{140/(sin(0.5*\x)^2)}; 
	\addplot+[mark=none,domain=10:350,samples=34,smooth,red]{160/(sin(0.5*\x)^2)}; 
	\addplot+[mark=none,domain=10:350,samples=34,smooth,red]{180/(sin(0.5*\x)^2)}; 
\end{polaraxis}
\end{tikzpicture}


\captionof{figure}{流线, 流函数图案}\label{fig:polarStream}
\end{center}
\end{multicols}
\end{solution}

\invisiblesubsection{\textcolor{blue}{点源和点涡的叠加复速度势}}
\begin{problem}[问题6.3]
已知在原点处有一强度为$Q$的源和强度为$\Gamma$的点涡, 求它们共同诱导的复
速度势$W$, 并绘出示意图.
\end{problem}

\begin{solution}
\textbf{解:} 原点处有一强度为$Q$的源和强度为$\Gamma$的点涡可分别表示为
\[
W(z)_Q = \frac{Q}{2\pi}\ln z, {~~~} W(z)_\Gamma = \frac{\Gamma}{2\pi i}\ln z
\]
共同诱导的复
速度势$W$可由点源和点涡的复
速度势线性叠加
\begin{eqnarray}
W(z)&=& W(z)_Q + W(z)_\Gamma = \frac{Q}{2\pi}\ln z + \frac{\Gamma}{2\pi i}\ln z\nonumber\\
&=& \frac{\ln z}{2\pi}[Q-i\Gamma]=\frac{1}{2\pi}[Q-i\Gamma](\ln r + i\theta)\nonumber\\
&=& \frac{1}{2\pi}\Big[Q\ln r + \Gamma\theta \Big] + \frac{i}{2\pi}\Big[Q\theta-\Gamma\ln r\Big]\nonumber
\end{eqnarray}
\vspace{-1em}

\begin{multicols}{2}
\noindent 因此势函数和流函数分别为
\[
\varphi = \frac{1}{2\pi}\Big[Q\ln r + \Gamma\theta \Big]
\]
\[
\psi = \frac{1}{2\pi}\Big[Q\theta-\Gamma\ln r\Big]
\]
令$\varphi=K_1$及$\psi=K_2$, 即
\[
Q\ln r + \Gamma\theta = K_1'
\]
\[
Q\theta-\Gamma\ln r = K_2'
\]
可得势函数和流函数等值线, 如图\ref{fig:polarStream63}所示(这里假设$Q>0$, $\Gamma>0$), 程序见附录\ref{sec:cPotentStream}.

\begin{center}
%\includegraphics[width=0.3\textwidth]{./figures/p3.pdf}
\begin{tikzpicture}
\begin{polaraxis}[width=0.45\textwidth,height=0.45\textwidth,ymin=0,ymax=1,
yticklabel style={font=\scriptsize},
xticklabel style={font=\scriptsize},
xlabel style={font=\footnotesize},
ylabel style={font=\footnotesize},
legend style={font=\scriptsize,legend cell align=left},]
	\addplot+[mark=none,domain=0:8*pi,samples=100,blue,dashed,smooth](deg(\x),{10*exp(-\x)}); 
        \addlegendentry{$\varphi = K_1$};
	\addplot+[mark=none,domain=-8*pi:2*pi,samples=100,red,solid,smooth](deg(\x), {0.01*exp(\x)}); 
        \addlegendentry{$\psi = K_2$};


	\addplot+[mark=none,domain=0:8*pi,samples=100,blue,dashed,smooth](deg(\x+pi/3),{10*exp(-\x)}); 
	\addplot+[mark=none,domain=-8*pi:2*pi,samples=100,red,solid,smooth](deg(\x+pi/3), {0.01*exp(\x)}); 

	\addplot+[mark=none,domain=0:8*pi,samples=100,blue,dashed,smooth](deg(\x+2*pi/3),{10*exp(-\x)}); 
	\addplot+[mark=none,domain=-8*pi:2*pi,samples=100,red,solid,smooth](deg(\x+2*pi/3), {0.01*exp(\x)}); 

	\addplot+[mark=none,domain=0:8*pi,samples=100,blue,dashed,smooth](deg(\x+pi),{10*exp(-\x)}); 
	\addplot+[mark=none,domain=-8*pi:2*pi,samples=100,red,solid,smooth](deg(\x+pi), {0.01*exp(\x)}); 

	\addplot+[mark=none,domain=0:8*pi,samples=100,blue,dashed,smooth](deg(\x+4*pi/3),{10*exp(-\x)}); 
	\addplot+[mark=none,domain=-8*pi:2*pi,samples=100,red,solid,smooth](deg(\x+4*pi/3), {0.01*exp(\x)}); 

	\addplot+[mark=none,domain=0:8*pi,samples=100,blue,dashed,smooth](deg(\x+5*pi/3),{10*exp(-\x)}); 
	\addplot+[mark=none,domain=-8*pi:2*pi,samples=100,red,solid,smooth](deg(\x+5*pi/3), {0.01*exp(\x)}); 

\end{polaraxis}
\end{tikzpicture}


\captionof{figure}{流线, 流函数图案}\label{fig:polarStream63}
\end{center}
\end{multicols}

\end{solution} 

\invisiblesubsection{\textcolor{blue}{求均匀流场中半圆柱的拔力}}
\begin{problem}[问题6.4]已知沿地面有一均匀来流, 流经
\vspace{-2em}
\begin{multicols}{2}
~

半径为$R$的半圆柱形暖棚, 求其所受到的
拔力.(设流动为理想不可压缩无旋流动, 暖棚内气压与无穷远处相等)

\begin{center}
\definecolor{cff0000}{RGB}{255,0,0}
\definecolor{cdb0024}{RGB}{219,0,36}
\definecolor{cb70049}{RGB}{183,0,73}
\definecolor{c92006d}{RGB}{146,0,109}
\definecolor{c6d0092}{RGB}{109,0,146}
\definecolor{c4900b7}{RGB}{73,0,183}
\definecolor{c2400db}{RGB}{36,0,219}
\definecolor{cd20000}{RGB}{210,0,0}
\begin{tikzpicture}[y=0.80pt, x=0.8pt,scale=0.75]
            \begin{scope}[cm={{4.55209,0.0,0.0,1.23779,(201.25,226.12)}}]
              \begin{scope}[xscale=0.125,yscale=0.459]
                      \begin{scope}[shift={(5.7805,24.704)}]
                        \path[draw=cdb0024,smooth,semithick]  (0.0000,0.0000)--
                           (79.1710,0.6840)-- 
                          (118.6300,1.5880) --
                           (143.1000,2.5410) --
                           (159.6800,3.6890) --
                          (175.9790,5.3990) --
                           (187.6600,7.3780) --
                           (196.0700,9.3570) --
                          (202.3690,11.3600) --
                          (209.1990,14.0710) --
                         (214.8090,16.7340) --
                          (219.6800,19.3480) --
                          (224.1690,21.9130) --
                          (228.7290,24.5760) --
                           (234.2900,27.8000) --
                         (238.1490,29.9010) --
                           (243.8800,32.8570) --
                          (247.7290,34.6160) --
                          (253.5400,36.9360) --
                          (257.3890,38.2560) --
                           (263.2900,39.8920) --
                           (269.1990,41.0890) --
                          (275.2190,41.8220) --
                         (281.2400,42.1150) --
                         (287.2700,41.9930) --
                          (293.2700,41.3580) --
                          (299.2400,40.3570) --
                          (305.1490,38.8170) --
                          (308.9990,37.6200) --
                          (314.8500,35.4220) --
                           (320.2400,32.9790) --
                           (324.4600,30.9270) --
                           (328.8800,28.4840) --
                           (334.1690,25.5280) --
                          (338.1990,23.0850) --
                          (342.5090,20.6670) --
                           (347.1000,18.0530) --
                         (352.3390,15.3660) --
                          (358.5090,12.7030) --
                          (366.2400,10.0410) --
                           (374.5400,7.8910) --
                           (383.8990,6.0590) --
                          (396.7290,4.4220) --
                          (416.1000,2.8830) --
                           (440.7290,1.7590) --
                          (472.0700,0.9040) --
                          (512.2900,0.3420) --
                          (563.8990,0.0000) --
                          (565.3200,0.0000);
                      \end{scope}
                      \begin{scope}[shift={(5.7805,46.324)}]
                        \path[draw=cb70049,smooth,semithick] 
                         (0.0000,0.0000) --
                         (77.1710,1.2940)-- 
                          (113.6800,2.6620) --
                           (129.6300,3.5660) --
                          (152.8090,5.5690) --
                           (166.6600,7.2550) --
                           (177.1000,8.9160) --
                          (187.0490,10.9440) --
                         (196.2400,13.1670) --
                          (205.4900,15.8290) --
                           (214.1990,18.6140) --
                           (220.1990,20.6660) --
                          (225.9300,22.5960) --
                         (233.9990,25.2590) --
                          (239.1690,26.8960) --
                          (246.7790,29.0450) --
                           (254.2190,30.8780) --
                          (259.1000,31.8300) --
                           (266.4190,32.9780) --
                           (273.7590,33.7110) --
                           (281.0200,34.0040) --
                          (288.3390,33.8330) --
                          (295.6090,33.2470) --
                           (302.9300,32.2940) --
                           (309.5890,30.9750) --
                          (315.2400,29.6810) --
                          (322.7590,27.7020) --
                           (327.8500,26.1140) --
                           (335.8090,23.5000) --
                          (341.3690,21.6190) --
                          (349.4900,18.8340) --
                          (356.8800,16.4650) --
                           (364.2400,14.2900) --
                          (372.6600,12.0430) --
                          (382.7100,9.8200) --
                           (395.1990,7.6700) --
                          (409.1690,5.7890) --
                          (422.1190,4.5430) --
                           (450.5090,2.6620) --
                           (470.4900,1.8810) --
                          ( (509.0490,0.8060) --
                           (561.3390,0.0490) --
                          (565.3200,0.0000);
                      \end{scope}
                      \begin{scope}[shift={(5.7805,67.943)}]
                        \path[draw=c92006d,smooth,semithick] (0.0000,0.0000) --
                         (41.9020,0.7330) -- 
                          (102.6090,2.9070) --
                           (121.7590,4.0310) --
                          (138.6090,5.3990) --
                         (155.7100,7.2070) --
                          (169.5590,9.0140) --
                           (181.4190,10.8950) --
                          (194.8290,13.4600) --
                         (204.3200,15.4390) --
                           (212.0700,17.1490) --
                         (222.8090,19.5920) --
                           (229.5090,21.0580) --
                         (239.1690,22.9870) --
                           (245.3690,24.1110) --
                          (254.3890,25.5520) --
                          (263.2900,26.5540) --
                          (272.0490,27.2380) --
                         (280.7790,27.4820) --
                          (289.5400,27.3600) --
                          (298.2700,26.8470) --
                          (307.1490,25.9430) --
                          (316.1000,24.6240) --
                          (322.2400,23.5490) --
                          (331.7290,21.7420) --
                          (340.5090,19.8120) --
                          (348.7100,18.0040) --
                         (356.1990,16.2940) --
                          (368.6300,13.6310) --
                          (378.1000,11.7990) --
                           (388.8990,9.9910) --
                           (401.5090,8.1100) --
                           (416.8290,6.3030) --
                           (436.1990,4.5440) --
                           (460.9490,2.9560) --
                           (481.1690,2.0520) --
                          (522.5590,0.7330) --
                           (557.1490,0.1220) --
                          (565.3200,0.0000);
                      \end{scope}
                      \begin{scope}[shift={(5.7805,89.465)}]
                        \path[draw=c6d0092,smooth,semithick] (0.0000,0.0000)  -- (42.5360,0.9520) -- (50.1950,1.1720) -- (68.6580,1.8070)--
                          (92.3410,2.8330) --
                           (113.8500,4.0790) --
                           (133.7290,5.5450) --
                          (152.3390,7.2550) --
                           (169.9990,9.3560) --
                         (181.7100,10.8950) --
                         (192.2190,12.4050) --
                         (206.2900,14.5850) --
                         (214.9300,16.0050) --
                       (227.0200,17.8050) --
                          (234.6300,18.8850) --
                           (245.7100,20.2550) --
                          (256.3690,21.3250) --
                         (265.9790,21.9650) --
                          (273.6800,22.3050) --
                           (283.9790,22.4050) --
                           (294.2400,22.1850) --
                          (301.1690,21.8350) --
                          (311.6800,20.9850) --
                           (322.4190,19.8650) --
                          (333.5890,18.3750) --
                           (341.4390,17.2450) --
                          (353.8090,15.3150) --
                           (362.6600,13.8950) --
                          (376.4190,11.7950) --
                         (388.4390,10.1380) --
                          (400.8800,8.5500) --
                           (415.2400,6.9620) --
                           (432.2190,5.4470) --
                           (452.9990,3.9080) --
                          (479.3390,2.4910) --
                           (514.3200,1.1720) --
                          (551.0700,0.2680) --
                          (563.4900,0.0000) -- (565.3200,0.0000);
                      \end{scope}
                      \begin{scope}[shift={(5.7805,110.91)}]
                        \path[draw=c4900b7,smooth,semithick] 
                           (0.0000,0.0000)  --
                           (74.1710,2.2000)
                           -- (100.3890,3.4500) --
                           (121.8090,4.8200) --
                          (139.8090,6.1800) --
                         (155.2900,7.5500) --
                           (173.0700,9.2400) --
                          (187.0490,10.8200) --
                         (197.8800,12.0200) --
                          (212.8290,13.7800) --
                          (223.7290,14.9300) --
                          (235.3690,16.1000) --
                          (248.0200,17.1300) --
                         (260.2900,17.8600) --
                          (272.2700,18.3200) --
                           (284.2400,18.4500) --
                           (296.1690,18.2000) --
                          (308.2700,17.7100) --
                          (320.6600,16.8600) --
                           (333.4900,15.7100) --
                        (342.3890,14.7600) --
                          (356.4900,13.2200) --
                           (368.0490,11.8500) --
                          (383.1690,10.2100) --
                          (395.4900,8.8500) --
                         (413.7100,7.0900) --
                           (433.2900,5.5000) --
                          (452.8800,4.1300) --
                          (476.5090,2.8400) --
                           (505.9300,1.5900) --
                          (543.0490,0.5200) --
                         (565.3200,0.0000);
                      \end{scope}
                      \begin{scope}[shift={(5.7805,137.17)}]
                        \path[draw=c2400db,smooth,semithick] (0.0000,0.0000) -- 
                          (34.1950,0.8600)-- (96.5400,3.3000) --
                           (119.7100,4.6000) --
                           (137.7590,5.7400) --
                           (153.8290,6.8700) --
                            (175.0700,8.5300) --
                            (187.7790,9.5600) --
                           (205.3200,11.0200) --
                           (216.4600,11.8300) --
                            (231.8500,12.9500) --
                            (246.7290,13.8100) --
                            (261.1000,14.3200) --
                           (275.2190,14.5900) --
                            (289.2400,14.5900) --
                           (303.3890,14.3200) --
                            (317.7590,13.8100) --
                            (332.6300,12.9500) --
                           (348.0200,11.8300) --
                            (359.1490,11.0200) --
                           (376.6800,9.5600) --
                           (389.4600,8.5300) --
                           (410.6300,6.8700) --
                           (426.7100,5.7400) --
                           (444.7790,4.6000) --
                         (467.9300,3.3000) --
                           (497.9790,2.0100) --
                           (530.2900,0.8600) --
                           (565.3200,0.0000);
                      \end{scope}
                      \begin{scope}[shift={(235.98,-2.7534)}]
                      \draw[red,semithick](-230,5)--(0,5) arc(-180:-360:52)--(335,5);
                      \end{scope}
              \end{scope}
\end{scope}
\end{tikzpicture}

\end{center}
\end{multicols}
\end{problem}

\begin{solution}
\textbf{解:} 该问题可看作无环量圆住绕流问题的上半平面, 故其复速度势为
\[
W(z) = v_\infty\big(z+\frac{R^2}{z}\big)
\]
对应的势函数和流函数分别为
\[
\varphi = v_\infty\big(r+\frac{R^2}{r}\big)\cos\theta,{~~~} \psi = v_\infty\big(r+\frac{R^2}{r}\big)\sin\theta
\]
因此速度在$r$和$\theta$上的速度分量分别为
\[
v_r = \frac{\partial\varphi}{\partial r} = v_\infty\big(1-\frac{R^2}{r^2}\big)\cos\theta,{~~~}
v_\theta = \frac{1}{r}\frac{\partial\varphi}{\partial\theta} = -v_\infty\big(1+\frac{R^2}{r^2}\big)\sin\theta
\]
半圆柱形暖棚表面有$v_r=0$, $v_\theta=-2v_\infty\sin\theta$, 因此对半圆柱形暖棚表面和无穷远处
应用伯努力方程
\[
\frac{v_\infty^2}{2} + \frac{p_\infty}{\rho} = \frac{v_\theta^2}{2} + \frac{p_\theta}{\rho}
{~~}\Longrightarrow{~~} p_\theta = p_\infty + \frac{\rho}{2}v_\infty^2(1-4\sin^2\theta)
\]
因此拔力为
\begin{eqnarray}
F_Y &=& \int_0^\pi (p_\infty - p_\theta)\cdot R\sin\theta d\theta\nonumber\\
&=& \frac{1}{2} \rho Rv_\infty^2\int_0^\pi (4\sin^2\theta-1)\sin\theta d\theta\nonumber\\
&=& \frac{1}{2} \rho Rv_\infty^2\int_0^\pi (4\cos^2\theta-3)d(\cos\theta)\nonumber\\
&=& \frac{1}{2} \rho Rv_\infty^2\Big[
\frac{4}{3}\cos^3\theta-3\cos\theta
\Big]_0^\pi\nonumber\\
&=& \frac{5}{3}\rho Rv_\infty^2\nonumber
\end{eqnarray}
\end{solution}

\invisiblesubsection{\textcolor{blue}{镜像法求复速度势}}
\begin{problem}[问题6.5]
直角域内$z_0$点存在强度为$\Gamma$的点涡, 求其复速度势.
\end{problem}

\begin{solution}
\begin{multicols}{2}
\textbf{解:} 利用镜像法, 分别以实轴虑轴为边界, 在$z_0$的对称点引入相应的点涡, 如图\ref{fig:RectPointVortex}所示. 则有
\begin{eqnarray}
W(z) &=& +\frac{\Gamma}{2\pi i}\ln(z-z_0) -\frac{\Gamma}{2\pi i}\ln(z-\overline{z_0})\nonumber\\
     & & -\frac{\Gamma}{2\pi i}\ln(z+\overline{z_0})+\frac{\Gamma}{2\pi i}\ln(z+z_0)\nonumber\\
     &=& \frac{\Gamma}{2\pi i}\ln\Big[\frac{z^2-z_0^2}{z^2-\overline{z_0}^2}\Big]\nonumber
\end{eqnarray}

\vspace{0.1em}
\begin{center}
\usetikzlibrary{%
    decorations.pathreplacing,%
    decorations.pathmorphing,arrows
}
\begin{tikzpicture}[ media/.style={font={\footnotesize\sffamily}},
    interface/.style={
        postaction={draw,decorate,decoration={border,angle=-45,
                    amplitude=0.3cm,segment length=2mm}}},scale=1.25]
\draw[semithick,interface](1,1.5)--(1,0)--(3,0);
\draw[semithick,->,>=stealth'](3,0)--(3.25,0) node[right]{$x$};
\draw[semithick,->,>=stealth'](1,1.5)--(1,1.75) node[above]{$y$};
\draw[semithick,blue,densely dashed](-1,0)--(1,0)--(1,-1.25);

\fill(2,0.75) circle(0.05) node[right]{$z_0$};
\draw[semithick,red,->,>=stealth'](2,1.25) arc(90:270:0.5);
\fill[gray,draw=black](-0.25,0.75) circle(0.05) node[left,black]{$-\bar{z}_0$};
\draw[semithick,red,->,>=stealth',dashed](-0.25,1.25) arc(90:-90:0.5);

\fill[gray,draw=black](-0.25,-0.75) circle(0.05) node[left,black]{$-z_0$};
\draw[semithick,red,->,>=stealth',dashed](-0.25,-1.25) arc(-90:90:0.5);

\fill[gray,draw=black](2,-0.75) circle(0.05)node[right,black]{$\bar{z}_0$};
\draw[semithick,red,->,>=stealth',dashed](2,-1.25) arc(-90:-270:0.5);

%\fill[blue!20](0.5,1)--(0.5,0.05)--(2,0.05)--(2,0.75);

%\draw[semithick] (0.5,1.25)--(0.5,0.05)--(2,0.05)--(2,1.25);
%\draw[blue, semithick] (0.5,1)--(2,0.75);
%\draw[blue,dashed](2,0.75)--(0.75,0.75);
%\draw (1,0.75) arc(180:170:1) node[blue,above]{$\theta$};
%\draw [semithick,->,>=stealth',blue] (2.1,0.5)--(2.75,0.5) node[right]{$a$};



\end{tikzpicture}

\captionof{figure}{点涡的镜像}\label{fig:RectPointVortex}
\end{center}
\end{multicols}
\end{solution} 

\invisiblesubsection{\textcolor{blue}{张角的保角变换}}
\begin{problem}[问题6.6]
Z平面内, 有张角为$\alpha$的角域, $z_0=ae^{i\frac{\alpha}{2}}$处有强度为$Q$的点源, 求复速
度势.
\end{problem}

\begin{solution}
\textbf{解:} 利用倒解变化将该问题变换为上半平面, 图\ref{p6f1}为保角变换坐标轴示意图, 图\ref{p6f2}为变换前后对应点的关系.
\begin{figure}[!htb]
\begin{minipage}[b]{.65\textwidth}
\centering
\usetikzlibrary{%
    decorations.pathreplacing,%
    decorations.pathmorphing,arrows
}
\begin{tikzpicture}[ media/.style={font={\footnotesize\sffamily}},
    interface/.style={
        postaction={draw,decorate,decoration={border,angle=-45,
                    amplitude=0.3cm,segment length=2mm}}},scale=1.5]
%\draw(0.7,-0.22) rectangle (6.7,1.425);
\clip(0.7,-0.22) rectangle (6.7,1.425);
\draw[semithick,interface](2.3,1.3)--(1,0)--(3,0);
\draw[semithick,->,>=stealth'](3,0)--(3.25,0) node[right]{$x$};
\draw[semithick,->,>=stealth'](0.75,0)--(1,0) (1,0)--(1,1.4) node[below left]{$y$};

\draw[semithick,dashed,blue](1,0)--(3,0.8284);

\fill[blue,draw=gray](2.5,0.6213) circle(0.05) node[above,black]{$Q$};

\fill[black,draw=gray](1,0)circle(0.05)node[below left]{$o$};
\fill[black,draw=gray](2,1) circle(0.05) node[below right,black]{$B$};
\fill[black,draw=gray](1.9,0) circle(0.05) node[above,black]{$A$};
\fill[black,draw=gray](2.8,0) circle(0.05) node[above,black]{$C$};
\draw[very thick,blue,->,>=stealth'](3.5,1)--(4,1);

\begin{scope}[xshift=90]
\draw[semithick,interface](1,0)--(3,0);
\draw[semithick,->,>=stealth'](3,0)--(3.25,0) node[right]{$\xi$};
\draw[semithick,->,>=stealth'](2,0)--(2,1.4) node[below left]{$\eta$};

\fill[black,draw=gray](2,0)circle(0.05)node[above left]{$o$};
\fill[blue,draw=gray](2,0.8) circle(0.05) node[above right,black]{$Q$};
\fill[black,draw=gray](2.5,0) circle(0.05) node[above,black]{$A$};
\fill[black,draw=gray](3,0) circle(0.05) node[above,black]{$C$};
\fill[black,draw=gray](1.25,0) circle(0.05) node[above,black]{$B$};
\end{scope}
\end{tikzpicture}

\vspace{-0.25em}
\caption{\label{p6f1}保角变换示意图}
\end{minipage}%
\begin{minipage}[b]{.35\textwidth}
\centering
\begin{tabular}{l|lll}
\multicolumn{1}{c|}{} & \multicolumn{1}{c}{$z$} & \multicolumn{1}{c}{$\alpha_i$} & \multicolumn{1}{c}{$a_i$} \\
\hline
\multicolumn{1}{c|}{A} & \multicolumn{1}{c}{1} & \multicolumn{1}{c}{$\pi$} & \multicolumn{1}{c}{1} \\
\multicolumn{1}{c|}{B} & \multicolumn{1}{c}{$\infty e^{i\alpha}$} & \multicolumn{1}{c}{$\pi$} & \multicolumn{1}{c}{$-\infty$} \\
\multicolumn{1}{c|}{C} & \multicolumn{1}{c}{$\infty$} & \multicolumn{1}{c}{$\pi$} & \multicolumn{1}{c}{$\infty$} \\
\multicolumn{1}{c|}{O} & \multicolumn{1}{c}{0} & \multicolumn{1}{c}{$\alpha$} & \multicolumn{1}{c}{0} \\
\end{tabular}
\caption{\label{p6f2}变换前后对应点的关系}
\end{minipage}
\end{figure}

\noindent 由Schwarz-christoffel变换
\begin{eqnarray}
\frac{dz}{d\zeta} &=& K\Pi_{i=1}^N(\zeta-a_i)^{\beta_i}, {~~} \beta_i = \frac{\alpha_i}{\pi}-1\nonumber\\
&=& K\zeta^{\alpha/\pi-1}\nonumber
\end{eqnarray}
积分可得$z=K\zeta^{\alpha/\pi}$或$\zeta=(z/K)^{\pi/\alpha}$,由$A,B,C,O$可得$K=1$, 代入$Q$点
\[
\zeta_0 = z_0^{\pi/\alpha} = a^{\pi/\alpha}e^{i\pi/2} = ia^{\pi/\alpha}
\]
又由镜像法可得$\xi-\eta$平面上的复速度势
\[
W(\zeta) = \frac{Q}{2\pi}\ln(\zeta-\zeta_0) + \frac{Q}{2\pi}\ln(\zeta-\overline{\zeta_0})
\]
因此$z$平面上的复速度势
\begin{eqnarray}
W(z) &=& \frac{Q}{2\pi}\ln(z^{\pi/\alpha}-ia^{\pi/\alpha}) + \frac{Q}{2\pi}\ln(z^{\pi/\alpha}+ia^{\pi/\alpha})\nonumber\\
&=&\frac{Q}{2\pi}\ln(z^{2\pi/\alpha} + a^{2\pi/\alpha})\nonumber
\end{eqnarray}
\end{solution} 

\invisiblesubsection{\textcolor{blue}{无限长狭缝的保角变换}}
\begin{problem}[问题6.7]
$Z$平面上, 宽$a$的无限长狭缝, 对称中心存在强度为$Q$的点源, 求复速度势.
\end{problem}
\begin{solution}
\textbf{解:} 图\ref{p7f1}为保角变换坐标轴示意图, 图\ref{p7f2}为变换前后对应点的关系.
\begin{figure}[!htb]
\begin{minipage}[b]{.65\textwidth}
\centering
\usetikzlibrary{%
    decorations.pathreplacing,%
    decorations.pathmorphing,arrows
}
\begin{tikzpicture}[ media/.style={font={\footnotesize\sffamily}},
    interface/.style={
        postaction={draw,decorate,decoration={border,angle=-45,
                    amplitude=0.3cm,segment length=2mm}}},scale=1.5]
%\draw(0.7,-0.22) rectangle (6.7,1.425);
\clip(0.7,-0.22) rectangle (6.7,1.425);
\draw[semithick,interface](3,1)--(1,1) (1,0)--(3,0) ;
\draw[semithick,->,>=stealth'](3,0)--(3.25,0) node[right]{$x$};
\draw[semithick,->,>=stealth'](2,0)--(2,1.4) node[below left]{$y$};
\fill[black,draw=gray](2,0) circle(0.05) node[above left]{$o$} node[above right]{$B$};
%\draw[semithick,dashed,blue](1,0)--(3,0.8284);

\fill[blue,draw=gray](2,0.5) circle(0.05) node[right,black]{$Q$};

\fill[black,draw=gray](3,0) circle(0.05) node[above]{$A$};


\fill[black,draw=gray](2,1) circle(0.05) node[below right]{$C$};
\fill[black,draw=gray](3,1) circle(0.05) node[below]{$D$};


\draw[very thick,blue,->,>=stealth'](3.5,0.75)--(4,0.75);

\begin{scope}[xshift=90]
\draw[semithick,interface](1,0)--(3,0);
\draw[semithick,->,>=stealth'](3,0)--(3.25,0) node[right]{$\xi$};
\draw[semithick,->,>=stealth'](2,0)node[above left]{$o$}--(2,1.4) node[below left]{$\eta$};
\fill[blue,draw=gray](2,0) circle(0.05); 
\node at (2.2,0.25)[above,black]{$\displaystyle\frac{Q}{2}$};
\fill[black,draw=gray](2.5,0) circle(0.05) node[above,black]{$B$};
\fill[black,draw=gray](3,0) circle(0.05) node[above,black]{$A$};

\fill[black,draw=gray](1.5,0) circle(0.05) node[above,black]{$C$};
\fill[black,draw=gray](1,0) circle(0.05) node[above,black]{$D$};

\end{scope}
\end{tikzpicture}

\vspace{-0.25em}
\caption{\label{p7f1}保角变换示意图}
\end{minipage}%
\begin{minipage}[b]{.35\textwidth}
\centering
\begin{tabular}{l|lll}
\multicolumn{1}{c|}{} & \multicolumn{1}{c}{$z$} & \multicolumn{1}{c}{$\alpha_i$} & \multicolumn{1}{c}{$a_i$} \\
\hline
\multicolumn{1}{c|}{A} & \multicolumn{1}{c}{$\infty$} & \multicolumn{1}{c}{$\pi$} & \multicolumn{1}{c}{$\infty$} \\
\multicolumn{1}{c|}{B} & \multicolumn{1}{c}{$0$} & \multicolumn{1}{c}{$\pi/2$} & \multicolumn{1}{c}{1} \\
\multicolumn{1}{c|}{C} & \multicolumn{1}{c}{$ai$} & \multicolumn{1}{c}{$\pi/2$} & \multicolumn{1}{c}{-1} \\
\multicolumn{1}{c|}{D} & \multicolumn{1}{c}{$\infty+ai$} & \multicolumn{1}{c}{$\pi$} & \multicolumn{1}{c}{$-\infty$} \\
\end{tabular}
\caption{\label{p7f2}变换前后对应点的关系}
\end{minipage}
\end{figure}

\noindent 如图\ref{p7f1}所示, 将无穷带域的右半部分用Schwarz-christoffel变换为上半平面
\begin{eqnarray}
\frac{dz}{d\zeta} &=& K\Pi_{i=1}^N(\zeta-a_i)^{\beta_i}, {~~} \beta_i = \frac{\alpha_i}{\pi}-1\nonumber\\
&=& K(\zeta+1)^{-1/2}(\zeta-1)^{-1/2}\nonumber\\
&=& K(\zeta^2-1)^{-1/2}\nonumber
\end{eqnarray}
积分可得$z=K\cosh^{-1}\zeta$或$\zeta=\cosh \frac{z}{K}$,由$A,B,C,D$得$K=a/\pi$, 代入$Q$点
\[
\zeta_0 = \cosh\frac{\pi}{a}\frac{ia}{2} = 0
\]
又由镜像法可得$\xi-\eta$平面上的复速度势
\[
W(\zeta) = \frac{Q/2}{2\pi}\ln(\zeta-\zeta_0) + \frac{Q/2}{2\pi}\ln(\zeta-\overline{\zeta_0}) = \frac{Q}{2\pi}\ln(\zeta)
\]
因此$z$平面上的复速度势
\begin{eqnarray}
W(z) &=& \frac{Q}{2\pi}\ln(\cosh \frac{\pi z}{a})
\end{eqnarray}


\end{solution}


\invisiblesubsection{\textcolor{blue}{点涡偶极子的复速度势}}
\begin{problem}[问题6.8]
在复平面上推导由两个极其靠近的反向点涡构成的偶极子的复速度势.
\end{problem}

\begin{solution}
\textbf{解:} 如图\ref{p8f1}平面上两个反向点涡, 其的复速度势可由各自的速度势叠加, 即
\[
W(z) = \frac{\Gamma}{2\pi i}\ln(z-a) + \frac{-\Gamma}{2\pi i}\ln(z+a)
= -\frac{2a\Gamma}{2\pi i}\frac{\ln(z-a)-\ln(z+a)}{(z-a)-(z+a)}
\]
两个点涡构成的偶极子, 则$a\rightarrow 0$, 因此上式有
\begin{eqnarray}
\lim_{a\rightarrow 0} W(z) &=& -\frac{2a\Gamma}{2\pi i}\lim_{a\rightarrow 0}\frac{\ln(z-a)-\ln(z+a)}{(z-a)-(z+a)}
= -\frac{a\Gamma}{\pi i}\frac{1}{z}\nonumber\\
&=& -\frac{a\Gamma}{\pi i}(\cos\theta - i\sin\theta)= \frac{a\Gamma}{\pi r}\sin\theta + i\frac{a\Gamma}{\pi r}\cos\theta\nonumber
\end{eqnarray}
由此可以作出势函数和流函数等值线, 如图\ref{p8f2}所示, 程序见附录\ref{sec:cPotentStream}.
\begin{figure}[!htb]
\begin{minipage}[b]{.5\textwidth}
\centering
%\includegraphics[width=0.8\textwidth]{./figures/p8f1.pdf}
\usetikzlibrary{%
    decorations.pathreplacing,%
    decorations.pathmorphing,arrows
}

\begin{tikzpicture}[ media/.style={font={\footnotesize\sffamily}},
    interface/.style={
        postaction={draw,decorate,decoration={border,angle=-45,
                    amplitude=0.3cm,segment length=2mm}}},scale=1.5]

\clip(-2,-2.4)rectangle(2.4,2.1);
\draw[semithick,->,>=stealth'](-2,0)--(2,0) node[right]{$x$};
\draw[semithick,->,>=stealth'](0,-1.8)--(0,1.6) node[above]{$y$};



\fill[gray,draw=blue](-1,0) circle(0.05) node[below,blue]{$-a$};
\draw[semithick,red,->,>=stealth'](-1,-0.5) arc(-90:90:0.5);

\fill[gray,draw=blue](1,0) circle(0.05)node[below,blue]{$a$};
\draw[semithick,red,->,>=stealth'](1,-0.5) arc(-90:-270:0.5);

\node[blue] at (1,1){$a\rightarrow 0$};

%\fill[blue!20](0.5,1)--(0.5,0.05)--(2,0.05)--(2,0.75);

%\draw[semithick] (0.5,1.25)--(0.5,0.05)--(2,0.05)--(2,1.25);
%\draw[blue, semithick] (0.5,1)--(2,0.75);
%\draw[blue,dashed](2,0.75)--(0.75,0.75);
%\draw (1,0.75) arc(180:170:1) node[blue,above]{$\theta$};
%\draw [semithick,->,>=stealth',blue] (2.1,0.5)--(2.75,0.5) node[right]{$a$};



\end{tikzpicture}

\caption{\label{p8f1}两个点涡构成的偶极子示意图}
\end{minipage}%
\begin{minipage}[b]{.5\textwidth}
\centering
\pgfplotsset{compat=1.7}
\begin{tikzpicture}
\begin{axis}[width=0.9\textwidth,height=0.9\textwidth,
xlabel={$x$},
ytick={-5,0,5},xtick={-5,0,5},
%yticklabels={0.035,0.1,0.2,0.3,0.4,0.5},
ylabel={$y$},
xmin=-5,xmax=5,ymin=-5,ymax=5, 
yticklabel style={font=\scriptsize},
xticklabel style={font=\scriptsize},
xlabel style={font=\footnotesize},
ylabel style={font=\footnotesize},
legend style={font=\scriptsize,legend cell align=left},
]



\addplot[blue,domain=-5:5,samples=2,dashed]{0};
\addplot[red] coordinates {(0,-5) (0,5)};
\addlegendentry{$\varphi = K_1$};
\addlegendentry{$\psi = K_2$};

\foreach \r in {0.5,1, 1.5, 2.25, 3.5, 5, 7, 10, 20}{
  \edef\temp{\noexpand\draw[blue,dashed] (axis cs:0,\r) circle(\r);}
  \temp
  \edef\temp{\noexpand\draw[blue,dashed] (axis cs:0,-\r) circle(\r);}
  \temp
  \edef\temp{\noexpand\draw[red] (axis cs:{\r},0) circle(\r);}
  \temp
  \edef\temp{\noexpand\draw[red] (axis cs:{-\r},0) circle(\r);}
  \temp
}


\end{axis}





\end{tikzpicture}

%\includegraphics[width=0.8\textwidth]{./figures/p8.pdf}
\caption{\label{p8f2}势函数和流函数等值线}
\end{minipage}
\end{figure}

\end{solution}

\newpage
\invisiblesubsection{蒙古包的拔力}
\begin{problem}[问题6.9]
设一蒙古包是一半径为$a$的半圆球形, 受到速度为$V_\infty$的大风袭击, 屋顶
承受一向上的升力, 有掀翻蒙古包的危险. 假设蒙古包内的压强为流体
的驻点压强, 求蒙古包受到的拔力.(假设流动为理想不可压缩无旋流)
\end{problem}
\begin{solution}
\begin{multicols}{2}
\noindent\textbf{解:} 建立如图\ref{fig:halfsphere}所示的极坐标系. 由轴对称无旋流动的圆球绕流结论知, 速度势为
\[
\varphi = v_\infty r\Big(1+\frac{a^3}{2r^3}\Big)\cos\theta
\]
因此有
\[
v_r = \frac{\partial \varphi}{\partial r} = v_\infty\Big(1-\frac{a^3}{r^3}\Big)\cos\theta
\]
\[
v_\theta = \frac{\partial \varphi}{r\partial\theta} =
-v_\infty r\Big(1+\frac{a^3}{2r^3}\Big)\sin\theta
\]
对于蒙古包表面有$r=a$, 因此蒙古包表面
\[
u_r = 0, {~} u_\theta = -\frac{3}{2}v_\infty\sin\theta
\]

\begin{center}
%\includegraphics[width=0.4\textwidth]{./figures/p9.pdf}

\usetikzlibrary{calc,fadings,decorations.pathreplacing}

\newcommand\pgfmathsinandcos[3]{%
  \pgfmathsetmacro#1{sin(#3)}%
  \pgfmathsetmacro#2{cos(#3)}%
}
\newcommand\LongitudePlane[3][current plane]{%
  \pgfmathsinandcos\sinEl\cosEl{#2} % elevation
  \pgfmathsinandcos\sint\cost{#3} % azimuth
  \tikzset{#1/.style={cm={\cost,\sint*\sinEl,0,\cosEl,(0,0)}}}
}
\newcommand\LatitudePlane[3][current plane]{%
  \pgfmathsinandcos\sinEl\cosEl{#2} % elevation
  \pgfmathsinandcos\sint\cost{#3} % latitude
  \pgfmathsetmacro\yshift{\cosEl*\sint}
  \tikzset{#1/.style={cm={\cost,0,0,\cost*\sinEl,(0,\yshift)}}} %
}
\newcommand\DrawLongitudeCircle[2][1]{
  \LongitudePlane{\angEl}{#2}
  \tikzset{current plane/.prefix style={scale=#1}}
   % angle of "visibility"
  \pgfmathsetmacro\angVis{atan(sin(#2)*cos(\angEl)/sin(\angEl))} %
  \draw[current plane] (\angVis:1) arc (\angVis:\angVis+180:1);
  \draw[current plane,dashed] (\angVis-180:1) arc (\angVis-180:\angVis:1);
}
\newcommand\DrawLatitudeCircle[2][1]{
  \LatitudePlane{\angEl}{#2}
  \tikzset{current plane/.prefix style={scale=#1}}
  \pgfmathsetmacro\sinVis{sin(#2)/cos(#2)*sin(\angEl)/cos(\angEl)}
  % angle of "visibility"
  \pgfmathsetmacro\angVis{asin(min(1,max(\sinVis,-1)))}
  \draw[current plane,black!60] (\angVis:1) arc (\angVis:-\angVis-180:1);
  \draw[current plane,dashed,black!60] (180-\angVis:1) arc (180-\angVis:\angVis:1);
}

%% document-wide tikz options and styles

\tikzset{%
  >=latex, % option for nice arrows
  inner sep=0pt,%
  outer sep=2pt,%
  mark coordinate/.style={inner sep=0pt,outer sep=0pt,minimum size=3pt,
    fill=black,circle}%
}



\begin{tikzpicture} % "THE GLOBE" showcase

\def\R{2.5} % sphere radius
\def\angEl{35} % elevation angle
\filldraw[ball color=white] (\R,0) arc (0:180:\R) arc(180:360:{\R} and 0.57*\R);
\foreach \t in {0,20,...,80} { \DrawLatitudeCircle[\R]{\t} }
%\foreach \t in {-120} { \DrawLongitudeCircle[\R]{\t} }
\draw[rotate=-18](0,\R) arc(90:218:{0.4*\R} and \R);
\draw[semithick,dashed](-\R,0)--(\R,0) (0,0)--(0,0.8*\R) (0,0)--(-0.5*\R,-0.5*\R) (0,0)--(0.4*\R,0.4*\R) (0.4*\R,0.4*\R)--(-0.2*\R,0.4*\R) (0,0)--(-0.25*\R,0.46*\R);
 %(0,0)--(-0.385*\R,0.1475*\R)--(0.39*\R,0.15*\R)--(0,0);
\draw[semithick,->] (\R,0)--(1.75*\R,0) node[right]{$x$};
\draw[semithick,->] (0,0.8*\R)--(0,1.5*\R) node[above]{$z$};
\draw[semithick,->] (-0.5*\R,-0.5*\R)--(-0.75*\R,-0.75*\R) node[below left]{$y$};

\fill[blue!20,draw=blue](0.4*\R,0.4*\R) circle(0.125) node[right=3pt,blue]{$dS$};
\draw[semithick,->] (0.4*\R,0.4*\R)--(\R,\R) node[above right]{$p$};
\draw[<->,semithick,blue] (0:0.75) arc(0:45:0.75);

\draw[<->,semithick,red] (90:0.75) arc(90:120:0.75);
\node[red] at (105:1){$\varphi$};

\node[blue] at (22.5:1) {$\theta$};


\draw[semithick,->] (1.75*\R,0.5)--(1.25*\R,0.5);
\draw[semithick,->] (1.75*\R,1)--(1.25*\R,1);
\draw[semithick,->] (1.75*\R,1.5)--(1.25*\R,1.5);
\draw[semithick,->] (1.75*\R,2.0)--(1.25*\R,2.0);

%\draw

\end{tikzpicture}
\captionof{figure}{蒙古包的极坐标系示意图}\label{fig:halfsphere}
\end{center}
\end{multicols}
\noindent 又由伯努力方程
\[
\frac{v_\infty^2}{2} + \frac{p_\infty}{\rho} = \frac{v_\theta^2}{2} + \frac{p_\theta}{\rho} \Longrightarrow p_\theta = p_\infty + \frac{\rho}{2}v_\infty^2\Big(1-\frac{9}{4}\sin^2\theta\Big)
\]
驻点压强为$p_\infty + \rho v_\infty^2/2$. 因此作用在蒙古包内外压强差
\[
p = p_\infty + \frac{\rho}{2} v_\infty^2 - p_\theta = \frac{9}{8}\rho v_\infty^2\sin^2\theta
\]
设蒙古包曲面为$S$, 则半球面(蒙古包)上的面积元$dS = a^2\sin\theta d\theta d\varphi$. 因此蒙古包受到的拔力为
\begin{eqnarray}
F_z &=& \iint p\sin\theta\sin\varphi dS\nonumber\\
    &=& \frac{9}{8}a^2\rho v_\infty^2\int_0^{\pi}\sin\varphi d\varphi  \int_0^{\pi}\sin^4\theta d\theta\nonumber\\
    &=& \frac{27}{32}\pi a^2 \rho v_\infty^2\nonumber
\end{eqnarray}
\end{solution} 


\newpage
\appendix
\appendixpage
\lhead{附录A: 程序}
\section{程序}


%
%\begin{subappendices}
\subsection[问题1.1程序]{问题1.1程序: \textattachfile[color=red]{./matlab/ShearExperiment.m}{点击可下载ShearExperiment.m}}
\label{sec:ShearExperiment}

%~~\attachfile{./matlab/ShearExperiment.m}
\matlabscript{./matlab/ShearExperiment}{}

\subsection[问题2.3程序]{问题2.3程序: \textattachfile[color=red]{./matlab/StreamPathStreak.m}{点击可下载StreamPathStreak.m}}
\label{sec:StreamPathStreak}
%~~\attachfile{./matlab/StreamPathStreak.m}

\matlabscript{./matlab/StreamPathStreak}{}

\subsection[问题6.*程序]{问题6.*程序: \textattachfile[color=red]{./matlab/cPotentStream.m}{点击可下载cPotentStream.m}}
\label{sec:cPotentStream}
%~~\attachfile{./matlab/plotSimplyFluid.m}
\matlabscript{./matlab/cPotentStream}{}

%\subsection[问题6.7程序]{问题6.7程序: \textattachfile[color=red]{./matlab/SqrtCos.m}{点击可下载SqrtCos.m}}
%\label{sec:SqrtCos}
%~~\attachfile{./matlab/SqrtCos.m}
%\matlabscript{./matlab/SqrtCos}{}
%\end{subappendices}

\newpage
\section{试题}
\lhead{附录B: 试题}  
\invisiblesubsection{2011年期末试题}                                        %
\problemlist{\bf 流体力学导论2011年期末试题\footnote{{\bf 说明:} 本试题是本人考试后立刻回忆出来的, 供后届同学参考.}}

\noindent{\bf 一. 概念题}
\begin{enumerate}
\item 小球在黏性流体中沿$x$方向平动, 球表面应力矢量分别为
\[
T_x = \frac{x}{a}p_0 -\frac{3}{2}\mu\frac{U}{a},{~~} T_y = \frac{y}{a}p_0,{~~} T_z = \frac{z}{a}p_0
\]
其中$a$为小球半径, $U$为运动速度, $p_0$是远场压强. 求小球受到的阻力.
\vspace{0.5em}
\item 已知某流动的速度分布为$\vec{v}=f(r)\frac{\vec{r}}{r}$. 试问若流动不可压缩, 则$f(r)$应取何种形式?
\vspace{0.5em}
\item 平面二维流动$|\vec{v}|=\sqrt{2y^2+x^2+2xy}$, 流线簇方程为$y^2+2xy=C$. 试找到速度分量的表达式.
\vspace{0.5em}
\item 若流动为下列动量式, 试问需要满足哪些假设
      \[
      \frac{\partial u_i}{\partial t} + u_j \frac{\partial u_i}{\partial x_j} = -\frac{1}{\rho}\frac{\partial p}{\partial x_i} + \frac{\mu}{\rho}\frac{\partial}{\partial x_i}\frac{\partial u_i}{\partial x_j}
      \]
\vspace{0.5em}
\item 什么条件下存在速度势函数$\varphi$? 什么条件下$\nabla^2\varphi=0$?
\vspace{0.5em}
\item 在对倾角为$\alpha$的斜面刷油漆时, 油漆会从高处流向低处, 若把油漆看作平面二维流动, 请建立物理模型, 给出定解条件.
\end{enumerate}

\vspace{3em}
\noindent{\bf 二. 计算题}
\begin{enumerate}
\item 已知一圆柱容器中装有3/4容积的水, 容器内径为10cm, 深20cm. 把容器放在旋转圆盘上, 使其转动, 问使水不溢出的最大转速.
\vspace{0.5em}
\item 不可压缩平面流动, 在复平面$z$内, 有一张角为$\pi/3$的角域, 在$z_0=e^{i\pi/12}$处有一强度为$\Gamma$的点涡, 在角点处有一点源, 其流量为$Q$. 求角域内的速度势函数.
\vspace{0.5em}
\item 已知有一平行于地面的均匀来流$v_\infty$, 垂直流经半径为$R$的半圆柱形暖棚, 设流动为理想不可压.
    \begin{enumerate}
    \item 求暖棚处流动的势函数和暖棚外表面压强分布
    \item 当暖棚顶部开有小窗时, 求单位长度的暖棚受到的合力.
    \end{enumerate}
\end{enumerate}

\newpage

\invisiblesubsection{2013年期末试题}  
\problemlist{\bf 流体力学导论2013年期末试题\footnote{{\bf 说明:}本试题由师弟沈文豪考试后立刻回忆出来的, 由于沈文豪的个人要求, 本试题仅供后届师妹参考, 师弟请跳过本页.}}

\noindent{\bf 一. 概念题}

\begin{enumerate}
\item 一些微小玻璃颗粒漂浮在盛有某溶液的玻璃杯中, 求其分布情况.
\item $\displaystyle \frac{\partial u}{\partial t} + u \frac{\partial u}{\partial x} 
= -\frac{1}{\rho}\frac{\partial p}{\partial x} + \frac{\partial}{\partial x_j} \bigg(\mu \frac{\partial u}{\partial x_j}\bigg)$,求其假设条件.

\item $u=u(x,z)$, $w=w(y,z)$ (具体函数形式记不清了), 三维不可压缩流动, 求: 
      \begin{itemize}
      \item 速度$v$;
      \item 是否有旋; 
      \item 求加速度$a_x$.
      \end{itemize}
\item 涂有油的方形木板在涂有水的斜坡上滑动, 建立模型, 给出定解条件.
\end{enumerate}

\noindent{\bf 一. 计算题}

\begin{enumerate}
\item 半圆柱$r = 10\textrm{cm}$, $h=20\textrm{cm}$ ,装有3/4的水,以角速度$\omega$转动,求不使水溢出的最大角速度.
\item $\pi/4$角域在$z_0=2\mathrm{e}^{i\pi/12}$处有一强度为$Q$的点源, 求其速度势函数.
\item 理想不可压缩无旋定常有环量圆柱绕流, $V_\infty$, $r=a$, $\Gamma=\frac{\pi a V_\infty}{2}$
      \begin{itemize}
      \item 其速度势函数, 圆柱表面的速度分布和压强系数$C_p$.
      \item 在圆柱上安装皮托管, 一端测总压, 一端测静压, 求开口位置在何处.
      \end{itemize}
\end{enumerate}

\end{document} 
